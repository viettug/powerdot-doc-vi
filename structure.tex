%\section{Presentation structure}
\section{\texorpdfstring{Cấu trúc trình diễn}{Cau truc trinh dien}}

\label{sec:structure}

% =====================================================================

%\subsection{Making sections}
\subsection{\texorpdfstring{Tạo mục}{Tao muc}}
\label{sec:section}

\DescribeMacro{\section}
%% This section describes the |\section| command which provides a way
%% to structure a presentation.
Lệnh |\section| cho phép tạo một mục mới trong trình diễn, tương tự
như cách dùng lệnh |\section| của lớp \pf{article}.
\begin{command}
 `\cs{section}\oarg{options}\marg{section title}'
\end{command}
%% This command will produce a slide with \meta{section title} on it
%% and will also use this text to create sections in the table of
%% contents and in the bookmarks list. There are several \meta{options}
%% to control its output.
Lệnh này sẽ tạo một |slide| chỉ với nội dung là (tựa đề) \meta{section title}
(hãy xem thêm về tuỳ chọn |slide| ở bên dưới).
Tựa đề \meta{section title} cũng sẽ xuất hiện ở Bảng Mục lục và danh sách |bookmark|.
Có vài tuỳ chọn cho lệnh này như sau:

\DescribeOption{tocsection}
%% This option controls the creation of a
%% section in the table of contents. The default value is |true|.
Tuỳ chọn điều khiển việc tạo phần tử tương ứng cho |section| ở Bảng Mục lục.
Giá trị mặc định |true|.
\begin{description}
\item\option{tocsection=true}\\
%% This does create a section in the table of contents. This means that
%% all following slides, until the next section, will be nested under
%% this section.
Tạo mục tương ứng với |section| hiện tại trong Bảng Mục lục.
\item\option{tocsection=false}\\
%% This does not create a section in the table of contents and hence
%% the section will be listed as an ordinary slide.
Chỉ tạo |slide| cho mục nhưng không tạo phần tử tương ứng trong trang Mục lục.
\item\option{tocsection=hidden}\\
%% This does create a section in the table of contents, but this is
%% only visible when you view a slide that is part of this section.
%% This could be used to append a section to the presentation which you
%% can discuss if there is some extra time.
Tạo mục tương ứng trong Bảng Mục lục, nhưng mục này chỉ nhìn thấy khi bạn
đang xem một trong các |slide| của Mục đang xét. Điều này có ích, chẳng hạn khi
bạn có thêm thời gian để trình bày mục này (một cách dự trữ!).
\end{description}

\DescribeOption{slide}
%% This option controls whether the |\section|
%% command creates a slide. The default value is |true|.
Tuỳ chọn này cho phép hay không lệnh |\section|
tạo riêng cho nó một |slide|. Mặc định là |true|.
\begin{description}
\item\option{slide=true}\\
% A slide is created.
Một |slide| được tạo ra với nội dung là tựa của mục.
\item\option{slide=false}\\
%% No slide will be created. If also |tocsection| is |false|, the
%% |\section| command doesn't do anything. If it does create a table of
%% contents section (|tocsection=| |true| or |hidden|), its link will
%% point to the first slide in the section as the section itself
%% doesn't have a slide.
Không |slide| nào được tạo ra khi gặp lệnh |\section|.
Nếu đồng thời |tocsection| nhận giá trị |false|,
thì lệnh |\section| không làm gì cả. Nếu |tocsection| nhận giá trị |true|
hoặc |hidden|, thì chọn mục tương ứng trong Bảng Mục lục, ta sẽ nhày đến
|slide| đầu tiên của mục (bởi không có |slide| riêng cho Mục).
\end{description}

\DescribeOption{template}
%% This option can be used to make the
%% section slide with another template. By default, a normal |slide|
%% environment is used to create the section slide, but if a style
%% offers other templates that could be used for this purpose (for
%% instance, the |wideslide| environment), then you can use this option
%% to select that template. See section~\ref{sec:styles} for an
%% overview of the available templates with every style.
Tuỳ chọn này cho phép Mục đang xét chọn một mẫu khác.
Theo mặc định, môi trường |slide| được dùng để tạo ra |slide| cho mục,
nhưng nếu bạn muốn mẫu khác được dùng cho mục đích này,
ví dụ |wideslide|, thì bạn có thể dùng tuỳ chọn này để chỉ định mẫu đó.
Xem thêm ở Mục~\vref{sec:styles} để biết thêm chi tiết về các mẫu
của các kiểu khác nhau.

%% Finally, all options available to normal slides are available to
%% slides created by |\section| as well (see section~\ref{sec:slides}).
%% However, when the section does make a |tocsection|, |toc=| or |bm=|
%% won't remove the table of contents entry or the bookmark
%% respectively.
\medskip
Cuối cùng, mọi tuỳ chọn của |slide| đều có thể dùng cho |\section|,
ví dụ |toc=|, |bm=|,\ldots Xem thêm ở Mục~\vref{sec:slides}.

% =====================================================================

%\subsection{Making an overview}
\subsection{\texorpdfstring{Tạo slide Mục lục}{Tao slide Muc luc}}
\label{sec:tableofcontents}

\DescribeMacro{\tableofcontents}
%% This command creates an overview of your presentation and can only
%% be used on a slide.
Lệnh này tạo |slide| Mục lục giúp bạn có cái nhìn tổng quan về trình diễn
hoặc một phần (mục) của trình diễn.
Lệnh  này chỉ được dùng bên trong môi trường tạo |slide|,
và bạn có thể gọi nó bao nhiêu lần tuỳ thích.

Nội dung được nói đến trong mục này là nội dung của |slide| Mục lục.
Để ý rằng, |slide| Mục lục khác với Bảng Mục lục. Một trình diễn chỉ
có một Bảng Mục lục nhưng có thể có nhiều |slide| mục lục; hơn nữa,
thường thì Bảng Mục lục được tự động tạo ra.

\begin{command}
 `\cs{tableofcontents}\oarg{options}'
\end{command}
%% There are several \meta{options} to control the output of this
%% command.
Có vài tuỳ chọn cho lệnh như sau:

\DescribeOption{type}
%% This option controls whether certain material (depending on the
%% input in the |content| option below) will be hidden or displayed in
%% the inactive color\index{inactive color}. The default value is |0|.
%% Compare with the |type| option for list environments
%% (section~\ref{sec:lists}).
Xác định giấu hoặc tô nhạt một vài phần (phụ thuộc vào giá trị của |content| dưói đây).
Giá trị mặc định là |0|. So sánh với tuỳ chọn |type| của môi trường
tạo danh sách ở Mục~\vref{sec:lists}.

\begin{description}
\item\option{type=0}\\
%% When material is not of the requested type as specified in the
%% |content| option, it will be hidden.
Nếu nội dung không đúng kiểu như chỉ ra ở tuỳ chọn |content|, nó sẽ được
giấu đi.
\item\option{type=1}\\
%% As |type=0|, but instead of hiding material, it will be typeset in
%% the inactive color.
Như trên, nhưng thay vì giấu đi, nội dung sẽ được hiện với màu tô nhạt.
\end{description}

\DescribeOption{content}
%% The |content| option controls which elements will be included in the
%% overview. The default value is |all|. The description below assumes
%% that |type=0| was chosen, but the alternative text for |type=1| can
%% easily be deduced.
Tuỳ chọn này cho phép xác định những phần tử nào sẽ được thể hiện ở
|slide| mục lục. Giá trị mặc định là |all|. Mô tả dưới đây giả định rằng
|type=0| được chọn. Bạn có thể dễ dàng suy ra kết quả khi |type=1| từ mô tả này.

\begin{description}
\item\option{content=all}\\
%% This will display a full overview of your presentation including all
%% sections and slides, except the slides in hidden sections (see
%% section~\ref{sec:section}).
Cho ra |slide| đầy đủ, gồm mọi mục và |slide| trong trình diễn
của bạn, trừ các mục ẩn (xem mô tả trong Mục~\vref{sec:section}). 
\item\option{content=sections}\\
%This displays only the sections in the presentation.
Chỉ liệt kê các Mục của trình diễn.
\item\option{content=currentsection}\\
%This displays the current section only.
Chỉ liệt kê các |slide| của mục hiện tại.
\item\option{content=future}\\
%This displays all content starting from the current slide.
Liệt kê mọi mục và |slide| bắt đầu từ |slide| hiện tại.
\item\option{content=futuresections}\\
%This displays all sections, starting from the current section.
Liệt kê mọi mục bắt đầu từ mục hiện tại.
\end{description}

%% We finish this section with a small example that will demonstrate
%% how you can make a presentation that contains an overall overview of
%% sections in the presentation, giving a general idea of the content,
%% and per section a detailed overview of the slides in that section.
Dưới đây là ví dụ nhỏ. Trình diễn ở ví dụ này gồm các mục,
dầu mỗi mục là |slide| mục lục có nhiệm vụ tóm tắt (liệt kê) các
phần của Mục đó.
\begin{example}
 \begin{slide}[toc=,bm=]{Overview}
   \tableofcontents[content=sections]
 \end{slide}
 \section{First section}
 \begin{slide}[toc=,bm=]{Overview of the first section}
   \tableofcontents[content=currentsection,type=1]
 \end{slide}
 \begin{slide}{Some slide}
 \end{slide}
 \section{Second section}
 ...
\end{example}
\endinput
