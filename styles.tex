%% $Id %%

%\section{Available styles}
\section{\texorpdfstring{Các kiểu trình diễn}{Cac kieu trinh dien}}
\label{sec:styles}

Lớp \pf{powerdot}
%% comes with a number of styles which are listed in the
%% overview below. The characteristics of each style are described
%% shortly.
%<<1.3>>
%shortly and a sample of a title slide and a normal slide is provided
%for each style. Styles support the |wideslide| environment, have a
%table of contents on the left part of the paper in landscape
%orientation and on the bottom part in portrait orientation and
%support portrait orientation unless states otherwise.
được phân phối cùng với một số kiểu (|style|, xì-tin :) được mô tả
dưới đây. Bạn nên thử qua chúng để chọn lấy kiểu ưa thích.
Nếu không có ghi chú đặc biệt, ta hiểu các kiểu hỗ trợ môi trường |wideslide|,
bảng mục lục ở bên trái |slide| khi xem ở chế độ |landscape|,
ở bên dưới |slide| khi xem ở chế độ |portrait|.

Việc lựa chọn kiểu đã nói đến trong Mục~\vref{sec:classopts}.

\medskip
%% Có thể xem hình ảnh minh họa cho các kiểu này có trong tập tin
%% |powerdot-styles.zip| (tải về từ \url{http://download.viettug.org}).
Hình ảnh minh họa cho các kiểu này có thể xem trong tập tin\\
\centerline{\texttt{powerdot-\pdversion-styles-vn.zip}}
tải về từ \url{http://download.viettug.org}.

\begin{description}
\item\pf{default}\\
%% This style has as main colors light blue and white. A flower in the
%% top left corner decorates the slide. The style supports a
%% |wideslide| and portrait orientation. Slides have a table of
%% contents on the left part of the paper in landscape orientation and
%% on the bottom part in portrait orientation. The style requires the
%% \pf{pifont} package.
%%<<1.3>>
%% Kiểu mặc định, chỉ với màu xanh sáng và trắng. Một bông hoa ở góc trên
%% bên trái của mỗi |slide|. Kiểu này cung cấp môi trường |wideslide| và hướng
%% |portrait|. Mỗi |slide| có Bảng Mục lục ở bên trái (chế dộ |landscape|)
%% hoặc bên dưới (chế độ |portrait|). Kiểu này đòi hỏi gói \pf{pifont}.
%<<1.3>>
%This style provides six different palettes. A flower in the top left
%corner decorates the slides for all palettes. The default palette is
%\texttt{blue} which has as main colors light blue and white. You can
%see an example of that palette below. Other available palettes are
%\texttt{red}, \texttt{green}, \texttt{yellow}, \texttt{brown} and
%\texttt{purple}.\\
Kiểu này cung cấp sáu |palette| khác nhau, gồm |red|, |green|, |yellow|
|brown| và |purple|; mặc định là |palette| màu xanh (|blue|) với hai màu chính
là xanh sáng và trắng.
\\\styleexample{default}
\item\pf{simple}\\
%% This is a simple style in black and white. The style supports a
%% |wideslide| and portrait orientation. A table of contents is present
%% on slides at the left hand side in landscape mode and in the bottom
%% of the slide in portrait mode. It requires the \pf{amssymb} and
%% \pf{pifont} packages.
Kiểu đơn giản chỉ với hai màu đen, trắng.
%%  Hỗ trợ môi trường |wideslide|
%% và hướng |portrait|. Bảng Mục lục được bố trí bên trái (|landscape|) hoặc
%% bên dưới (|portrait|) của |slide|. Kiều này cần gói \pf{amssymb}
%% và gói \pf{pifont}.
Kiể này có ích khi bạn muốn in trình diễn của mình.
\\\styleexample{simple}
\item\pf{tycja}\\
%% This style is set in shades of yellow and dark blue. The table of
%% contents on slides is on the right side of the paper in landscape
%% orientation and on the bottom part in portrait.\\
Kiểu này được thiết kế với đường viền vàng và xanh sẫm.
Bảng Mục lục được bố trí bên phải (chế độ |landscape|) hoặc bên dưới
(chế độ |portrait|).
\\\styleexample{tycja}
\item\pf{ikeda}\\
%% This style uses dark shades of red and blue and a light text color.
%% It has nice patterns on the slide for decoration.
Đường viền màu đỏ và xanh tối, màu chữ sáng.
Kiểu có mẫu trang trí khá đẹp.
\\\styleexample{ikeda}
\item\pf{fyma}\\
%% This style was originally created by Laurent Jacques for
%% \pf{prosper}. Based on that style, he created a version for
%% \pf{HA-prosper} with extended features. With his kind permission,
%% this style has been converted by Shun'ichi J. Amano for
%% \pf{powerdot}. The style has an elegant design with a light blue and
%% white gradient background in the default \texttt{blue} palette.
%% Other available palettes are \texttt{green}, \texttt{gray},
%% \texttt{brown} and \texttt{orange}. It has special templates for
%% sections on slides and sections on wide slides. Below is a sample of
%% the blue palette.\\
Nguyên được viết cho \pf{propser} bởi Laurent Jaques.
Dựa trên bản đó, L. Jaques viết một bản tương ứng cho \pf{HA-prosper}
với vài tính năng mở rộng. Kiểu này được Shun'ichi J. Amano 
viết lại cho \pf{powerdot}. Kiểu có màu xanh sáng, với màu nền
|gradient| trắng. Các |palette| được hỗ trợ: |green|, |gray|,
|blue| (mặc định), |brown| và |orange|. Kiểu này có một |template|
(mẫu) đặc biệt để bố trí các mục trên |slide|.
\\\styleexample{fyma}
\item\pf{ciment}\\
%% This style was originally created by Mathieu Goutelle for
%% \pf{prosper} and \pf{HA-prosper}. With his kind permission, this style
%% has been converted for \pf{powerdot}. The style has a background
%% that is hatched with light gray horizontal lines. Titles and table
%% of contents highlighting are done with dark red.
Kiểu này nguyên được viết bởi Mathieu Goutelle cho \pf{prosper} và \pf{HA-prosper}.
Được phép của tác giả, kiểu này được viết lại cho \pf{powerdot}.
Kiểu có màu nền có các đường thẳng ngang màu sáng bạc.
Tiêu đề và trang mục lục có màu đỏ tối.
\\\styleexample{ciment}
\item\pf{elcolors}\\
%% This is a style using light shades of the elementary colors red,
%% blue and yellow.
Đường viền màu sáng, các màu cơ bản gồm đỏ, xanh và vàng.
\\\styleexample{elcolors}
\item\pf{aggie}\\
%% This style was created by Jack Stalnaker for \pf{HA-prosper} and he
%% has converted this style for \pf{powerdot}. The style uses dark red
%% and light brown colors.
Tác giả Jack Stalnaker đã từ chuyển kiểu tương ứng ông viết cho \pf{HA-propsper}.
Kiểu có màu đỏ tối và nâu sáng.
\\\styleexample{aggie}
\item\pf{husky}\\
%% This style is created by Jack Stalnaker and has a background of
%% light gray sun beams combined with dark red highlights.
Tác giả là Jack Stalnaker.% Kiểu có màu nền sáng bạc
\\\styleexample{husky}
\item\pf{sailor}\\
%% This style is contributed by Mael Hill\'ereau and supplies five
%% different palettes: \texttt{Sea} (the default), \texttt{River},
%% \texttt{Wine}, \texttt{Chocolate} and \texttt{Cocktail}. Below is a
%% sample of the palette \texttt{Sea}.
Kiểu này được đóng góp bởi Mael Hill\'ereau. Các |palette| màu
gồm có: |Sea| (mặc định), |River|, |Wine|, |Chocolate| và |Cocktail|.
\\\styleexample{sailor}
\item\pf{upen}\\
%% This style has a nice dark blue background and text in yellow. It is
%% contributed by Piskala Upendran.
Đóng góp bởi Piskal Upendran. Kiểu có  màu nền xanh tối, chữ vàng.
\\\styleexample{upen}
\item\pf{bframe}\\
%% The \pf{bframe} style has blue frames on the slide in which text is
%% positioned. The style is contributed by Piskala Upendran.
Đóng góp bởi Piskala Upendran. Kiểu có các khung viền màu xanh.
\\\styleexample{bframe}
\item\pf{horatio}\\
%% The \pf{horatio} style has been contributed by Michael Lundholm and
%% is a more conservative blue style.
Đóng góp bởi Michael Lundholm.
\\\styleexample{horatio}
\item\pf{paintings}\\
%% This is a simple style without a table of contents on slides. It has
%% been contributed by Thomas Koepsell and provides 10 different
%% palettes. The colors used in the palettes are drawn from famous
%% paintings.\footnote{The style defines a color \texttt{pdcolor7}
%% which is not used in the style but comes from the same painting and
%% complements the other colors. It can be used, for example, to
%% highlight text against the main background color.} If you are
%% interested, open the style file to read which paintings have been
%% used. The available palettes are: \texttt{Syndics} (the default),
%% \texttt{Skater}, \texttt{GoldenGate}, \texttt{Lamentation},
%% \texttt{HolyWood}, \texttt{Europa}, \texttt{Moitessier},
%% \texttt{MayThird}, \texttt{PearlEarring} and \texttt{Charon} (all
%% case sensitive). Below is a sample of the \texttt{Syndics}
%% palette.
Đây là kiểu đơn giản, đóng góp bởi Thomas Koepsell, không có bảng Mục lục.
Gồm 10 |palette| khác nhau, các màu của |palette| được lấy từ các tác phẩm
hội họa nổi tiếng\footnote{%
Kiểu này còn cung cấp màu \texttt{pdcolor7}. Màu này thực ra không dùng trong
thiết kế của kiểu, nhưng bổ sung cho các màu khác và được lấy cũng từ các tác phẩm
hội họa. Màu này có ích, chẳng hạn khi bạn muốn đánh dấu một đoạn văn, làm
nó nổi bật so với màu nền}. Nếu thích, bạn có thể mở tập tin định nghĩa kiểu này
để biết tác phẩm tương ứng với mỗi |palette|. Các |palette| đó là:
|Syndics| (mặc định), |Skater|, |GoldenGate|, |Lamentation|,
|HolyWood|, |Europa|, |Moitessier|, |MayThird|, |PearlEarring| và |Charon|.
\\\styleexample{paintings}
\item\pf{klope}\\
%% The \pf{klope} style implements a horizontal table of contents that
%% only lists the sections. The style is available in the following
%% palettes: \texttt{Spring}, \texttt{PastelFlower}, \texttt{BlueWater}
%% and \texttt{BlackWhite}. The \texttt{Spring} palette is the default
%% and you can see a sample of that below.
Kiểu này bố trí bảng Mục lục nằm ngang, trong đó chỉ liệt kê các Mục lớn.
Các |palette| được hỗ trợ: |Spring| (mặc định), |PastelFlower|, |BlueWater|
và |BlackWhite|.
\\\styleexample{klope}
%\item\pf{jefka}\\
%The \pf{jefka} style comes with four palettes: \texttt{brown} (the
%default), \texttt{seagreen}, \texttt{blue} and \texttt{white}. Below
%you see a sample of the \texttt{brown} palette.\\
%\styleexample{jefka}
%\item\pf{pazik}\\
%This style is available in two palettes: \texttt{red} and
%\texttt{brown}. Below is a sample of the default \texttt{red}
%palette.\\
%\styleexample{pazik}
\item\pf{jefka}\\
Các |palette| gồm có: |brown| (mặc định), |seagreen|, |blue| và |white|.
\\\styleexample{jefka}
\item\pf{pazik}\\
Các |palette| gồm có: |red|, |brown|.
\\\styleexample{pazik}
\end{description}
\endinput
