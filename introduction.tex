\section{\texorpdfstring{Giới thiệu}{Gioi thieu}}
\label{sec:intro}

%% This class gives you the possibility to easily create professionally
%% looking slides. The class is designed to make the development of
%% presentations as simple as possible so that you can concentrate on
%% the actual content instead of keeping yourself busy with technical
%% details. Of course, some knowledge of \LaTeX\ is still required
%% though.
Lớp \pf{powerdot} cung cấp khả năng tạo trình diễn dễ dàng, chuyên nghiệp.
Lớp được thiết kế để việc thiết kế trình diễn trở nên đơn giản nhất có thể,
nhờ đó bạn không phải tốn thời gian với các yếu tố kỹ thuật.
Tất nhiên, bạn phải có các kiến thức cơ bản về \LaTeX{}.

%% This class builds on and extends the \pf{prosper} class
%% \cite{prosper} and the \pf{HA-prosper} package \cite{HA-prosper}.
%% The \pf{HA-prosper} package was initially intended to extend
%% \pf{prosper} and correct some bugs and problems of that class. As
%% developments on that package progressed, it was found that
%% unfortunately, not all of the problems could be overcome with the
%% package. That discovery was the start of a new project set up to
%% make a new class to replace the \pf{prosper} plus \pf{HA-prosper}
%% combination. You're currently reading the result of that project.
Lớp được xây dựng nhờ mở rộng lớp \pf{prosper} \cite{prosper}
và gói \pf{HA-prosper} \cite{HA-prosper}. Gói \pf{HA-prosper}
có mục đích ban đầu là mở rộng và khắc phục vài lỗi, nhược điểm của lớp
\pf{prosper}. Thật không may là, không phải mọi nhược điểm của \pf{propser}
đều có thể khắc phục được. Chính vì lý do này, một dự án mới ra đời
nhằm thay thế cho cả \pf{prosper} và \pf{HA-prosper}.
Bạn đang đọc tài liệu về chính dự án đó ;)

%% The remainder of this section will be devoted to giving a feel of
%% what the \pf{powerdot} presentation source looks like and giving an
%% overview of this documentation.
Phần còn lại của mục này giúp bạn có cái nhìn tổng quan về lớp \pf{powerdot}
và tài liệu hướng dẫn này.

%% The document structure of a presentation is always the same. You can
%% find it in the example below.
Cấu trúc của trình diễn luôn tương tự như ví dụ sau đây:

\begin{example}
 \documentclass[<class options>]{powerdot}
 \pdsetup{<pd options>}
 \begin{document}
   \begin{slide}{slide}
     noi dung
   \end{slide}
   \section{section}
   \begin{slide}[<slide options>]{slide}
     noi dung
   \end{slide}
   \begin{note}{ghi chu ca nhan}
     ghi chu
   \end{note}
 \end{document}
\end{example}

%% There are several elements that define the document structure. First
%% of all, the class accepts some class options that control the output
%% of the class, for instance, paper type and style. These class
%% options will be discussed in section~\ref{sec:classopts}. Then there
%% are presentation specific options which control some of the elements
%% of the presentation globally, for instance, the footers. These will
%% be discussed in section~\ref{sec:pdsetup}.
Có vài yếu tố tạo nên cấu trúc đó.
Đầu tiên, lớp chấp nhận vài tuỳ chọn (|class options|) cho phép điều khiển kết quả xuất
ví dụ, cỡ giấy, kiểu. Các tuỳ chọn này được bàn kỹ đến trong Mục~\vref{sec:classopts}.
Thứ đến, là các tuỳ chọn trình diễn (|pd options|) điều khiển toàn cục các
tính chất của trình diễn, ví dụ, các ghi chú ở chân trang.
Những tuỳ chọn này đưọc nói đến ở Mục~\vref{sec:pdsetup}.

%% Once the setup has been decided on, you can use the slide
%% environment to produce slides (see section~\ref{sec:slides}) and the
%% note environment to produce notes that go with the slides (see
%% section~\ref{sec:notes}). You can use overlays to display material
%% in steps. This is described in section~\ref{sec:overlays}. The
%% |\section| command provides a way to structure your presentation.
%% This is discussed in section~\ref{sec:structure}.
%% Section~\ref{sec:styles} will show an overview of the styles that
%% come with this class and the characteristics of each style.
%% Section~\ref{sec:compiling} will tell you more about how to produce
%% output. This section contains important information on required
%% packages.
Sau khi thiết lập với các tuỳ chọn, bạn có thể dùng môi truờng |slide|
để tạo các trang (|slide|) trình diễn (xem Mục~\vref{sec:slides}) và môi trường |note|
để tạo các ghi chú đi cùng với |slide| (xem Mục~\vref{sec:notes}).
Bạn có thể dùng |overlay| để thể hiện nội dung theo từng bước
(xem Mục~\vref{sec:overlays}). Lệnh |\section| giúp bạn tạo cấu trúc
cho trình diễn, giống như việc tạo chương, mục với tài liệu \LaTeX{} thông thường
(xem Mục~\vref{sec:structure}). Bạn cũng có thể lựa chọn các kiểu dáng
của trình diễn sau khi xem Mục~\vref{sec:styles}. Cuối cùng,
với Mục~\vref{sec:compiling}, bạn sẽ biết cách biên dịch tài liệu nguồn
để có kết quả là trình diễn thật sự. Ở mục này cũng có vài lưu ý về
việc cài đặt lớp \pf{powerdot}.

%% Section~\ref{sec:writestyle} is mostly interesting for people that
%% want to develop their own style for this class or want to modify
%% an existing style. This documentation concludes with a section
%% devoted to questions (section~\ref{sec:questions}), like `Where can
%% I find examples?'. It also tells you where to turn to in case your
%% questions are still not solved.
%%<<1.3>>
%% Section~\ref{sec:writestyle} is mostly interesting for people that
%% want to develop their own style for this class or want to modify an
%% existing style. Section~\ref{sec:lyx} explains how \LyX\
%% \cite{LyXWeb} can be used to create \pf{powerdot} presentations.

Mục~\vref{sec:writestyle} có lẽ là phần hấp dẫn đối với ai quan tâm
đến việc tạo kiểu dáng riêng cho trình diễn của mình, hoặc làm đẹp
các kiểu dáng đã có. Mục~\vref{sec:lyx} hướng dẫn tạo trình diễn bằng
|powerdot| với \LyX\ \cite{LyXWeb}.

Tài liệu này kết thúc với Hỏi-Đáp (Mục~\vref{sec:questions}), có thể
giúp bạn trong những bước đầu làm quen với lớp \pf{powerdot}.

\endinput
