%%
%% $Id$
%%
%% This file is part of `powerdot-doc-vi' bundle
%% See `powerdot-doc-vi.tex' for details
%%
\ifx\style\undefined
	\def\style{default}
\fi
\documentclass[
	style=\style,
	clock,
	size=11pt,
	paper=screen,
	orient=landscape
]{powerdot}

\usepackage[utf8x]{vietnam}

\title{Ví dụ kiểu \style}
\author{kyanh}
\date{\today}

% =====================================================================

\pddefinetemplate[slide]{slide}{tocpos}{}
\pdsetup{
	lf=left footer,rf=right footer,
	randomdots={dotstyle=ocircle},
	dminwidth=.4\slidewidth
}

% =====================================================================

\makeatletter
\def\textheightrule{\raisebox\baselineskip{\rule{1cm}\pd@@textheight}}
\makeatother

% =====================================================================

\begin{document}

\newpage

\maketitle

\begin{slide}{Ví dụ về slide}
  Công thức Newton:
  \begin{equation}\label{binomium}
    (a+b)^n=\sum_{k=0}^n{n\choose k}a^{n-k}b^k
  \end{equation}
  Ta sẽ chứng minh (\ref{binomium}) trên bảng đen.\\
  \begin{itemize}
    \item Đây
    \begin{itemize}
      \item là
      \begin{itemize}
        \item danh
        \begin{itemize}
          \item sách
        \end{itemize}
        \item với
      \end{itemize}
      \item bảy
    \end{itemize}
    \item phần tử
  \end{itemize}
\end{slide}

% =====================================================================

\begin{slide}{Một slide bình thường}
  Một chiều đi trên con đường này. Hoa điệp vàng trải dưới chân tôi.
  Ngập ngừng trong tôi như thầm hỏi. Đường về trường ôi sao lạ quá...

  Để ý đến màu của chỉ số phương trình!
  \begin{equation}
    (a+b)^n=\sum_{k=0}^n{n\choose k}a^{n-k}b^k
  \end{equation}
\end{slide}

\begin{slide}{Kiểm tra danh sách}
  Đường về trường ôi sao lạ quá...
  \pause
  \begin{itemize}
    \item mức 1\pause
    \begin{itemize}
      \item mức 2\pause
      \begin{itemize}
        \item mức 3\pause
        \begin{itemize}
          \item mức 4
        \end{itemize}
      \end{itemize}
    \end{itemize}
  \end{itemize}
  Đường về trường ôi sao lạ quá...
  \footnote{Thay đổi textheight để đặt footnote đúng chỗ.}
\end{slide}

\section{Mục bình thường}

\begin{slide}{Kiểm tra danh sách thứ tự}
  Đường về trường ôi sao lạ quá...
  \pause
  \begin{enumerate}[type=1]
    \item mức 1\pause
    \begin{enumerate}
      \item mức 2\pause
      \begin{enumerate}
        \item mức 3\pause
        \begin{enumerate}
          \item mức 4
        \end{enumerate}
      \end{enumerate}
    \end{enumerate}
  \end{enumerate}
  Đường về trường ôi sao lạ quá...
\end{slide}

\begin{slide}{Thước có độ cao \texttt{textheight}}
  \textheightrule
\end{slide}

\section[template=wideslide]{Mục với slide khổ rộng}

\begin{wideslide}{Kiểm tra wideslide}
  Một chiều đi trên con đường này. Hoa điệp vàng trải dưới chân tôi.
  Ngập ngừng trong tôi như thầm hỏi. Đường về trường ôi sao lạ quá...
  \footnote{Thay đổi textheight để đặt footnote đúng chỗ.}

  Để ý đến màu của chỉ số phương trình!
  \begin{equation}
    (a+b)^n=\sum_{k=0}^n{n\choose k}a^{n-k}b^k
  \end{equation}
\end{wideslide}

\begin{wideslide}{Thước có độ cao \texttt{textheight}}
  \textheightrule
\end{wideslide}

\end{document}
