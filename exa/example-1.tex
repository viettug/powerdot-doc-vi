\IfFileExists{./header.tex}{\input header.tex}{\documentclass[a4paper,style=default]{powerdot}}

%% \item\pf{default}\\
%% \item\pf{simple}\\
%% \item\pf{tycja}\\
%% \item\pf{ikeda}\\
%% \item\pf{fyma}\\

\title{powerdot example 1}
\author{kyanh, kyanh@o2.pl}
\date{\today}

\begin{document}

\maketitle

\begin{slide}{muc luc}
\tableofcontents
\end{slide}

\section{A}

\begin{slide}{Simple onslide+}
 \texttt{onslide }: \onslide{1}{power}\onslide{2}{dot}\\
 \texttt{onslide+}: \onslide+{1}{power}\onslide+{2}{dot}\\
 \texttt{onslide+}: \onslide*{1}{power}\onslide*{2}{dot}\\
\end{slide}

\begin{slide}{twocolumn}
\twocolumn{lineheight=1cm}%
	{asdfl;kjas;dfkjpoej;laksdjf;laksdjf;askdjf;lasjkdf}%
	{;asdkfj;askdjf ;klasjdf;lkj2;rja;sdlkfj;asdf}
\end{slide}

\begin{slide}{Slide 1}
  \begin{enumerate}[type=1]
    \item Here
      \item we
        \item demonstrate
          \item the enumerate environment
  \end{enumerate}
\end{slide}

\begin{slide}{Slide 2}
  \begin{enumerate}[type=0]
    \item<1-> Here
      \item<2-> we
        \item<3-> demonstrate
          \item<4-> the enumerate environment
  \end{enumerate}
\end{slide}

\section{B}

\begin{slide}{Slide 3}
  \begin{enumerate}[type=1]
    \item<1-> Here
      \item<2-> we
        \item<3-> demonstrate
          \item<4-> the enumerate environment
  \end{enumerate}
\end{slide}

\begin{slide}{Slide 4}
  \begin{enumerate}[type=1]
    \item<1> Here
      \item<2> we
        \item<3> demonstrate
          \item<4> the enumerate environment
  \end{enumerate}
\end{slide}

\section{C}

\begin{slide}{Slide 5}
  \begin{enumerate}[type=0]
    \item<1> Here
      \item<2> we
        \item<3> demonstrate
          \item<4> the enumerate environment
  \end{enumerate}
\end{slide}

\begin{slide}{Relative overlays}
   \begin{itemize}
     \item A \pause
     \item B \onslide{+1}{(visible 1 overlay after B)}\pause
     \item C \onslide{+2-}{(appears 2 overlays after C, visible until the end)}
     \pause
     \item D \onslide{+1-6}{(appears 1 overlay after D, visible until overlay 6)}
     \pause
     \item E \pause
     \item F \pause
     \item G \onslide{+1-+3}{(appears 1 overlay after G for 3 overlays)}\pause
     \item H \pause
     \item I \pause
     \item J \pause
     \item K
   \end{itemize}
 \end{slide}
\end{document}
