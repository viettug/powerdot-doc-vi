\IfFileExists{./header.tex}{\input header.tex}{\documentclass[a4paper,style=default]{powerdot}}

\newif\ifvntex
\IfFileExists{vietnam.sty}{\vntextrue}{}

\ifvntex
	\usepackage[utf8x]{vietnam}
	\long\def\text#1#2{#1}
\else
	\long\def\text#1#2{#2}	
\fi

\title{\text{Ví dụ về powerdot}{Powerdot example}}
\author{kyanh, kyanh@o2.pl}
\date{\today}

\begin{document}

\maketitle

\begin{slide}{\text{Mục lục}{Contents}}
\tableofcontents
\end{slide}

\section{\text{Mục A}{Section A}}

\begin{slide}{\text{Lệnh onslide}{Onslide command}}
 \texttt{onslide }: \onslide{1}{power}\onslide{2}{dot}\\
 \texttt{onslide+}: \onslide+{1}{power}\onslide+{2}{dot}\\
 \texttt{onslide*}: \onslide*{1}{power}\onslide*{2}{dot}\\
\end{slide}

\begin{slide}{\text{Chế độ hai cột}{Two columns}}
\twocolumn{lineheight=1cm}%
	{asdfl;kjas;dfkjpoej;laksdjf;laksdjf;askdjf;lasjkdf}%
	{;asdkfj;askdjf ;klasjdf;lkj2;rja;sdlkfj;asdf}
\end{slide}

\section{\text{Danh sách}{Danh sach}}

\begin{slide}{\text{Danh sách kiểu 1}{List, type=1}}
\text{Danh sách này sử dụng các item mặc định}{Default items}\pause
  \begin{enumerate}[type=1]
    \item Here
      \item we
        \item demonstrate
          \item the enumerate environment
  \end{enumerate}
\end{slide}

\begin{slide}{\text{Danh sách kiểu 0}{List, type=0}}
\text{Danh sách này sử dụng các item mặc định}{Default items}\pause
  \begin{enumerate}[type=0]
    \item Here
      \item we
        \item demonstrate
          \item the enumerate environment
  \end{enumerate}
\end{slide}

\begin{slide}{\text{Danh sách kiểu 1}{List, type=0}}
\text{Item xuất hiện ở mọi overlay}{Items xuat hien o moi overlay}\pause
  \begin{enumerate}[type=1]
    \item<1-> Here
      \item<2-> we
        \item<3-> demonstrate
          \item<4-> the enumerate environment
  \end{enumerate}
\end{slide}

\begin{slide}{\text{Danh sách kiểu 0}{List, type=0}}
\text{Item xuất hiện ở mọi overlay}{Items xuat hien o moi overlay}\pause
  \begin{enumerate}[type=0]
    \item<1-> Here
      \item<2-> we
        \item<3-> demonstrate
          \item<4-> the enumerate environment
  \end{enumerate}
\end{slide}

\begin{slide}{\text{Danh sách kiểu 1}{List, type=0}}
\text{Chỉ xuất hiện dúng một phần tử ở mỗi overlay}{Xuat hien dung mot phan tu o moi overlay}
  \begin{enumerate}[type=1]
    \item<1> Here
      \item<2> we
        \item<3> demonstrate
          \item<4> the enumerate environment
  \end{enumerate}
\end{slide}

\begin{slide}{\text{Danh sách kiểu 0}{List, type=0}}
\text{Chỉ xuất hiện dúng một phần tử ở mỗi overlay}{Xuat hien dung mot phan tu o moi overlay}
  \begin{enumerate}[type=0]
    \item<1> Here
      \item<2> we
        \item<3> demonstrate
          \item<4> the enumerate environment
  \end{enumerate}
\end{slide}

\section{\text{Mục C}{Section C}}

\text{Mục này nói về Overlay tương đối}{}

\begin{slide}{\text{Overlay tương đối}{Relative overlays}}
   \begin{itemize}
     \item A \pause
     \item B \onslide{+1}{(visible 1 overlay after B)}\pause
     \item C \onslide{+2-}{(appears 2 overlays after C, visible until the end)}
     \pause
     \item D \onslide{+1-6}{(appears 1 overlay after D, visible until overlay 6)}
     \pause
     \item E \pause
     \item F \pause
     \item G \onslide{+1-+3}{(appears 1 overlay after G for 3 overlays)}\pause
     \item H \pause
     \item I \pause
     \item J \pause
     \item K
   \end{itemize}
 \end{slide}
\end{document}
