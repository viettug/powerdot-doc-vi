%\section{Using \LyX\ for presentations}
\section{\texorpdfstring{Tạo trình diễn với \LyX}{Tao trinh dien voi LyX}}%
%% \protect\footnote{Mục này được dịch chay ;) kyanh chưa có thời gian và điều kiện kiểm tra các hướng dẫn ở mục này.
%% Vui lòng feedback. Cám ơn!}}
\label{sec:lyx}

%% \LyX\ \cite{LyXWeb} is a WYSIWYM (What You See Is What You Mean)
%% document processor based on \LaTeX. It supports standard \LaTeX\
%% classes but needs special files, called layout files, in order to
%% support non-standard classes such as \pf{powerdot}.
\LyX\ là chương trình soạn thảo \LaTeX{} với nguyên lý WYSIWYM
(What You See Is What You Mean), chạy trên môi trường |*nix|.
\LyX\ hỗ trợ các lớp \LaTeX{} chuẩn; với các lớp mới, ví dụ \pf{powerdot},
\LyX\ cần thêm các tập tin đặc biệt, gọi là tập tin |layout|.

%% To start using \LyX\ for \pf{powerdot} presentations, copy the
%% layout file |powerdot.layout| to the \LyX\ layout directory. You can
%% find this file in the doc tree of your \LaTeX\ installation:
%% \url{texmf/doc/latex/powerdot}. If you can't find it there, download
%% it from \url{CTAN:/macros/latex/contrib/powerdot/doc}. Once that is
%% done, reconfigure \LyX\ (\texttt{Edit\LyXarrow Reconfigure} and
%% restart \LyX\ afterwards). Now you can use the \pf{powerdot}
%% document class as any other supported class. Go to
%% \texttt{Layout\LyXarrow Document} and select \texttt{powerdot
%% presentation} as document class. For more information, see the \LyX\
%% documentation, which is accessible from the |Help| menu.
Để có thể dùng \pf{powerdot} trong \LyX, trước hết, hãy chép tập tin |powerdot.layout|
vào thư mục của \LyX. Tập tin |layout| có thể tìm thấy trong thư mục
\url{texmf/doc/latex/powerdot} (khi vừa xả nén tập tin |powerdot.zip|).
Nếu bạn không tìm thấy ở đó, hãy tải về từ \url{CTAN:/macros/latex/contrib/powerdot/doc}.
Sau đó, hãy cấu hình lại cho \LyX{} nhờ sử dụng \texttt{Edit\LyXarrow Reconfigure}
rồi khởi động lại chương trình \LyX. Bây giờ bạn có thể thưởng thức
\pf{powerdot} trong \LyX ;) Để biết thêm chi tiết, tham khảo tài liệu
của \LyX{} (xem từ Menu |Help| của \LyX).

% =====================================================================

\subsection{\texorpdfstring{Cách dùng layout}{Su dung layout the nao}}% {How to use the layout}

%% \pf{powerdot} \LyX\ layout provides some environments\footnote{Don't
%% confuse these with \LaTeX\ environments.} which can be used in \LyX.
%% Some of these environments (for instance |Title| or |Itemize|) are
%% natural to use since they exist also in the standard document
%% classes such as \pf{article}. For more information on these standard
%% environments, see the \LyX\ documentation.
|layout| dành cho \LyX{} của \pf{powerdot} hỗ trợ vài môi trường (không phải
là môi trường trong \LaTeX{}). Một vài môi trường như |Title| và |Itemize|
là chuẩn, đã có trong lớp \pf{article}. Xem thêm tài liệu của \LyX{}
về các môi trường chuẩn này.

%% This section will explain how to use the \pf{powerdot} specific
%% environments |Slide|, |WideSlide|, |EmptySlide| and |Note|. These
%% environments correspond to the \pf{powerdot} environments |slide|,
%% |emptyslide|, |wideslide| and |note|.
Mục này giải thích cách dùng các môi trường riêng của \pf{powerdot}
trong \LyX, đó là |Slide|, |WideSlide|, |EmptySlide| và |Note|.
Các môi trường này lần lượt tương ứng với |slide|, |wideslide|,
|emptyslide| và |note| (môi trường \LaTeX{} của \pf{powerdot}).

%We start with a simple example. The following \LaTeX\ code
Hãy bắt đầu bằng ví dụ đơn giản sau, tương ứng với đoạn mã \LaTeX{}
\begin{example}
 \begin{slide}{Slide title}
   Slide content.
 \end{slide}
\end{example}
%% can be obtained using the following \LyX\ environments. The right
%% column represents the text typed into the \LyX\ window and the left
%% column represents the environment applied to this text).
trong \LyX{} ta dùng các đoạn mã như sau (cột phải chỉ ra nội dung của |slide|,
cột bên trái là môi trường \LyX):
\begin{example}
  Slide         Slide title
  Standard      Slide content.
  EndSlide
  ...
\end{example}
%Some remarks concerning this example.
Có vài lưu ý liên quan đến ví dụ này
\begin{itemize}[leftmargin=0pt,itemsep=0pt,parsep=0pt]
\item
%% You can use the environment menu (under the menu bar, top-left
%% corner) to change the environment applied to text.
Bạn có thể dùng môi trường |menu| (dưới thanh công cụ |menu| nằm ở
góc trên bên trái của \LyX) để thay đổi môi trường cho phần nội dung được chọn.
\item
%%  The slide title should be typed on the line of the |Slide|
%% environment.
Tựa của |slide| (ở trên là |Slide title|) nên đặt trên cùng một dòng với
từ khóa |Slide| bắt đầu môi trường \LyX.
\item% |EndSlide| finishes the slide and its line is left blank.
|EndSlide| kết thúc |slide|, theo sau là khoảng trắng.
\end{itemize}

%% In the \LyX\ window, the |Slide| environment (that is, the slide
%% title) is displayed in magenta, the |WideSlide| style in green, the
%% |EmptySlide| style in cyan and the |Note| style in red and hence
%% these are easily identifiable.
Ở của sổ của \LyX, môi trường |Slide| (cùng với tựa đề của |slide|)
được bắt đầu với màu |magenta|, trong khi màu cho |WideSlide| là |green|,
cho |EmptySlide| là |cyan| và cho |Note| là đỏ. Rất dễ phân biệt, nhỉ ;)

%Here is another example.
Dưới đây là ví dụ khác (trước hết là mã \LaTeX{})
\begin{example}
 \begin{slide}{First slide title}
   The first slide.
 \end{slide}
 \begin{note}{First note title}
   The first note, concerning slide 1.
 \end{note}
 \begin{slide}{Second slide title}
   The second slide.
 \end{slide}
\end{example}
%This can be done in \LyX\ in the following way.
Tương ứng, ta có mã \LyX{} như sau đây:
\begin{example}
 Slide         First slide title
 Standard      The first slide.
 Note          First note title
 Standard      The first note, concerning slide 1.
 Slide         Second slide title
 Standard      The second slide.
 EndSlide
\end{example}
%% This example demonstrates that it is often sufficient to insert the
%% |EndSlide| style after the last slide or note only. Only when you
%% want certain material not to be part of a slide, you need to finish
%% the preceding slide manually using the |EndSlide| style. Example:
Ví dụ này cho thấy rằng chỉ cần dùng |EndSlide| một lần cho nhiều |slide|
kế tiếp nhau. Chỉ khi nào cần thêm nội dung không thuộc vào |slide| nào cả,
bạn mới phải dùng thêm |EndSlide| như ví dụ sau đây:
\begin{example}
 Slide         First slide title
 Standard      The first slide.
 EndSlide
 [ERT box with some material]
 Slide         Second slide title
 ...
\end{example}
%% You could use this, for instance, to have verbatim material on
%% slides (see also section~\ref{sec:FAQ}).
Trường hơp này xảy ra, ví dụ khi bạn cần sử dụng môi trường |verbatim|
với |slide|. Xem thêm ở Mục~\vref{sec:FAQ}.

\subsection{\texorpdfstring{Cú pháp được hỗ trợ}{Cu phap duoc ho tro}} % {Support of syntax}

%% This section lists options, commands and environments that are
%% supported through the \LyX\ interface directly, without using an ERT
%% box (\TeX-mode).
Mục này liệt kê các tùy chọn, lệnh và môi trường mà \pf{powerdot}
hỗ trợ cho \LyX{} một cách trực tiếp, không phải thông qua việc sử dụng
|ERT| |box| (hay chế độ \TeX).

%% All class options (see section~\ref{sec:classopts}) are supported
%% via the \texttt{Layout\LyXarrow Document} dialog (|Layout| pane).
%% Options for the |\pdsetup| command (see section~\ref{sec:setup})
%% should be specified in the |Preamble| pane of the
%% \texttt{Layout\LyXarrow Document} dialog.
Mọi tùy chọn cho lớp (Mục~\vref{sec:classopts}) được hỗ trợ thông
qua hộp thoại \texttt{Layout\LyXarrow Document}. Các tùy chọn của lệnh |\pdsetup|
đã nói ở Mục~\vref{sec:setup} được thay đổi thông qua phần |Preamble|
tìm thấy ở hộp thoại \texttt{Layout\LyXarrow Document} của \LyX.

%% Table \vref{tab:lyxcommands} lists the \pf{powerdot} commands that
%% are supported in \LyX.
Bảng~\vref{tab:lyxcommands} liệt kê các lệnh \pf{powerdot} hỗ trợ cho \LyX.
\begin{table}[htb]
\centering
\begin{tabular}{r|l}
Lệnh & Tương ứng trong \LyX\\\hline
\cs{title} & môi trường \texttt{Title}\\
\cs{author} & môi trường \texttt{Author}\\
\cs{date} & môi trường \texttt{Date} \\
\cs{maketitle} & điều khiển bởi \LyX.\\
\cs{section} & môi trường \texttt{Section}.
%% Options to this command (see section~\ref{sec:section}) can be specified using
%% \texttt{Insert\LyXarrow Short title} in front of the section title.\\
Tuỳ chọn cho lệnh này (xem Mục~\ref{sec:section}) được thay đổi\\
& thông qua \texttt{Insert\LyXarrow Short title} ở trước tựa của |slide|.\\
\cs{tableofcontents} & dùng \texttt{Insert\LyXarrow Lists \& TOC\LyXarrow Table of contents}.
\\&Cần dùng |ERT box| nếu lệnh này được dùng với tùy chọn. Xem bên dưới.
\end{tabular}
%\caption{Supported \pf{powerdot} commands in \LyX}
\caption{Lệnh \pf{powerdot} được hỗ trợ trong \LyX}
\label{tab:lyxcommands}
\end{table}
%% Table \vref{tab:lyxenvs} lists the \pf{powerdot} environments that,
%% besides the earlier discussed |slide|, |wideslide|, |note| and
%% |emptyslide| environments, are supported in \LyX.
Bảng~\vref{tab:lyxenvs} liệt kê các môi truờng \pf{powerdot}
được hỗ trợ trong \LyX, bên cạnh các môi trường |slide|, |wideslide|,
|emptyslide|, |note| đã nói.
\begin{table}[htb]
\centering
\begin{tabular}{r|l}
môi trường & tương ứng trong \LyX\\\hline
\texttt{itemize} & môi trường \texttt{Itemize} hoặc \texttt{ItemizeType1}.
\\& |Type1| tương ứng với |type=1| nói ở Mụ~\vref{sec:lists}).\\
\texttt{enumerate} & môi trường \texttt{Enumerate} hoặc \texttt{EnumerateType1}.\\
\texttt{thebibliography} & môi trường \texttt{Bibliography}.
\end{tabular}
%\caption{Supported \pf{powerdot} environments in \LyX}
\caption{Môi trường \pf{powerdot} được hỗ trợ trong \LyX}
\label{tab:lyxenvs}
\end{table}
%% Table \vref{tab:lyxERT} lists commands that can only be done by using
%% an ERT box (via \texttt{Insert\LyXarrow TeX}).
Bảng~\vref{tab:lyxERT} liệt kê các lệnh chỉ dùng được bằng cách sử dụng |ERT| |box|
(thông qua \texttt{Insert\LyXarrow TeX}).
\begin{table}[htb]
\centering
\begin{tabular}{r|l}
lệnh & tương ứng trong \LyX\\\hline
\cs{and} & bên trong môi trường \texttt{Author}.\\
\cs{pause} & \\
\cs{item} & %An ERT box is only required for the optional argument,
%not mandatory for overlays specifications.\\
ERT box chỉ cần thiết khi dùng tham số tùy chọn,\\
& tham số bắt buộc thì không cần ERT box\\
\cs{onslide} & cùng với \cs{onslide+} và \cs{onslide*}\\
\cs{twocolumn} & \\
\cs{tableofcontents} & chỉ dùng khi có tham số tuỳ chọn.
\end{tabular}
%\caption{\pf{powerdot} commands needing an ERT box in \LyX}
\caption{Lệnh \pf{powerdot} cần ERT box của \LyX}
\label{tab:lyxERT}
\end{table}
%% Note that you may use the clipboard in order to repeat often used
%% commands like |\pause|. Finally, table \vref{tab:lyxadd} lists
%% additional commands and environments that are supported by the layout.
Bạn có thể dùng bộ nhớ đệm |clipboard| để giữ các lệnh hay dùng như |\pause|.
Cuối cùng, Bảng~\vref{tab:lyxadd} liệt kê các lệnh và môi trường khác
được hỗ trợ trong \LyX.
\begin{table}[htb]
\centering
\begin{tabular}{r|l}
lệnh/môi trường & tương tứng trong \LyX\\\hline
\texttt{quote} & môi trường \texttt{Quote}.\\
\texttt{quotation} & môi trường \texttt{Quotation}.\\
\texttt{verse} & môi trường \texttt{Verse}.\\
\cs{caption} & môi trường \texttt{Caption} với float chuẩn
\end{tabular}
%\caption{Additional environments for \LyX}
\caption{Lệnh,môi trường bổ sung cho \LyX}
\label{tab:lyxadd}
\end{table}

\subsection{\texorpdfstring{Biên dịch với LyX}{Bien dich voi LyX}} %{Compiling with \LyX}

%% First of all, make sure that you have also read
%% section~\ref{sec:compiling}. Then, in order to get a proper
%% PostScript or PDF file, you have to set your \LyX\ document
%% properties depending on which paper and orientation you want. When
%% your \LyX\ document is open, go to the \texttt{Layout\LyXarrow
%% Document} dialog. In the \texttt{Layout} pane, put the |nopsheader|,
%% |orient| and |paper| keys as class options (see
%% section~\ref{sec:classopts} for a description). Then, go to the
%% |Paper| pane and select corresponding paper size and orientation
%% (you may choose |letter| paper in the case you set |paper=screen| in
%% the class options). Finally, go to the |View| (or
%% \texttt{File\LyXarrow Export}) menu and select your output
%% (PostScript or PDF).
Trước hết, cần nắm được các ý ở Mục~\vref{sec:compiling}.
Để có kết quả |PostScript| hoặc |PDF| như ý với \LyX, cần phải
đặt cài đặt hợp lý cho tài liệu \LyX{}: kiểu giấy, hướng giấy.
Khi tài liệu \LyX{} được mở, hãy mở hộp thoại \texttt{Layout\LyXarrow Document}.
Ở phần \texttt{Layout}, thiết lập cho các tuỳ chọn |nopsheader|, |orient| và |paper|
(theo các đã trình bày ở Mục~\ref{sec:classopts}). Sau đó, ở phần |Paper|,
chọn khổ giấy và hướng giấy tương ứng với thiết lập ấy. Có thể chọn
|letter| nếu dùng tuỳ chọn |paper=screen|. Cuối cùng, trong menu |View|
hoặc (\texttt{File\LyXarrow Export}), chọn kết quả |PostScript| hoặc |PDF|.

%% \subsection{Extending the layout}
%% If you have created a custom style (see section~\ref{sec:writestyle})
%% which defines custom templates, you may want to extend the layout
%% file\footnote{The LPPL dictates to rename a file if you modify it as
%% to avoid confusion.} so that these templates are also supported in
%% \LyX. The explanation below assumes that you have defined a template
%% called |sunnyslide|.
%% 
%% To support this new template in \LyX, you have to use the following
%% command.
%% \begin{command}
%%  `\cs{pddefinelyxtemplate}\meta{cs}\marg{template}'
%% \end{command}
%% \DescribeMacro{\pddefinelyxtemplate}
%% This will define the control sequence \meta{cs} such that it will
%% create a slide with template \meta{template} (which has been defined
%% using |\pddefinetemplate|. This new control sequence can be used in
%% the layout file as follows.
%% \begin{example}
%%  # SunnySlide environment definition
%%  Style SunnySlide
%%    CopyStyle     Slide
%%    LatexName     lyxend\lyxsunnyslide
%%    Font
%%      Color       Yellow
%%    EndFont
%%    Preamble
%%      \pddefinelyxtemplate\lyxsunnyslide{sunnyslide}
%%    EndPreamble
%%  End
%% \end{example}
%% Note that you must begin the |LatexName| field with |lyxend|. The
%% definition of the \LyX\ template has been inserted in between
%% |Preamble| and |EndPreamble| which assures that the new \LyX\
%% environment will work. After modifying the layout file, don't forget
%% to restart \LyX. See for more information about creating \LyX\
%% environments, the documentation of \LyX\ in the |Help| menu.
\endinput
