\section{Overlay}
\label{sec:overlays}

%% It is often the case that you don't want all the information on the
%% slide to appear at once. Rather, the information should appear one
%% item at a time. In \pf{powerdot}, this is achieved with overlays.
%% Each slide can be comprised of many overlays, and the overlays are
%% displayed one at a time.
Với trình diễn, nhiều khi bạn không muốn mọi thông tin trên |slide|
xuất hiện cùng một lúc, mà tuần tự từng ý từng ý một. Với \pf{powerdot},
bạn có thể đạt kết quả này nhờ |overlay|. Mỗi |slide| hiển nhiên
có thể gồm nhiều |overlay|; tại mỗi thời điểm, chỉ có một |overlay|
được thể hiện.

% =====================================================================

%\subsection{The \cs{pause} command}
\subsection{\texorpdfstring{Lệnh}{Lenh} \cs{pause}}

\label{sec:pause}

\DescribeMacro{\pause}
%% The easiest way to display information
%% sequentially is to use the |\pause| command.
Cách đơn giản nhất để tuần tự hóa thông tin trên |slide|
là dùng lệnh |\pause|.
\begin{command}
 `\cs{pause}\oarg{number}'
\end{command}
Dưới đây là ví dụ đơn giản:
\begin{example}
 \begin{slide}{Simple overlay}
   power\pause dot
 \end{slide}
\end{example}
%% The slide's information is displayed and continues until the
%% |\pause| command is encountered. No further output within the same
%% slide is displayed until the click of the mouse or the touch of the
%% keyboard. Then, the content will continue to display until all the
%% information is displayed or until another |\pause| command is
%% encountered. In this example, |power| is displayed on the first
%% overlay, and |powerdot| is the displayed on the second overlay.  The
%% |\pause| command is often used within the |itemize| and |enumerate|
%% environments. For example,
Thông tin của |slide| được thể hiện và tiếp tục cho đến khi gặp lệnh |\pause|
đầu tiên. Không có thêm kết quả nào xuất hiện cho đến khi bạn |click|
chuột hoặc gõ phím bất kỳ. Sau đó, nội dung |slide| sẽ được thể hiện
cho đến khi mọi thông tin đều xuất hiện hoặc cho tới khi gặp lệnh |\pause|
kế tiếp. Trong ví dụ trên, |power| sẽ xuất hiện trong |overlay| đầu tiên,
và |powerdot| sẽ xuất hiện ở |overlay| tiếp theo. Lệnh |\pause| thường được
dùng cùng với các môi trường |itemize| hoặc |enumerate|. Ví dụ
\begin{example}
 \begin{slide}{Multiple pauses}
   power\pause dot \pause
   \begin{itemize}
     \item Let me pause\ldots \pause
     \item \ldots while I talk \pause and chew bubble gum. \pause
     \item Perhaps you'll be persuaded.
     \item Perhaps not.
   \end{itemize}
 \end{slide}
\end{example}
%% Since |\pause| was used before the |itemize| environment, no item
%% will appear until the third overlay. Then, each item will be
%% displayed one at a time, each on their own overlay. More information
%% on using lists will follow in the next section.
Bởi vì |\pause| được dùng trước môi trường |itemize|, không có
phần tử nào của danh sách xuất hiện cho tới |overlay| thứ ba.
Sau đó, các phần tử sẽ lần lượt xuất hiện trong các |overlay|
kế tiếp. Việc điều khiển danh sách sẽ được bàn kỹ hơn ở Mục kế tiếp.

%% The optional argument of the |\pause| command specifies the number
%% of overlays to pause. An example usage is:
Lệnh |\pause| chấp nhận tham số bổ sung cho biết phải dừng bao nhiêu |overlay|.
\begin{example}
 \begin{slide}{Pause longer}
   \begin{itemize}
     \item A \pause
     \item B \pause[2]
     \item C
   \end{itemize}
 \end{slide}
\end{example}
%% In the example above, item |C| will appear on the fourth overlay.
%% The usefulness of this option will become more apparent in the next
%% section; so we will revisit a similar example at that time.
Trong ví dụ trên, phần tử |C| của danh sách xuất hiện ở |overlay|
thứ bốn. Tuỳ chọn có vẻ vô dụng này sẽ được đề cập lại ở Mục tiếp theo,
ở đó ta sẽ gặp lại ví dụ tương tự ví dụ trên đây.

% =====================================================================

%\subsection{List environments}
\subsection{\texorpdfstring{Tạo danh sách}{Tao danh sach}}

\label{sec:lists}

%% The list environments, |itemize| and |enumerate|, have special
%% treatments in \pf{powerdot}. They have an optional argument that
%% will be taken care off by the \pf{enumitem} package (see
%% \cite{enumitem}). \pf{powerdot} supplies an extra key for this
%% optional argument. In the examples that follow, features will be
%% described using the |itemize| environment but they also apply to the
%% |enumerate| environment.
Các môi trường tạo danh sách |itemize| và |enumerate| được xử lý
đặc biệt trong \pf{powerdot}. Chúng đều có tham số tuỳ chọn điều khiển
bởi gói \pf{enumitem} (xem \cite{enumitem}). Lớp \pf{powerdot} cung
cấp nhiều khóa cho tuỳ chọn này. Trong các ví dụ sau đây, các tính năng
được minh họa với môi trường |itemize|, nhưng bạn cũng có thể áp dụng
cách làm tương tự cho môi trường |enumerate|.

%Here is the typical usage of the |itemize| environment:
Dưới đây là ví dụ đơn giản nhất về cách dùng môi trường |itemize|:
\begin{example}
 \begin{slide}{Basic itemize}
   \begin{itemize}
     \item A \pause
     \item B \pause
     \item C
   \end{itemize}
 \end{slide}
\end{example}
%% The display is simple, each item appears one at a time with each
%% overlay.
Kết quả của ví dụ là danh sách đơn giản, mỗi phần tử của nó
được thể hiện ở một |overlay|.

\DescribeOption{type}
%% Suppose we wanted every item to show, but we only wanted one item to
%% appear `active' at once. This can be accomplished via the |type|
%% option for the |itemize| environment. The default value is |0|.
Giả định rằng ta muốn mọi phần tử của danh sách sẽ được thể hiện,
nhưng tại mỗi thời điểm chỉ một và một phần tử của danh sách xuất hiện
ở |slide|. Điều này có thể đạt được bằng cách chỉ định kiểu với |type|
trong phần tham số bổ sung của môi trường (giá trị mặc định là |0|).
\begin{example}
 \begin{slide}{Type 1 itemize}
   \begin{itemize}[type=1]
     \item A \pause
     \item B \pause
     \item C
   \end{itemize}
 \end{slide}
\end{example}
%% Now, every item will be displayed in the \emph{inactive
%% color}\index{inactive color} (which is defined by the style that you
%% use), and the item's font color will become the active one on the
%% overlay that it would normally appear on. The default behavior is
%% given by |type=0|.
Bây giờ, mọi phần tử đều được thể hiện với \emph{màu nhạt}\index{inactive color}
(màu được định nghĩa bởi kiểu trình diễn); khi |overlay| của một phần
tử được thể hiện, phần tử đó sẽ được tô đậm để làm nổi bật nó so
với các phần tử khác. Sự phân biệt của hai |type| có thể minh họa như
sau (dấu |*| chỉ màu nhạt):
\begin{example}
     type=0        type=1
  1: x------       x------
  2: xx-----       *x----- 
  3: xxx----       **x---- 
  4: xxxx---       ***x--- 
  5: xxxxx--       ****x--
  6: xxxxxx-       *****x-
\end{example}

%% Lists can also be nested to create complicated structures. When a
%% list is nested, it inherits the setting of the |type| option from
%% the `parent' list, but that can be overruled by specifying the
%% |type| option in the optional argument of the nested list. We
%% present here one example, but many more can be created by nesting
%% lists of different types in different ways.
Các danh sách có thể lồng nhau. Khi xảy ra trường hợp này, các danh sách
thứ cấp sẽ thừa hưởng giá trị của |type| ở môi trường mẹ.
Tất nhiên, với bất kỳ danh sách nào, ta cũng có thể đặt lại giá trị cho |type|.
Dưới đây là ví dụ về hai danh sách lồng nhau:
\begin{example}
 \begin{slide}{Nested lists}
   \begin{itemize}
     \item A\pause
     \begin{itemize}[type=1]
       \item B\pause
     \end{itemize}
     \item C
   \end{itemize}
 \end{slide}
\end{example}
%% This displays |A| and |B| on the first overlay, but |B| is inactive.
%% On overlay 2, |B| will become active and on overlay 3, |C| will
%% become visible.
Kết quả là phần tử |A| và |B| được thể hiện ở |overlay| đầu tiên,
nhưng phần tử |B| được tô màu nhạt. Ở |overlay| thứ hai, phần tử |B|
sẽ được tô màu đậm, và ở |overlay| thứ ba, |C| sẽ được tô đậm.

% =====================================================================

%\subsection{The \cs{item} command}
\subsection{\texorpdfstring{Lệnh}{Lenh} \cs{item}}

\DescribeMacro{\item}
%% The |\item| command has an extra \emph{optional} argument in
%% \pf{powerdot} which allows for creating overlays in a more flexible
%% way then |\pause| provides.
Lệnh |\item| để tạo phần tử cho danh sách có thể nhận tham số bổ sung (\emph{optional})
cho phép tạo các |overlay| linh hoạt hơn lệnh |\pause|.
\begin{command}
 `\cs{item}\oarg{label}\larg{overlays}'
\end{command}
%% This optional argument should contain an overlay specification
%% stating on which overlays you want the item to appear. This
%% specification is a comma separated list where each item can used the
%% notation as in table~\ref{tab:item}.
Tham số bổ sung \meta{overlays} này là danh sách các |overlay|
mà phần tử sẽ xuất hiện (có nghĩa, một phần tử
có thể xuất hiện ở một hay nhiều |overlay| được chỉ ra). Trong
danh sách này, các |overlay| được cho bởi số, cách nhau bởi dấu phảy.
\begin{table}[htb]\centering
\begin{tabular}{d}
Cú pháp&Ý nghĩa\\\hline
\texttt{x}&Chỉ ở |overlay| \texttt{x}\\
\texttt{-x}&Các |overlay| nhỏ hơn hoặc bằng \texttt{x}\\
\texttt{x-}&Các |overlay| lớn hơn hoặc bằng \texttt{x}\\
\texttt{x-y}&Các |overlay| từ \texttt{x} đến \texttt{y},
bao gồm cả \texttt{x} và \texttt{y}\\
\end{tabular}
%\caption{\cs{item} and \cs{onslide} notation}
\caption{Quy ước cho \cs{item} và \cs{onslide}}
\label{tab:item}
\end{table}
%% The \meta{label} argument is the standard optional argument for
%% |\item| in \LaTeX. A \LaTeX\ manual \cite{companion} can tell you
%% more about this argument.
Tham số \meta{label} được hiểu tương tự như trong \LaTeX{} chuẩn,
xem chi tiết ở chẳng hạn \cite{companion}.

%Here is an example.
Dưới đây là ví dụ
\begin{example}
 \begin{slide}{Active itemize}
   \begin{itemize}[type=1]
    \item<1> A
    \item<2> B
    \item<3> C
   \end{itemize}
 \end{slide}
\end{example}
%% Here we have said that |A| should only be active on overlay 1, |B|
%% should only be active on overlay 2, and |C| should only be active on
%% overlay 3. Again, when the item is not active, it appears in the
%% inactive color because of |type=1|.
Với ví dụ này, phần tử |A| chỉ xuất hiện với màu tô đậm ở |overlay 1|,
phần tử |B| chỉ xuất hiện và được tô đậm ở |overlay| thứ hai,... Để ý rằng,
do chỉ định |type=1|, nếu phần tử không phải là phần tử hiện tại,
màu của nó được tô nhạt. Ta có thể minh họa như sau (chữ THƯỜNG chỉ màu nhạt):
\begin{example}
  1: --A---- 
  2: --aB--- 
  3: --abC--
\end{example}
%% If |type=0| is specified and if each item is given an overlay
%% option, then each item will appear only when it is active. When the
%% item is not active, then it will not show on the slide at all. More
%% examples demonstrating the syntax for \meta{overlays} will be
%% discussed in the next section.
Nếu chỉ định |type=0| và nếu mỗi phần tử của danh sách đều có thêm
tham số bổ sung chỉ định |overlay|, thì một phần tử sẽ bị giấu
hoàn toàn chứ không phải được tô màu nhạt. Ví dụ:
\begin{example}
  1: --A---- 
  2: ---B--- 
  3: ----C--
\end{example}

Ví dụ nhiều hơn về cú pháp của \meta{overlays} được cho ở Mục tiếp theo.

% =====================================================================

%\subsection{The \cs{onslide} command}
\subsection{\texorpdfstring{Lệnh}{Lenh} \cs{onslide}}
\label{sec:onslide}

\DescribeMacro{\onslide}
%Overlays can also be achieved using the |\onslide| command.
Các |overlay| có thể thu được nhờ lệnh |\onslide|.
\begin{command}
 `\cs{onslide}\marg{overlays}\marg{text}'
\end{command}
%% This command takes an \meta{overlays} specification as first
%% argument and the \meta{text} to apply it to as second argument. The
%% \meta{overlays} on which the text will appear are specified as a
%% comma separated list with syntax as in table~\ref{tab:item}. We
%% start off with a simple example.
Ở lệnh này, tham số \meta{overlays} ở vị trí thứ nhất;
tham số \meta{text} ở vị trí thứ hai chỉ nội dung cần biểu diễn.
Tham số \meta{overlays} là danh sách các |overlay|,
với cú pháp được nêu ở Bảng~\vref{tab:item}. Xét ví dụ sau đây.
\begin{example}
 \begin{slide}{Simple onslide}
   \onslide{1,2}{power}\onslide{2}{dot}
 \end{slide}
\end{example}
%% We have instructed |power| to appear on overlays one and two, and
%% |dot| to appear only on overlay two. As you might guess, this
%% example has the same output as our first |\pause| example. Yet, it
%% is clearly the case that our syntax is more complicated. However,
%% this slight ``complication'' also allows for much more flexibility.
Ở đây, chúng ta muốn |power| xuất hiện ở |overlay| thứ nhất và hai,
và |dot| chỉ xuất hiện ở |overlay| thứ hai. Bạn có thể đoán được rằng,
kết quả của ví dụ trên giống như khi dùng lệnh |\pause| ở ví dụ đầu tiên.
Bạn có thể ngạc nhiên là, nếu vậy thì việc đưa ra lệnh |\oneslide|
có vẻ rắc rối. Thực tế thì điều này có dụng ý, giúp cho bạn linh hoạt
hơn trong quá trình tạo |overlay|.

\DescribeMacro{\onslide+}
%% Consider the above example with the following modifications:
Ta hãy xét ví dụ trên, nhưng với một chút thay đổi như sau:
\begin{example}
 \begin{slide}{Simple onslide+}
   \onslide+{1}{power}\onslide+{2}{dot}
 \end{slide}
\end{example}
%% The |\onslide+| command displays its content in a different manner
%% altogether. Now, |dot| appears on every overlay, but it is in
%% inactive color\index{inactive color} and matches the normal font
%% color \textit{only} on overlay two. This is comparable to the
%% |type=1| behavior for lists (see section~\ref{sec:lists}).
%Lệnh |\onslide+| cho phép biểu diễn nội dung của nó nhiều cách khác nhau.
Bây giờ, cụm từ |dot| xuất hiện ở mọi |overlay|, nhưng nó
có màu tô nhạt,\index{inactive color} và được tô màu bình thường
chỉ khi nó xuất hiện ở |overlay| thứ hai. Điều này có nét tương
tự như khi dùng |type=1| đối với danh sách (xem Mục~\vref{sec:lists}).
Để ý rằng, ở đây ta nói đến các |overlay| tổng quát, không chỉ đơn
giản là danh sách. Ta minh họa kết quả trên như sau đây, chữ HOA
chỉ rằng chữ được tô màu bình thường, còn chữ THƯỜNG chỉ màu nhạt
\begin{example}
  1: POWERdot
  2: powerDOT
\end{example}
%% When executing this example, we will also notice that the |\onslide|
%% command does hide material, but still reserves the right amount of
%% space for it: on overlay 2, the |dot|s appear right above each
%% other. The next command does not reserve space.
Ta để ý rằng, mặc dù thực hiện 'giấu' (|\oneslide|)
hoặc tô màu nhạt (|\oneslide+|) cho nội dung \meta{text}, nó vẫn giữ một khoảng
trắng cho phần nội dung đó -- hệt như khi nội dung đó xuất hiện.
Lệnh |\onslide*| trong được mô tả sau đây sẽ không làm như vậy.

\DescribeMacro{\onslide*}
%% Instead of hiding and reserving space
%% (|\onslide|) or putting \meta{text} in the inactive color
%% (|\onslide+|) when the overlay doesn't match \meta{overlays}, this
%% command just eats the material altogether. To understand the
%% differences, consider the following example:
Thay vì 'giấu' (hoặc tô màu nhạt) cho phần nội dung \meta{text},
nếu nội dung \meta{text} không được thiết lập để hiển thị ở \meta{overlay},
lệnh này bỏ qua hoàn toàn nội dung đó. Để rõ hơn, ta hãy xét ví dụ sau đây:
\begin{example}
 \begin{slide}{Simple onslide*}
   \onslide{1}{power}\onslide{2}{dot}\\
   \onslide+{1}{power}\onslide+{2}{dot}\\
   \onslide*{1}{power}\onslide*{2}{dot}
 \end{slide}
\end{example}
%% The output of the first two lines, we are already familiar with. The
%% third line displays |power| on overlay 1 and |dot| on overlay 2, but
%% no space for |power| is reserved on overlay 2. Hence |dot| will
%% start on the cursor position that |power| started on overlay 1 and
%% it is not aligned below the other two |dots|.
Kết quả của hai dòng đầu tiên chúng ta đã quen thuộc. Hãy xét dòng thứ ba.
Chữ |power| xuất hiện ở |overlay| thứ nhất, còn |dot| ở |overlay| thứ hai.
Tuy nhiên, không có khoảng trắng nào được dành cho |power| ở |overlay| thứ
hai. Do đó, |dot| sẽ xuất hiện ở vị trí mà |power| xuất hiện. Kết quả có
thể minh họa như sau (chữ HOA: tô màu thường, chữ THƯỜNG: tô màu nhạt;
mỗi chữ x tượng trưng cho một vị trí bỏ trống).
\begin{example}
  1: POWERxxx   2: xxxxxDOT
  1: POWERdot   2: powerDOT
  1: POWERxxx   2: DOTxxxxx
\end{example}
%% We finish with an example of the syntax that is possible with
%% |\item| and |\onslide|. Remember that these commands take a comma
%% separated list for the \meta{overlays} specification and that each
%% element can used the syntax as explained in table~\ref{tab:item}.
%% The various variations are demonstrated in the example below.
Ta kết thúc mục này với ví dụ phức tạp. Nhớ rằng những lệnh
|\item| và |\onslide| nhận tham số \meta{overlays} là danh sách
các |overlay| theo cú pháp ở Bản~\vref{tab:item}.
\begin{example}
 \begin{slide}{Lists}
   \onslide{10}{chi o overlay 10}\par
   \onslide{-5}{cac onverlay <= 5}\par
   \onslide{5-}{cac overlay >= 5}\par
   \onslide{2-5}{cac overlay 2,3,4,5}\par
   \onslide{-3,5-7,9-}{overlay >= 1, tru 4 va 8}
 \end{slide}
\end{example}

% =====================================================================

%\subsection{Relative overlays}
\subsection{Overlay \texorpdfstring{tương đối}{tuong doi}}
%% Sometimes it is a pain to keep track of when an item should appear
%% or become active. You might, for example, just care that some text
%% appears on the overlay \textit{after} some other item. This
%% functionality is provided through the use of relative overlays which
%% should not be used outside list environments that use |\item|. Let's
%% consider a simple, illuminating example.
Thật phiền phải nhớ thứ tự các |overlay| để biểu diễn đúng nội dung nào
trước, nội dung nội sau. 
Trong thực tế, ta chỉ cần nhớ, chẳng hạn chữ |power| sẽ xuất hiện
trước chữ |dot| là đủ. Điều này có thể thực hiện nhờ |overlay| tương đối
và môi trường tạo danh sách. Hãy xem ví dụ đơn giản sau đây

\begin{example}
 \begin{slide}{Relative overlays}
   \begin{itemize}
     \item A \pause
     \item B \onslide{+1}{(visible 1 overlay after B)}\pause
     \item C \onslide{+2-}{(appears 2 overlays after C, visible until the end)}
     \pause
     \item D \onslide{+1-6}{(appears 1 overlay after D, visible until overlay 6)}
     \pause
     \item E \pause
     \item F \pause
     \item G \onslide{+1-+3}{(appears 1 overlay after G for 3 overlays)}\pause
     \item H \pause
     \item I \pause
     \item J \pause
     \item K
   \end{itemize}
 \end{slide}
\end{example}
%% As you can see, we still use |\onslide|. The only change is with the
%% syntax of the list of overlays. Now, we can specify a `|+|' symbol
%% in the list. In its simplest usage, |\onslide{+1}| will make text
%% display one overlay after the overlay it would \textit{normally}
%% appear on. You can still use the syntax in table~\ref{tab:item}.
%% These are demonstrated in the above example. Notice,
%% |\onslide{+1-6}| means that the text will appear one overlay after
%% the overlay it would normally appear on and that the text should
%% remain shown until overlay seven. To make text appear for a range of
%% relative overlays, see the final demonstration in the above example.
Ở đây, ta vẫn dùng |\oneslide|, nhưng với cú pháp mới của cho các |overlay|.
Đó là ta có thể sử dụng dấu `|+|' trong danh sách. Ở ví dụ đơn giản nhất,
|\onslide{+1}| sẽ thể hiện nội dung ở một |\overlay| kế tiếp. Ta vẫn có thể
dùng cú pháp đã nêu ở Bảng~\vref{tab:item}. Ví dụ, |\onslide{+1-6}{power}|
sẽ thể hiện chữ |power| ở các |overlay| kế tiếp cho đến |overlay| \underline{mang số 6}.
Ở ví dụ cuối cùng, |\onslide{+1-+3}{power}|, chữ |power| sẽ xuất hiện ở |overlay|
kế tiếp và sẽ tiếp tục xuất hiện thêm ở 3 |overlay| nữa. Ta minh họa kết quả
trên như sau đây (dấu |*| đại diện cho tham số của các lệnh |\onslide| ở trên:
\begin{example}
  1: A
  2: A_B
  3: A_B*_C
  4: A_B__C__D
  5: A_B__C*_D*_E
  6: A_B__C*_D*_E_F
  7: A_B__C*_D__E_F_G_
  8: A_B__C*_D__E_F_G*_H
  9: A_B__C*_D__E_F_G*_H_I
 10: A_B__C*_D__E_F_G*_H_I_J
 11: A_B__C*_D__E_F_G__H_I_J_K
\end{example}
\endinput
