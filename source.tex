%\section{Source code documentation}
\section{\texorpdfstring{Mã nguồn}{Ma nguon}}

% =====================================================================

\subsection{\texorpdfstring{Mã nguồn}{Ma nguon} powerdot}
\label{ssec:sande}

%% In case you want regenerate the package files from the source or
%% want to have a look at the source code description, locate
%% |powerdot.dtx|, search in the file for |\OnlyDescription| and remove
%% that and do
Mã nguồn của lớp \pf{powerdot} là tập tin |powerdot.dtx|.
Biên dịch mã nguồn này bạn thu được tài liệu đầy đủ nhất về \pf{powerdot},
kể cả mã \LaTeX{} của lớp với các chú thích kỹ thuật.

Bạn có thể biên dịch mã nguồn này theo các bước sau:
\begin{command}
  latex powerdot.dtx
  latex powerdot.dtx
  bibtex powerdot
  makeindex -s gglo.ist -o powerdot.gls powerdot.glo
  makeindex -s gind.ist -o powerdot.ind powerdot.idx
  latex powerdot.dtx
  latex powerdot.dtx
\end{command}

% =====================================================================

\subsection{\texorpdfstring{Mã nguồn}{Ma nguon} powerdot-doc-vn}

Mã nguồn tài liệu \pf{powerdot-doc-vn} có trong tập tin |powerdot-*-doc-vn-src-*.zip|.
%% Có hai tập tin |*.tex| là |powerdot-doc-vn.tex| và |powerdot-doc-vn-tcvn.tex|.
%% Bạn có thể biên dịch tập tin đầu tiên nếu gói |vntex| trên hệ thống của
%% bạn hỗ trợ |UTF-8| (
Tập tin |powerdot-doc-vn.tex| cần hỗ trợ UTF-8 của gói |vntex|.
Có thể bạn phải thay |\usepackage[utf8x]| bởi |\usepackage[utf8]|
nếu hệ thống của bạn không có gói |ucs|.
%% Tập tin còn lại dùng với hỗ trợ |TCVN| của |vntex|.

Việc biên dịch tập tin (|powerdot-doc-vn.tex|) để có bản |PDF| như sau:
\begin{command}
  latex powerdot-doc-vn
  latex powerdot-doc-vn
  latex powerdot-doc-vn
  dvips -o powerdot-doc-vn.ps powerdot-doc-vn.dvi
  ps2pdf powerdot-doc-vn.ps
\end{command}

Nếu bạn muốn có bản in (hai màu đen trắng), trước khi biên dịch như trên,
tạo tập tin |printctl.tex| như sau:
\begin{command}
  echo '\printtrue' > printctl.tex
\end{command}

Tập in |Makefile| đi kèm với mã nguồn chỉ dành cho mục đích tham khảo!
Nó có thể hoạt động không như ý, thậm chí phá hỏng hệ thống của bạn!

\endinput
