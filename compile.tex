%\section{Compiling your presentation}
\section{\texorpdfstring{Biên dịch. Cài đặt}{Bien dich. Cai dat}. Xem}
\label{sec:compiling}

% =====================================================================

%\subsection{Dependencies}
\subsection{\texorpdfstring{Gói phụ thuộc}{Goi phu thuoc}}

\label{sec:dependencies}

%% In table~\ref{tab:dependencies} is a list of packages that
%% \pf{powerdot} uses to perform specific tasks. Dependencies of
%% packages in this table are not listed. Notice further that styles
%% may have extra requirements (see section~\ref{sec:styles}). In the
%% table, `required' means that you should have a version \emph{at
%% least} as new as listed and `tested' means that \pf{powerdot} was
%% tested with this version, but that it could equally well work with
%% an older or newer version than the one listed in the table. So, when
%% trying to solve an error, first concentrate on solving version
%% issues for the `required' packages. To find out which version of a
%% package you are currently using, put |\listfiles| on the first line
%% of your document, run it with \LaTeX, open the |.log| file and read
%% the file list (see a \LaTeX\ manual for more information). If you
%% need to update a package, you can get it from CTAN \cite{CTAN}.

Ở Bảng~\vref{tab:dependencies} là danh sách các gói đòi hỏi bởi \pf{powerdot}
(gói phụ thuộc của các gói trong bảng này không được liệt kê). Chú ý rằng
mỗi kiểu có những đòi hỏi riêng (xem Mục~\vref{sec:styles}). Trong bảng dưới đây,
các gói `|bắt buộc|' cho biết bạn không thể dùng phiên bản cũ hơn của gói đó;
trong khi các gói `|đã kiểm tra|' có nghĩa \pf{powerdot} đã được kiểm tra
sự tương thích với phiên bản đã chỉ ra, nhưng có thể vẫn gặp lỗi với các
phiên bản \underline{cũ hơn} hoặc \underline{mới hơn} của gói đó.
Để biết phiên bản của các gói được dùng, bạn hãy thêm lệnh |\listfiles|
vào đầu tài liệu, biên dịch bằng \LaTeX{} rồi mở tập tin |.log| để xem
kết quả. Để có bản cập nhật của các gói trên, bạn có thể tải chúng về
từ CTAN \cite{CTAN}.% hoặc VietTUG \cite{VietTUG}.

\begin{table}[htb]
\centering
\begin{tabular}{e}
Gói/tập tin & Phiên bản & Ngày & Mức độ\\\hline
\pf{xkeyval} \cite{xkeyval} & 2.5c & 2005/07/10 & bắt buộc\\
\texttt{pstricks.sty} \cite{PSTricksWeb,PSTricks} & 0.2l & 2004/05/12 & bắt buộc\\
\pf{xcolor} \cite{xcolor} & 1.11 & 2004/05/09 & bắt buộc\\
\pf{enumitem} \cite{enumitem} & 1.0 & 2004/07/19 & bắt buộc\\\hline
\pf{article} class & 1.4f & 2004/02/16 & đã kiểm tra\\
\pf{geometry} \cite{geometry} & 3.2 & 2002/07/08 & đã kiểm tra\\
\pf{hyperref} \cite{hyperref} & 6.74m & 2003/11/30 & đã kiểm tra\\
\pf{graphicx} \cite{graphics} & 1.0f & 1999/02/16 & đã kiểm tra
%%<<1.3>>
%% \pf{float} \cite{float} & 1.3d & 2001/11/08 & đã kiểm tra
\end{tabular}
\caption{Gói phụ thuộc}\label{tab:dependencies}
\end{table}

% =====================================================================

\subsection{\texorpdfstring{Cài đặt}{Cai dat}}

Bạn tải về các gói sau đây:

\medskip
\noindent
{\small
\url{ftp://tug.ctan.org/pub/tex-archive/macros/latex/contrib/powerdot.zip}\\
\url{ftp://tug.ctan.org/pub/tex-archive/macros/latex/contrib/enumitem.zip}\\
\url{ftp://tug.ctan.org/pub/tex-archive/macros/latex/contrib/xcolor.zip}\\
\url{ftp://tug.ctan.org/pub/tex-archive/macros/latex/contrib/xkeyval.zip}\\
\url{ftp://ftp.jaist.ac.jp/pub/TeX/tex-archive/graphics/pstricks/latex/pstricks.sty}
}

\medskip
Sau khi xả nén các tập tin |.zip| (vào cùng thư mục nào đó),
bạn có được cấu trúc thư mục như sau.
\begin{example}
  enumitem
  `-- enumitem.sty
  powerdot
  |-- doc
  |-- run
  |   |-- powerdot-*.sty
  |   `-- powerdot.cls
  `-- source
      `-- powerdot.dtx
  xcolor
  |-- xcolor.ins
  xkeyval
  |-- doc
  |-- run
  |   |-- *.tex
  |   |-- *.sty
\end{example}

Bạn phải làm công việc tương đối phức tạp sau đây: biên dịch 
tập tin |xcolor.ins|. Nếu chương trình soạn thảo của bạn không
hỗ trợ  việc biên dịch tập tin này, bạn thử chạy từ dòng lệnh
(tại thư mục |xcolor|)
\begin{example}
  latex xcolor.ins
\end{example}
Việc biên dịch này đảm bảo rằng trong thư mục |xcolor| ở trên
sẽ xuất hiện thêm các tập tin |.def|, |.sty| và |.pro|.

Bây giờ hãy tìm đến thư mục |LocalTeXMF| trên hệ thống của bạn.
Đó thường là |c:\localtexmf| đối với Mik\TeX{} hoặc là |~/texmf|
đối với te\TeX. Hãy tạo các thư mục sau bên dưới thư mục |LocalTeXMF|
\begin{example}
  tex/latex/xcolor
  tex/latex/enumitem
  tex/latex/xkeyval
  tex/latex/powerdot
  tex/latex/pstricks
  dvips/xcolor
\end{example}
Hãy chép các tập tin vào thư mục tương ứng
\begin{example}
  dvips/xcolor        <= xcolor/*.pro
  tex/latex/xcolor    <= xcolor/*.def xcolor/*.sty
  tex/latex/enumitem  <= enumitem/enumitem.sty
  tex/latex/xkeyval   <= xkeyval/run/*
  tex/latex/powerdot  <= powerdot/run/*
  tex/latex/pstricks  <= pstricks.sty
\end{example}
Cuối cùng, hãy cập nhật hệ thống \TeX{} của bạn.
Với te\TeX{}, bạn thi hành lệnh sau
\begin{command}
  $ texhash ~/texmf
\end{command}
còn với Mik\TeX{}, bạn chạy chương trình |mo.exe| (từ Menu |RUN|
chẳng hạn) rồi chọn `\textbf{Refresh now}' để cập nhật hệ thống \TeX.

% =====================================================================

%\subsection{Creating and viewing output}
\subsection{\texorpdfstring{Biên dịch và xem kết quả}{Bien dich va xem ket qua}}
\label{sec:creation}

%% To compile your presentation, run it with \LaTeX. The DVI that is
%% produced this way can be viewed with MiK\TeX's DVI viewer
%% YAP.\footnote{Unless you are using \pf{pstricks-add} which distorts
%% the coordinate system in DVI.} Unfortunately, xdvi and kdvi (kile)
%% do not support all PostScript specials and hence these will display
%% the presentation incorrectly. If your DVI viewer does support
%% this, make sure that your DVI display settings match that of the
%% presentation. In case you are using the |screen| paper, you should
%% set the DVI display setting to using the letter paper format. If
%% your DVI viewer allows for custom paper formats, use 8.25 inch by 11
%% inch.
Để có được trình diễn thực sự, đầu tiên, bạn biên dịch tài liệu nguồn
bằng \LaTeX. Kết quả |DVI| cho bởi bước này có thể xem bằng trình |YAP|
của Mik\TeX.\footnote{Trừ khi bạn sử dụng gói \pf{pstricks-add} làm ảnh
hưởng đến cách tính tọa độ của YAP.} Thật không may, các chương trình |xdvi|
và |kdvi| trên hệ thống |*nix| không hỗ trợ tập tin |DVI| có nhúng mã
lệnh |PostScript|, do đó, bạn không thể thấy kết quả như ý với các trình xem này.
Nếu trình xem |DVI| hỗ trợ |PostScript|, hãy chắc rằng các cài đặt của trình xem
đó khớp với các thuộc tính tương ứng của trình diễn. Khi bạn tạo ra trình diễn
|screen|, bạn nên cài đặt để trình xem hiển thị khổ xem |letter|.
Tốt hơn nữa, nếu trình xem cho phép, bạn hãy chỉnh cỡ trang xem là 8.25 inch
và 11 inch.

%% Note that certain things that are produced with PostScript or PDF
%% techniques will not work in a DVI viewer. Examples are hiding of
%% material via postscript layers (as is done, for instance, by
%% |\pause|, see section~\ref{sec:overlays}) and hyperlinks, for
%% instance in the table of contents.
Chú ý rằng, các kết quả, hiệu ứng tạo với mã lệnh |PostScript| hoặc |PDF|
không thể thấy được với trình xem |DVI|. Ví dụ, việc che giấu các phần
tử của danh sách, việc tạm dừng |slide| bằng lệnh |\pause|, \ldots

%% If you want to produce a postscript document, run dvips over the DVI
%% \emph{without any particular command line options related to
%% orientation or paper size}. \pf{powerdot} will write information to
%% the DVI file that helps dvips and ps2pdf (ghostscript) to create a
%% proper document. If you have some reason that this does not work for
%% you and you want to specify the paper and orientation yourself, you
%% should use the |nopsheaders| option that is described in
%% section~\ref{sec:setup}. The PostScript document could, for
%% instance, be used to put multiple slides on a page using the |psnup|
%% utility.
Nếu bạn muốn có kết quả |postscript| của tài liệu, sử dụng chương trình 
|dvips| để chuyển từ tập tin |DVI| thu đưọc sang sạng |PS|. \emph{Nhớ
đừng chỉ ra cỡ trang ghi gọi lệnh này}, bởi \pf{powerdot} đã ghi sẵn các
mã lệnh |PostScript| để định cỡ trang.% Xem thêm ở Mục~\vref{sec:setup}.

Cuối cùng, từ kết quả |PS|, bạn có thể có được kết quả |PDF| bằng cách
sử dụng chương trình chuyển đổi |ps2pdf|.\footnote{Hãy xem thêm về các chương
trình ps2pdf12, ps2pdf13, ps2pdf14. Theo mặc định, ps2pdf tương đương với ps2pdf12.}
Để ý rằng, chương trình |ps2pdf| cần sử dụng các công cụ của |PostScript|.
Trên hệ thống Slackware Linux chẳng hạn, sau khi cài te\TeX{},
bạn có thể cài thêm các gói |espgs| và |gnu-gs-fonts|
để đảm bảo chương trình |ps2pdf| làm việc tốt. Với hệ thống \TeX{} trên |Windows|,
bạn phải cài thêm chương trình |ghostscript| chẳng hạn.

Từ kết quả |PS|, bạn cũng có thể dùng chương trình |psnup|
để đặt nhiều |slide| lên cùng một trang.

Trong quá trình chuyển đổi |DVI| sang sạng |PS| rồi |PDF|,
nếu bạn gặp trục trặc với cỡ trang hoặc hướng trang,
thử dùng tuỳ chọn |nopsheader| như đã nói ở Mục~\vref{sec:classopts};
khi đó, bạn nhớ chỉ ra thiết lập của mình ở dòng lệnh khi gọi các chương trình
|dvips| hoặc |ps2pdf|.

%% To create a PDF document for your presentation, run ps2pdf over the
%% PS file created with dvips. Also here, you can \emph{leave out any
%% command line arguments related to paper size or orientation}. If
%% this is problematic for you somehow, use the |nopsheaders| option as
%% before and specify the paper and orientation at each intermediate
%% step yourself.

Tóm lại, nếu bạn có tài liệu |foo.tex|, thì quá trình biên dịch
để tạo trình diễn gồm các bước sau (dấu \$ chỉ dấu nhắc ở dòng lệnh)
\begin{command}
  $ latex foo
  $ latex foo
  $ dvips -o foo.ps foo.dvi
  $ ps2pdf foo.ps
\end{command}
(phải biên dịch một, hai hoặc nhiều lần hơn bằng |latex| để các tham khảo chéo
và bảng Mục lục được chính xác.)

\endinput
