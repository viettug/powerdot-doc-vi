%% $Id$ %%

\section{\texorpdfstring{Tạo trình diễn}{Tao trinh dien}}
\label{sec:slides}

% =====================================================================

\subsection{\texorpdfstring{Trang tiêu đề}{Trang tieu de}}
\label{sec:titleslide}

\DescribeMacro{\title}
\DescribeMacro{\author}
\DescribeMacro{\and}
\DescribeMacro{\date}
\DescribeMacro{\maketitle}
%% The title slide is created by the |\maketitle| command. Its use is
%% the same as in the standard \LaTeX\ document classes. See an example
%% below.
Trang tiêu đề của trình diễn được tạo bởi |\maketitle|, tương tự như cách
quen thuộc trong \LaTeX{} chuẩn. Từ phiên bản 1.3, lệnh này còn chấp nhận
các tham số bổ sung:
\begin{command}
 `\cs{maketitle}\oarg{options}'
\end{command}
%Its use is the same as in the standard \LaTeX\ document classes. The
%optional argument \meta{options} can contain any option from
%section~\ref{sec:glopts}. Specifying such an option in the
%|\maketitle| command will only have an effect on the title slide and
%not on other slides. See an example below.
Phần tham số bổ sung \meta{options} có thể chứa bất kỳ tùy chọn nào
được nói đến trong Mục~\vref{sec:glopts}. Việc chỉ ra các bổ sung này
tất nhiên chỉ ảnh hưởng đến |slide| tiêu đề mà thôi! Ví dụ
\begin{example}
 \documentclass{powerdot}
   \title{Title}
   \author{You \and me}
   \date{September 14, 2005}
 \begin{document}
    \maketitle
    ...
 \end{document}
\end{example}
%% The |author|, |title| and |date| declarations provide the text to be
%% used when making a title page. The design of the title page is
%% specific to the style in use. Notice the use of |\and| for
%% separating multiple authors. See a \LaTeX\ manual \cite{companion}
%% for more information on commands such as |\title| and |\author|.
Các khai báo |author|, |title| và |date| cho biết lần lượt
tác giả, tiêu đề và ngày tháng cùa trình diễn. Trang tiêu đề được
trình bày theo thiết kế của kiểu trình diễn. Xem tài liệu \cite{companion}
về chi tiết của các lệnh |\title| và |\author|.

% =====================================================================

%\subsection{Other slides}
\subsection{\texorpdfstring{Tạo slide}{Tao slide}}

\label{sec:otherslides}

\DescribeEnv{slide}
%% The centerpiece of every presentation is the
%% slide. In \pf{powerdot}, the content of each slide is placed in a
%% |slide| environment.
Phần trung tâm của mọi trình diễn là |slide|.
Với \pf{powerdot}, nội dung của mỗi |slide| được đặt trong môi trường
|slide|.
\begin{command}
 `\cs{begin}\texttt{\{slide\}}\oarg{options}\marg{slide title}'
 `\meta{body}'
 `\cs{end}\texttt{\{slide\}}'
\end{command}

%% In section~\ref{sec:overlays} we'll see how to give some life to the
%% slides, but for now, let's look at a simple example.
Ở Mục~\vref{sec:overlays}, bạn sẽ biết tinh chỉnh với tuỳ chọn |overlay|.
Bây giờ, hãy xem một ví dụ đơn giản:
\begin{example}
 \begin{slide}{First slide}
   Hello World.
 \end{slide}
\end{example}
%% The slide environment has one required argument, namely the slide
%% title. When a slide is created, the slide title is used to create an
%% entry in the table of contents and in the list of bookmarks. The
%% table of contents is a listing of the slides and section titles in
%% the presentation that appears on each slide.
Môi trường |slide| có một tham số bắt buộc, là tựa của |slide|.
Khi một |slide| được tạo ra, tựa của nó sẽ được liệt kê trong
Bảng Mục lục và trong danh sách các |bookmark|. Bảng Mục lục
liệt kê tựa của tất cả các |slide| và các mục của trình diễn, và nó sẽ
xuất hiện ở mọi |slide|.

%% The table of contents is clickable (when the presentation is
%% compiled into PDF) and serves as a nice way to jump from location to
%% location within the presentation. The bookmark list is only present
%% when compilation is taken all the way to the PDF file format. It
%% also serves as a table of contents, but this list does not appear on
%% \textit{any} of the slides, but in a separate window in a PDF
%% viewer. In the example above, the entries in both table contents and
%% the list of bookmarks would be titled |First slide|.
Ở Bảng Mục lục, có các liên kết (khi biên dịch qua dạng |PDF|) nhờ đó
bạn có thể dễ dàng tìm kiếm các |slide| trong trình diễn của mình.
Trong khi đó, danh sách |bookmark| cũng tương tự như Bảng Mục lục, nhưng
nó không xuất hiện bất kỳ |slide| nào, mà ở một cửa sổ đặc biệt của trình
xem |PDF|. Trong ví dụ trên, tựa |First slide| sẽ xuất hiện trong cả
Bảng Mục lục và danh sách |bookmark|. Danh sách |bookmark| chỉ có khi
biên dịch kết quả thành dạng |PDF|.

%% The \meta{options} for the |slide| environment allow the user to
%% specify alternative titles for the table of contents and bookmark
%% entries. There is also a |trans| option that works only for the
%% current slide.
%% Phần tham số bổ sung (hay tuỳ chọn) (\meta{options})
%% của |slide| cho phép chỉ định tựa (tiêu đề) sẽ xuất hiện ở Bảng Mục lục
%% và danh sách |bookmark|. Tuỳ chọn cũng giúp ta chỉ định hiệu ứng (|trans|)
%% dùng riêng cho |slide| đó thôi.
Phần tuỳ chọn này bao gồm bất kỳ tùy chọn nào được nói đến ở Mục~\vref{sec:glopts}
và một trong các tuỳ chọn sau đây:
%The \meta{options} for the |slide| environment can contain any
%option listed in section~\ref{sec:glopts}. Additionally, these allow
%the user to specify alternative titles for the table of contents and
%bookmark entries.

\begin{description}
\item\DescribeOption{toc}\option{toc}\\
%% When specified, the value is used for the entry in the table of
%% contents; otherwise, the slide title is used. If |toc=| is
%% specified, then no entry is created.
Chỉ định tựa sẽ xuất hiện trong Bảng Mục lục, thay vì tựa của |slide|.
Nếu dùng |toc=|, sẽ không có phần tử tương ứng nào được tạo ra ở
Bảng Mục lục.
\item\DescribeOption{bm}\option{bm}\\
%% When specified, the value is used for the bookmark entry; otherwise,
%% the slide title is used. If |bm=| is specified, then no entry is
%% created.
Tương tự như trên, nhưng cho danh sách |bookmark|.
%%<<dropped since 1.3>>
%% \item\DescribeOption{trans}\option{trans}\\
%% %% This works the same as the |trans| option described in
%% %% section~\ref{sec:pdsetup}, except that it sets the transition effect
%% %% of the current slide only (when used in the slide \meta{options})
%% %% and not for the entire presentation.
%% Cách dùng tùy chọn này hoàn toàn tương tự như mô tả ở Mục~\vref{sec:pdsetup};
%% các thay đổi mà tuỳ chọn này tạo ra chỉ có tác dụng trong |slide|
%% đang xét, không phải cho toàn bộ trình diễn hay các |slide| khác.
\end{description}

%% These optional arguments are especially useful when the title of a
%% slide is extremely long or when the title contains \LaTeX\ commands
%% that do not render correctly in the bookmarks.\footnote{The
%% bookmarking procedure uses \cs{pdfstringdef} from the \pf{hyperref}
%% package, and it can process accented characters such as \cs{"i}.}
%% When specifying entries, be sure to hide special characters `|,|'
%% and `|=|' between curly brackets `|{|' and `|}|'. Let's look at an
%% example that uses these optional arguments.
Các tuỳ chọn |toc| và |bm| đặc biệt hữu ích nếu tựa của |slide|
quá dài hoặc khi nó chứa các lệnh \LaTeX{} mà kết quả của lệnh đó
không thể hiện tốt trong danh sách |bookmark|.\footnote{Các bookmark
được tạo bằng cách dùng \cs{pdfstringdef} từ gói \pf{hyperref},
có thể chấp nhận vài ký tự có dấu chẳng hạn \cs{"i}.}
Khi dùng tuỳ chọn |toc| và |bm|, bạn nhớ đặt các ký tự đặc biệt
`|,|' và `|=|' bên trong cặp dấu ngoặc `|{|' và `|}|'. Dưới đây là ví dụ:

\begin{example}
 \begin{slide}[toc=,bm={LaTeX, i*i=-1}]{\color{red}\LaTeX, $i^2=-1$}
   My slide contents.
 \end{slide}
\end{example}

%% In this example, the slide title will appear as {\color{red}\LaTeX,
%% $i^2=-1$}. This text will not render correctly in a bookmark entry.
%% An attempt is made to correct this, but often, the correction does
%% not produce an equivalent text. This particular title would be
%% rendered in the bookmark list as |redLaTeX, i2=-1|. On the other
%% hand, the manually specified bookmark entry is rendered as:
%% |LaTeX, i*i=-1|. Notice, no entry is created in the table of contents,
%% because of the use of |toc=|.
Ở ví dụ này, tựa của |slide| sẽ là {\color{red}\LaTeX, $i^2=-1$}.
Kết quả này không thể nào thể hiện đúng trong danh sách |bookmark|.
Đôi khi, việc cố gắng sửa chữa tựa có thể mang lại kết quả, nhưng
thường thì cố gắng không như ý. Chính vì thế, ta đã dùng tuỳ chọn |bm|,
nhờ đó kết quả thu được ở danh sách |bookmark| sẽ là |LaTeX, i*i=-1|.
Chú ý rằng, không có tựa nào được tạo ra ở Bảng Mục lục, do ta đã chỉ
định |toc=|.

%% In addition to the |slide| environment, each individual style can
%% define its own environments. Many styles have a |wideslide|
%% environment. The idea is that one might have information that does
%% not fit nicely on a slide with a table of contents listed, as this
%% consumes some space. In such cases, it is preferable to use a slide
%% that does not list the table of contents. The |wideslide|
%% environment provides this functionality and has more space for the
%% actual slide content. See section~\ref{sec:styles} for information
%% on the various environments provided by the styles.
Cùng với môi trường |slide|, các kiểu trình diễn có cung cấp các
môi trường riêng để tạo |slide|. Phổ biến nhất là môi trường |wideslide|
để giải quyết vấn đề: đôi khi, nội dung của |slide| quá khổ, không thể
bố trí trên một trang cùng với Bảng Mục lục; môi trường |wideslide|
cho ta |slide| đặc biệt ở đó không xuất hiện Mục lục. Xem thêm Mục~\vref{sec:styles}
để biết thêm các môi trường cung cấp bởi các kiểu trình diễn.

\endinput
