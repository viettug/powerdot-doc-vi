
%\section{Miscellaneous}
\section{Linh tinh}

% =====================================================================

\subsection{\texorpdfstring{Ghi chú}{Ghi chu}}\label{sec:notes}

\DescribeEnv{note}
%% The |note| environment can be used to make personal notes that
%% accompany a slide. You can control displaying notes using the
%% |display| option (see section~\ref{sec:classopts}). Here is an
%% example.
Môi trường |note| dùng để có các ghi chú (cá nhân) đi cùng với trình diễn.
Có thể điều làm xuất hiện các ghi chú này nhờ tuỳ chọn |display|
đã nói ở Mục~\ref{sec:classopts}. Dưới đây là ví dụ
\begin{example}
 \begin{slide}{Chewing gum}
 ...
 \end{slide}
 \begin{note}{Reminder for chewing gum}
   Don't forget to mention that chewing gum is sticky.
 \end{note}
\end{example}

% =====================================================================

\subsection{Slide \texorpdfstring{trắng}{trang}}

\label{sec:emptyslides}

\DescribeEnv{emptyslide}
%% The |emptyslide| environment creates a totally empty slide. The text
%% box on the slide can be used for special things like displaying
%% photos. This allows for creating a dia show. Example:
Môi trường |emptyslide| tạo ra một |slide| trắng (không có nội dung gì).
Nhờ đó, bạn có thể nạp vào hình vẽ chẳng hạn. Nên nhớ rằng, nếu
không dùng |slide| trắng, bạn phải chỉ ra tựa đề của |slide|
và điều này sẽ ảnh hưởng đến Bảng Mục lục cũng như danh sách |bookmakr|.
Ví dụ
\begin{example}
 \begin{emptyslide}{}
   \centering
   \vspace{\stretch{1}}
   \includegraphics[height=0.8\slideheight]{me_chewing_gum.eps}
   \vspace{\stretch{1}}
 \end{emptyslide}
\end{example}
%% The |\includegraphics| command is defined by the \pf{graphicx}
%% package \cite{graphics}. The |\stretch| command is used to
%% vertically center the picture. Both commands are described in your
%% favorite \LaTeX\ manual, for instance \cite{companion}. Note that
%% you can use the lengths |\slideheight| and |\slidewidth| to scale
%% pictures to fit nicely on the slide.
Lệnh |\includegraphics| được lấy từ gói \pf{graphicx} \cite{graphics}.
Lệnh |\stretch| dùng để canh hình vẽ theo chiều đứng. Cả hai lệnh này
đưọc mô tả trong \cite{companion}. Chú ý rằng, các biến độ dài
|\slideheight| và |slidewidth| được dùng để bố trí hình vẽ
vừa vẹn lên |slide|.

% =====================================================================

\subsection{Slide \texorpdfstring{tài liệu tham khảo}{tai lieu tham khao}}

\label{sec:bib}

\DescribeEnv{thebibliography}
\pf{powerdot}
%% redefines the standard \pf{article}
%% |thebibliography| environment to suppress the creation of a section
%% heading and running headers. All other properties are maintained.
%% You can do either of the next two (depending whether you are
%% using BiB\TeX\ or not):\\
định nghĩa lại môi trường |thebibliography| của lớp chuẩn \pf{article}
để tạo trang tài liệu tham khảo. Sự khác biệt là môi trường mới không
tạo ra tựa đề và không tạo các dòng chữ ở đầu trang (|header|); các
tính chất khác đều được bảo toàn. Bạn có thể dùng một trong hai cách
sau đây (tuỳ thuộc bạn có dùng BiB\TeX{} hay không):\\
\begin{minipage}[t]{.49\linewidth}
\begin{example}
 \begin{slide}{Slide}
   \cite{someone}
  \end{slide}
 \begin{slide}{References}
   \begin{thebibliography}{1}
   \bibitem{someone} Article of someone.
   \end{thebibliography}
 \end{slide}
\end{example}
\end{minipage}\hfill
\begin{minipage}[t]{.49\linewidth}
\begin{example}
 \begin{slide}{Slide}
   \cite{someone}
 \end{slide}
 \begin{slide}{References}
   \bibliographystyle{plain}
   \bibliography{YourBib}
 \end{slide}
\end{example}
\end{minipage}

%% In case you have a big reference list that you want to spread over
%% multiple slides, have a look at the packages \pf{natbib} and
%% \pf{bibentry} \cite{natbib}. Using both packages allows you to do:
Trong trường hợp bạn có một danh sách rất dài các tài liệu tham khảo,
bạn có thể muốn chia danh sách đó thành nhiều |slide|, giống như cách làm
của gói \pf{natbib} và \pf{bibentry} \cite{natbib}. Việc dùng cả hai
gói đó sẽ làm bạn vừa ý:
\begin{example}
 \begin{slide}{References (1)}
   \bibliographystyle{plain}
   \nobibliography{YourBib}
   \bibentry{someone1}
   \bibentry{someone2}
 \end{slide}
 \begin{slide}{References (2)}
   \bibentry{someone3}
 \end{slide}
\end{example}
%% Have a look at your favorite \LaTeX\ manual for more information
%% about citations and bibliographies.
Nên tìm kiếm tài liệu về việc trích dẫn và các tạo danh sách
tài liệu tham khảo trong \cite{companion}.

% =====================================================================

%% \subsection{Verbatim on slides}
\subsection{Verbatim}

\label{sec:verbatim}\DescribeOption{verbatim}
%% \pf{powerdot} has three different methods of processing slides, from
%% which two have mainly been developed to make the inclusion of
%% verbatim content\footnote{And other content that needs catcode
%% changes when processing.} on slides easier. These methods can be
%% accessed by the |method| key which is available in slide
%% environments and the |\pdsetup| command (see
%% section~\ref{sec:glopts}).
Lớp \pf{powerdot} có ba cách khác nhau để xử lý các |slide|,
trong đó có hai cách dùng để điều khiển các môi trường tựa |verbatim|.
Các cách này được thay đổi nhờ tuỳ chọn |method|,
như đã nói ở ~\vref{sec:glopts}.
\begin{description}
\item\option{method=normal}\\
%% This is the default method for processing slides. It is fast and
%% allows for overlays, but it does not allow for
%% verbatim.\footnote{Except when it has been saved in a box outside
%% the slide.}
Đây là các xử lý thông thường, mặc định. Cách này nhanh, cho phép
|overlay| nhưng không cho phép các môi trường |verbatim|\footnote{Trừ khi
nội dung của |verbatim| được lưu trong một |box| bên ngoài |slide|.}.
\item\option{method=direct}\\
%% This method is also fast, but does not allow for overlays. Overlays
%% will silently be disabled. However, it does allow for verbatim
%% content on slides.
Cách này cũng nhanh, nhưng không cho phép |overlay|. Các thiết lập |overlay|
sẽ được bỏ qua tự động. Tuy nhiên, cách này cho phép sử dụng môi trường
tựa |verbatim|.
\item\option{method=file}\\
%% This method uses a temporary file to export the slide body to and
%% read it back in. This method does allow for verbatim content and
%% overlays, but could be slow when many slides use this method because
%% the filesystem is used.
Cách này sử dụng các tập tin tạm để lưu nội dung của |slide| rồi sau đó
đọc ngược trở lại. Cách này cho phép sử dụng đồng thời |overlay| và các
môi trường |verbatim|, nhưng lại khá chậm khi có nhiều |slide|.
\end{description}

%% Below is an example demonstrating the use of all three different
%% methods of slide processing.
Dưới đây là ví dụ cách dùng các |method|
\begin{example}
  \documentclass{powerdot}
  \usepackage{listings}
  \lstnewenvironment{code}{%
    \lstset{frame=single,escapeinside=`',
    backgroundcolor=\color{yellow!20},
    basicstyle=\footnotesize\ttfamily}
  }{}
  \begin{document}
  \begin{slide}{Slide 1}
    Normal \pause content.
  \end{slide}
  \begin{slide}[method=direct]{Slide 2}
    Steps 1 and 2:
  \begin{code}
    compute a;`\pause'
    compute b;
  \end{code}
  \end{slide}
  \begin{slide}[method=file]{Slide 3}
    Steps 1 and 2:
  \begin{code}
    compute a;`\pause'
    compute b;
  \end{code}
  \end{slide}
  \end{document}
\end{example}
%% The first slide shows the default behavior for normal content. It
%% produces two overlays. The second slide does not produce overlays,
%% despite the use of the |\pause| command. This command has been
%% disabled by choosing the |direct| method to process the verbatim
%% content. The third slide has the same body as the second slide, but
%% now does create two overlays, because the method using a temporary
%% file has been chosen. Notice that we used |\pause| inside the listing,
%% but that it can also be used outside the listing.
|Slide| đầu tiên sử dụng các xử lý mặc định, có hai |overlay|.
|Slide| thứ hai không có |overlay| nào, do trong đó ta cần thể hiện
môi trường |code| -- vì lý do đó, ta đã dùng |\pause| thay cho |overlay|.
|Slide| cuối cùng có nội dung tương tự như ở |slide| thứ hai, nhưng
lại có hai |overlay|. Chú ý rằng ta đã dùng |\pause| bên trong môi trường
|listing|, nhưng lệnh đó cũng có thể dùng bên ngoài |listing|.

% =====================================================================

%\subsection{The \cs{twocolumn} command}
\subsection{\texorpdfstring{Lệnh}{Lenh} \cs{twocolumn} \texorpdfstring{chia cột}{chia cot}}

\label{sec:twocolumn}

\DescribeMacro{\twocolumn}
%The |\twocolumn| macro allows to split content into two columns.
Lệnh |\twocolumn| cho phép bố trí nội dung ở hai cột của trang.
\begin{command}
 `\cs{twocolumn}\marg{options}\marg{left}\marg{right}'
\end{command}
%% This typesets \meta{left} and \meta{right} in two columns. The
%% dimensions of those columns can be controlled by \meta{options}.
%% Below are the available options.
Lệnh này sẽ bố trí \meta{left} và \meta{right} vào hai cột bên trái
và bên phải của |slide|. Kích thước của các cột được cho ở tuỳ chọn
\meta{options}.
\begin{description}
\item\DescribeOption{lineheight}\option{lineheight}\\
%% If |lineheight| is specified, a line of the specified height will be
%% created using |\psline| in between the two columns. Example:
%% |lineheight=6cm|.
Nếu |lineheight| được chỉ ra, một dòng kẻ (tạo ra nhờ lệnh |\psline|)
với chiều cao chỉ định được chèn để phân cách hai cột. Ví dụ |lineheight=6cm|.
\item\DescribeOption{lineprop}\option{lineprop}\\
%Any \pf{pstricks} declaration to specify the line properties. Example:
Các khai báo \pf{pstricks} để chỉ thuộc tính của dòng kẻ. Ví dụ
\begin{example}
 lineprop={linestyle=dotted,linewidth=3pt}
\end{example}
\item\DescribeOptions{lfrheight,lfrprop}\option{lfrheight} \option{lfrpop}\\
%% Creates a frame of the specified height around the left column.
Tạo ra một khung (|frame|) với chiều cao chỉ định xung quanh cột bên trái.
%% \item\DescribeOption{lfrprop}\option{lfrprop}\\
%As |lineprop|, but for the left frame.
Tuỳ chọn |lfrprop| cũng như |lineprop|, nhưng cho đường kẻ của khung bên trái.
\item\DescribeOptions{rfrheight,rfrprop}\option{rfrheight} \option{rfrprop}\\
%% Creates a frame of the specified height around the right column.
%Tạo khung bên phải với chiều cao chỉ định.
Như |lfrheight| và |lfrprop|, nhưng cho khung bên phải.
%\item\DescribeOption{rfrprop}\option{rfrprop}\\
% As |lineprop|, but for the left frame.
%Như |lineprop| nhưng dành cho đường kẻ của khung phải.
\item\DescribeOption{frsep}\option{frsep}\\
% Space between text and the frames. Default: |1.5mm|.
Khoảng cách giữa nội dung và khung (|frame|). Mặc định |1.5mm|.
\item\DescribeOption{colsep}\option{colsep}\\
% Space between the two columns. Default: |0.06\linewidth|.
Khoảng cách giữa hai cột. Mặc định |0.06\linewidth|.
\item\DescribeOptions{lcolwidth,rcolwidth}\option{lcolwidth} \option{rcolwidth}\\
%Width of the left column. Default: |0.47\linewidth|.
Chiều rộng của khung bên trái và phải. Mặc định |0.47\linewidth|.
%\item\DescribeOption{rcolwidth}\option{rcolwidth}\\
%Width of the right column. Default: |0.47\linewidth|.
%Chiều rộng của khung bên phải. Mặc định |0.47\linewidth|.
\item\DescribeOptions{topsep,bottomsep}\option{topsep} \option{bottomsep}\\
%% The extra space (additional to |\baselineskip|) between text above
%% the columns and the text within the columns. Default: |0cm|.
Khoảng trắng thêm vào bên trên cột và bên trên các dòng nội dung
(cộng thêm vào giá trị đã có của |\baselineskip|). Mặc định |0cm|.
Tương tự, |bottomsep| chỉ khoảng cách bên dưới cột và dòng.
%\item\DescribeOption{bottomsep}\option{bottomsep}\\
%Idem for the bottom of the columns. Default: |0cm|.
%Nhưng trên, nhưng cho bên dưới cột và dòng. Mặc định |0cm|.
\item\DescribeOption{indent}\option{indent}\\
%Horizontal indent left to the left column. Default: |0cm|.
Khoảng cách thụt đầu dòng so với cột bên trái.\footnote{Horizontal indent left to the left column}
Mặc định |0cm|.
\end{description}
%% The dimensions described above are represented graphically in
%% figure~\ref{fig:twocolumndim}.
Mô tả hình học của các tuỳ chọn trên được cho ở Hình~\vref{fig:twocolumndim}.
\begin{figure}[htb]
\centering
\begin{pspicture}(0,.5)(13,10.5)
\psline(0,0.5)(0,10)
\rput[tl](.05,9.95){Top}
\psframe[dimen=middle](1,9)(7,2)
\psline{C-C}(8.5,9)(11,9)
\psline{C-C}(8.5,2)(8.5,9)
\psline{C-C}(8.5,2)(11,2)
\qdisk(1.7,8.3){.1cm}
\psset{linestyle=dashed}
\psline{C-C}(1.7,8.3)(6.3,8.3)
\psline{C-C}(1.7,8.3)(1.7,3)
\psline{C-C}(6.3,5)(6.3,8.3)
\psline{C-C}(11,9)(12,9)
\psline{C-C}(11,2)(12,2)
\psline{C-C}(11,7)(12,7)
\psline{C-C}(9.2,8.3)(12,8.3)
\psline{C-C}(9.2,8.3)(9.2,3)
\rput[tl](1.75,8.25){Nội dung bên trái}
\rput[tl](9.25,8.25){Nội dung bên phải}
\rput[tl](.05,1){Bottom}
\psset{linestyle=dotted,dotsep=2pt}
\psline(0,8.3)(1.7,8.3)
\psline(0,9.6)(1,9.6)
\psline(0,2)(1,2)
\psline(0,1.1)(1,1.1)
\psset{linestyle=solid}
\psline{<->}(.2,8.33)(.2,9.57)
\psline{<->}(4,8.33)(4,8.97)
\psline{<->}(1.73,7)(6.27,7)
\psline{<->}(1.03,6.5)(1.67,6.5)
\psline{<->}(0.03,5.5)(1.67,5.5)
\psline{<->}(6.33,7.4)(9.17,7.4)
\psline{<->}(8.53,6.5)(9.17,6.5)
\psline{<->}(6.33,6.5)(6.97,6.5)
\psline{<->}(10.7,8.33)(10.7,8.97)
\psline{<->}(7.3,8.97)(7.3,2.03)
\psline{<->}(.2,1.13)(.2,1.97)
\psline{->}(1.7,9.3)(1.7,8.45)
\psline{<-}(9.23,7)(11,7)
\cput(4,6.6){\small 1}
\cput(11.1,6.6){\small 2}
\cput(8,7){\small 3}
\cput(7.7,3){\small 4}
\cput(4.4,8.65){\small 5}
\cput(1.35,6.1){\small 5}
\cput(8.85,6.1){\small 5}
\cput(11.1,8.65){\small 5}
\cput(6.65,6.1){\small 5}
\cput(0.6,8.95){\small 6}
\cput(0.6,5.1){\small 7}
\cput(0.6,1.55){\small 8}
\cput(1.7,9.6){\small 9}
\end{pspicture}
\begin{tabular}{c p{4cm}cl}
\multicolumn{4}{c}{Ý nghĩa của nhãn}\\\hline
1&|lcolwidth|&5&|frsep|\\
2&|rcolwidth|&6&|topsep|\\
3&|colsep|&7&|indent|\\
4&|lfrheight|, |rfrheight|,&8&|bottomsep|\\
&|lineheight|&9&Điểm tham khảo
\end{tabular}
\caption{Các kích thước ở chế độ  hai cột.}\label{fig:twocolumndim}
\end{figure}
%% Important to notice is that the |\twocolumn| macro uses the current
%% cursor position as the reference point to position the first line of
%% text of the left column (see also figure~\ref{fig:twocolumndim}). This
%% means that optional frames can extend to the text on the previous
%% line. Use for instance |topsep=0.3cm| in that case to add extra
%% space between the two lines of text. The default value of |topsep|
%% is based on the situation that there is no text on top of the two
%% columns. In that case, it is best to locate the first line of text
%% of the left column at the same spot as text that is not created by
%% |\twocolumn| on other slides. The setting |topsep=0cm| does exactly
%% this. However, with a combination of |topsep| and |indent| you can
%% change this behavior and position the first line of text of the left
%% column anywhere you want.
Điều chú ý quan trọng là, lệnh |\twocolumn| dùng vị trí hiện tại của
con trỏ như  là điểm tham khảo đến dòng đầu tiên của nội dung ở cột
bên trái (xem ở Bảng~\vref{fig:twocolumndim}). Điều này có nghĩa rằng,
khung được tạo ra sẽ ảnh hưởng đến dòng trước đó. Hãy dùng chẳng hạn |topsep=0.3cm|
để thêm khoảng trắng thích hợp vào giữa hai dòng này. Giá trị mặc định của |topsep|
là |0cm| được lấy với giả định rằng không có nội dung nào
trước khi bắt đầu chế độ hai cột. Trong trường hợp này, dòng đầu tiên
của cột bên trái có vị trí như các dòng (đầu tiên) tạo ở các |slide| khác.
Việc cài đặt |topsep=0cm| bảo đảm điều này. Tuy nhiên, việc phối hợp
hai giá trị |topsep| và |indent| cho phép bạn định vị dòng đầu tiên của
cột bên trái theo ý bạn.

%% The |\twocolumn| macro computes the height of the construction to
%% position text below the construction correctly. The computation is
%% done by taking the maximum height of |lfrheight|, |rfrheight|,
%% |lineheight| (if requested) and the left and right column content.
%% Hence when frames nor a line is requested, |bottomsep| is the
%% vertical space between the lowest line of text in the columns and
%% the text below the columns (additional to |\baselineskip|). Here is
%% an example.
Lệnh |\twoclumn| tính toán chiều cao của cột dựa trên nội dung của nó,
nhờ đó mới có thể bố trí đúng nội dung vào các cột. Việc tính toán này
đưọc tinh chỉnh nhờ lấy giá trị lớn nhất trong hai giá trị |lfrheight| và |rfrheight|.
\begin{example}
 \begin{slide}{Two columns}
   Here are two columns.
   \twocolumn{lfrprop={linestyle=dotted,linewidth=3pt},
     lfrheight=4cm,rfrheight=5cm,lineheight=3cm,topsep=0.3cm
   }{left}{right}
   That were two columns.
 \end{slide}
\end{example}
\endinput
