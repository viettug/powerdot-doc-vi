
\documentclass[a4paper]{ltxdoc}

\usepackage[utf8x]{vietnam}
\usepackage{ktv-buildnum,hyperref,indentfirst,url}

\input{pdpream.ble}

\def\fileversion{v1.0}
\def\filedate{2005/09/04}

%\let\Section\section\def\section*#1{\Section*{#1}\addcontentsline{toc}{section}{#1}}

\bibliographystyle{plain}
\bibliography{powerdot}

\begin{document}

\section*{Acknowledgements}

The authors are grateful to Darren Dale and Herbert Vo\ss\ for
testing the class and its output. Further, we like to thank all
style contributors (see section~\ref{sec:styles}) and the testers of
the first beta releases.

\PrintChangesX\PrintIndexX

\title{\vspace*{-2cm}\mktitledecor The \pf{powerdot} class
\thanks{This class can be downloaded from the CTAN mirrors:
\texttt{/macros/latex/contrib/powerdot}. See \texttt{powerdot.dtx} for
information on installing \pf{powerdot} into your \LaTeX\
distribution and for the license of this class.}}
\author{Hendri Adriaens\and Christopher Ellison
	\\\and Biên dịch: kyanh <\url{mailto:kyanh@o2.pl}>}
\date{Bản dịch số {\the\buildnum}}
\maketitle

\begin{abstract}
\noindent
%% \pf{powerdot} is a presentation class for \LaTeX\ that allows for
%% the quick and easy development of professional presentations. It
%% comes with many tools that enhance presentations and aid the
%% presenter. Examples are automatic overlays, personal notes and a
%% handout mode. To view a presentation, DVI, PS or PDF output can be
%% used. A powerful template system is available to easily develop new
%% styles.
\pf{powerdot} là lớp \LaTeX\ cho phép tạo trình diễn nhanh chóng,
chuyên nghiệp với kiểu dáng dễ dàng thay đổi. Lớp cung cấp nhiều công cụ
giúp thiết kế trình diễn: overlay, ghi chú cá nhân, chế độ handout.
Để xem trình diễn, bản DVI, PS, PDF của tài liệu đều có thể dùng được.
Lớp cung cấp hệ thống mẫu mạnh, giúp phát triển dễ dàng các mẫu mới.
\end{abstract}

\begin{multicols}{2}
[\section*{Contents}
\setlength{\columnseprule}{.4pt}
\setlength{\columnsep}{18pt}]
\tableofcontents
\end{multicols}

\newpage
\section{\texorpdfstring{Giới thiệu}{Gioi thieu}}\label{sec:intro}

%% This class gives you the possibility to easily create professionally
%% looking slides. The class is designed to make the development of
%% presentations as simple as possible so that you can concentrate on
%% the actual content instead of keeping yourself busy with technical
%% details. Of course, some knowledge of \LaTeX\ is still required
%% though.
Lớp \pf{powerdot} cung cấp khả năng tạo trình diễn dễ dàng, chuyên nghiệp.
Lớp được thiết kế để việc thiết kế trình diễn trở nên đơn giản nhất có thể,
nhờ đó bạn không phải tốn thời gian với các yếu tố kỹ thụât.
Tất nhiên, bạn phải có các kiến thức cơ bản về \LaTeX{}.

%% This class builds on and extends the \pf{prosper} class
%% \cite{prosper} and the \pf{HA-prosper} package \cite{HA-prosper}.
%% The \pf{HA-prosper} package was initially intended to extend
%% \pf{prosper} and correct some bugs and problems of that class. As
%% developments on that package progressed, it was found that
%% unfortunately, not all of the problems could be overcome with the
%% package. That discovery was the start of a new project set up to
%% make a new class to replace the \pf{prosper} plus \pf{HA-prosper}
%% combination. You're currently reading the result of that project.
Lớp được xây dựng nhờ mở rộng lớp \pf{prosper} \cite{prosper}
và gói \pf{HA-prosper} \cite{HA-prosper}. Gói \pf{HA-prosper}
có mục đích ban đầu là mở rộng và khắc phục vài lỗi, nhược điểm của lớp
\pf{prosper}. Thật không may là, không phải mọi nhưọc điểm của \pf{propser}
đều có thể khắc phục được. Chính vì lý do này, một dự án mới ra đời
nhằm thay thế cho cả \pf{prosper} và \pf{HA-prosper}.
Bạn đang đọc tài liệu về chính dự án đó ;)

%% The remainder of this section will be devoted to giving a feel of
%% what the \pf{powerdot} presentation source looks like and giving an
%% overview of this documentation.
Phần còn lại của mục này giúp bạn có cái nhìn tổng quan về lớp \pf{powerdot}
và tài liệu hướng dẫn này.

%% The document structure of a presentation is always the same. You can
%% find it in the example below.
Cấu trúc của trình diễn luôn tương tự như ví dụ sau đây:

\begin{example}
 \documentclass[<class options>]{powerdot}
 \pdsetup{<pd options>}
 \begin{document}
   \begin{slide}{slide}
     noi dung
   \end{slide}
   \section{section}
   \begin{slide}[<slide options>]{slide}
     noi dung
   \end{slide}
   \begin{note}{ghi chu ca nhan}
     ghi chu
   \end{note}
 \end{document}
\end{example}

%% There are several elements that define the document structure. First
%% of all, the class accepts some class options that control the output
%% of the class, for instance, paper type and style. These class
%% options will be discussed in section~\ref{sec:classopts}. Then there
%% are presentation specific options which control some of the elements
%% of the presentation globally, for instance, the footers. These will
%% be discussed in section~\ref{sec:pdsetup}.
Có vài yếu tố tạo nên cấu trúc đó.
Đầu tiên, lớp chấp nhận vài tuỳ chọn (|class options|) cho phép điều khiển kết quả xuất
ví dụ, cỡ giấy, kiểu. Các tuỳ chọn này được bàn kỹ đến trong Mục~\ref{sec:classopts}.
Thứ đến, là các tuỳ chọn trình diễn (|pd options|) điều khiển toàn cục các
tính chất của trình diễn, ví dụ, các ghi chú ở chân trang.
Những tuỳ chọn này đưọc nói đến ở Mục~\ref{sec:pdsetup}.

%% Once the setup has been decided on, you can use the slide
%% environment to produce slides (see section~\ref{sec:slides}) and the
%% note environment to produce notes that go with the slides (see
%% section~\ref{sec:notes}). You can use overlays to display material
%% in steps. This is described in section~\ref{sec:overlays}. The
%% |\section| command provides a way to structure your presentation.
%% This is discussed in section~\ref{sec:structure}.
%% Section~\ref{sec:styles} will show an overview of the styles that
%% come with this class and the characteristics of each style.
%% Section~\ref{sec:compiling} will tell you more about how to produce
%% output. This section contains important information on required
%% packages.
Sau khi thiết lập với các tuỳ chọn, bạn có thể dùng môi truờng |slide|
để tạo các trang (|slide|) trình diễn (xem Mục~\ref{sec:slides}) và môi trường |note|
để tạo các ghi chú đi cùng với |slide| (xem Mục~\ref{sec:notes}).
Bạn có thể dùng |overlay| để thể hiện nội dung theo từng bước
(xem Mục~\ref{sec:overlays}). Lệnh |\section| giúp bạn tạo cấu trúc
cho trình diễn, giống như việc tạo chương, mục với tài liệu \LaTeX{} thông thường
(xem Mục~\ref{sec:structure}). Bạn cũng có thể lựa chọn các kiểu dáng
của trình diễn sau khi xem Mục~\ref{sec:styles}. Cuối cùng,
với Mục~\ref{sec:compiling}, bạn sẽ biết cách biên dịch tài liệu nguồn
để có kết quả là trình diễn thật sự. Ở mục này cũng có vài lưu ý về
việc cài đặt lớp \pf{powerdot}.

%% Section~\ref{sec:writestyle} is mostly interesting for people that
%% want to develop their own style for this class or want to modify
%% an existing style. This documentation concludes with a section
%% devoted to questions (section~\ref{sec:questions}), like `Where can
%% I find examples?'. It also tells you where to turn to in case your
%% questions are still not solved.
Mục~\ref{sec:writestyle} có lẽ là phần hấp dẫn đối với ai quan tâm
đến việc tạo kiểu dáng riêng cho trình diễn của mình, hoặc làm đẹp
các kiểu dáng đã có.

Tài liệu này kết thúc với Hỏi-Đáp (Mục~\ref{sec:questions}), có thể
giúp bạn trong những bước đầu làm quen với lớp \pf{powerdot}.

\section{Setting up the presentation}\label{sec:setup}
This section will describe all options that are available to control
the output of the presentation and the looks of it.

\subsection{Document class options}\label{sec:classopts} We will
start with the class options that are typed in the |\documentclass|
command as a comma-separated list. For each option, the default
value will be mentioned in the description. This is the value that
will be used if you decide to not give a value to the option or not
use the option at all.

\DescribeOption{mode}
This options controls the kind of output that we want to produce.
The default value is |present|.
\begin{description}
\item\option{mode=present}\\
This mode is used when you want to create the actual presentation. It
will enable overlays and transition effects. You can read more about
overlays in section~\ref{sec:overlays}.
\item\option{mode=print}\\
This mode can be used when printing the slides including their visual
markup, but without any overlay or transition effects.
\item\option{mode=handout}\\
This mode will produce a black and white overview of your slides that
can be used to make personal notes on, for distribution to students,
a personal guide during your talk, etcetera.
\begin{description}
\item\option{nopagebreaks}\\
By default, the handout mode produces a document with two slides per
page. If you want to fit more slides on a page, specify this option
in the |\documentclass| command and \pf{powerdot} will let \LaTeX\
decide on the places to insert a page break, namely when a page is
full.
\end{description}
\end{description}

\DescribeOption{paper} This option has three possible values. The default
value is |screen|.
\begin{description}
\item\option{paper=screen}\\
This is a special format with screen optimized ratio (4/3). The
actual page dimensions will be 8.25 inch by 11 inch. This paper format
is not available for print or handout mode. In these modes, \pf{powerdot}
will switch to a4 paper and put a warning that it did this in the
log file of your presentation.
\item\option{paper=a4paper}\\
A4 paper will be used for the presentation or handout.
\item\option{paper=letterpaper}\\
Letter size paper will be used.
\end{description}
Some important information with respect to paper size, compiling and
viewing presentations is available in section~\ref{sec:compiling}.

\DescribeOption{orient} This controls the orientation of the
presentation. The default value is |landscape|.
\begin{description}
\item\option{orient=landscape}\\
The presentation will be in landscape format. This value is not
available in handout mode. In that mode, \pf{powerdot} will switch
to portrait orientation and will warn you about this in the log
file.
\item\option{orient=portrait}\\ This produces slides in portrait
format. Notice that not all styles support portrait orientation. Please
refer to section~\ref{sec:styles} for information about which styles
do support the portrait orientation.
\end{description}

\DescribeOption{display} This controls the production of slides and
notes. The default value is |slides|.
\begin{description}
\item\option{display=slides}\\
This will only typeset the slides in your presentation.
\item\option{display=slidesnotes}\\
This will typeset both the slides and the notes in your
presentation. See also section~\ref{sec:notes} for more information
about notes.
\item\option{display=notes}\\ This will typeset the notes only. To
be able to typeset the slide numbers of the notes correctly, one should
first run the presentation in slidesnotes mode once.
\end{description}

Here are some more options to control the output.
\begin{description}
\item\DescribeOption{size}\option{size}\\
This is the size of the normal text font in points. Possible values
are 8, 9, 10, 11, 12, 14, 17, 20 and the default value is
11.\footnote{Note that sizes other than 10, 11 and 12 are
non-standard and it is assumed that you have the \pf{extsizes}
bundle \cite{extsizes} installed, which provides these sizes.}
\item\DescribeOption{style}\option{style}\\
This controls the style to be loaded for the presentation. By
default, the \pf{default} style will be loaded. For more styles, see
section~\ref{sec:styles}.
\item\DescribeOption{fleqn}\option{fleqn}\\
This option makes equations flushed left. It does the same as the
equally named option for the article class.
\item\DescribeOption{leqno}\option{leqno}\\
Put equation numbers at the left. Also the same as in the article
class.
\item\DescribeOption{nopsheader}\option{nopsheader}\\
By default, \pf{powerdot} will write a postscript command to the ps
file to make sure that post processors like ps2pdf know which paper
to use without the need to specify it on the command line. See also
section~\ref{sec:compiling}. If you experience problems with post
processing or printing or you want to specify the paper size in the
post processing steps yourself, use this option.
\item\DescribeOption{hlentries}\option{hlentries}\\
This highlights table of contents entries when the entry matches
with the current slide and is |true| by default. See also
section~\ref{sec:structure}. If you don't want highlighting of table
of contents entries (for instance in print mode), use
|hlentries=false|.
\item\DescribeOption{hlsections}\option{hlsections}\\
This highlights table of contents sections when the section matches
with the current section in the presentation and is |false| by
default. See also section~\ref{sec:structure}. Specifying this
option turns highlighting of sections on. This could be useful when
you are using a style that implements a split table of contents.
\item\DescribeOption{blackslide}\option{blackslide}\\
This option inserts a black slide in the presentation on page 1 and
will automatically advance to page 2 when opening the presentation
in a PDF viewer like Acrobat (Reader). The black slide has an
embedded target called |blackslide| and you can make a clickable
link to this slide by using
\begin{example}
 \hyperlink{blackslide}{Click here to go to the black slide}
\end{example}
When you click anywhere in the black slide, you will go back to the
originating slide. This option can be used to temporarily pause a
presentation, for instance, to do a proof on the black board.
\end{description}

Here is an example of a |\documentclass| command.
\begin{example}
 \documentclass[
   size=12,
   paper=screen,
   mode=present,
   display=slidesnotes,
   style=tycja,
   nopagebreaks,
   blackslide,
   fleqn
 ]{powerdot}
\end{example}
This example sets up a presentation in \pf{tycja} style, with a black
slide, normal size 12 points and flushed left equations.
\begin{example}
 \documentclass[
   size=12,
   paper=letterpaper,
   mode=handout,
   display=slidesnotes,
   style=tycja,
   nopagebreaks,
   blackslide,
   fleqn
 ]{powerdot}
\end{example}
Changing the |paper| and |mode| options, now produces a handout with
possibly more than two slides per page due to the |nopagebreaks|
option.

\subsection{Setup options}\label{sec:pdsetup}
\DescribeMacro{\pdsetup}
There are several extra options that can help customizing your
presentation. These options are not available via the
|\documentclass| command. This has a technical reason.\footnote{The
interested reader is referred to the section about the \pf{xkvltxp}
package in the \pf{xkeyval} package documentation \cite{xkeyval}.}
The options can be accessed via the |\pdsetup| command, which can
only be used in the preamble of your presentation. The command takes
one argument, which should contain a comma-separated list of
options. The available options are described below.

\begin{description}
\item\DescribeOption{lf}\option{lf}\\
This determines the content of the left footer. By default, this is
empty.
\item\DescribeOption{rf}\option{rf}\\
This determines the content of the right footer. By default, this is
empty.
\item\DescribeOption{theslide}\option{theslide}\\
This option controls how the slide number appears on the slide. By
default this has the value |\arabic{slide}~/~\pageref*{lastslide}|,
which could appear like |5/22|. Notice that the |\arabic{slide}|
typesets the number of the current slide and that
|\pageref*{lastslide}| typesets the number of the last
slide.\footnote{We use the starred version of \cs{pageref} which is
defined by \pf{hyperref} and does not create a link to the page that
it is referring to.}
\item\DescribeOption{thenote}\option{thenote}\\
This is similar to the |theslide| option, but typesets the slide
numbers of notes. The default value is
|note~\arabic{note}~of~slide~\arabic{slide}| and |\arabic{note}|
here typesets the number of the current note that goes with the
current slide. This could appear like |note 2 of slide 7|.
\item\DescribeOption{trans}\option{trans}\\
This option sets the default transition effect to be used in the
presentation. These transition effects only work after compiling the
presentation to PDF format. See also section~\ref{sec:compiling}.
The following transition effects are supported: |Split|, |Blinds|,
|Box|, |Wipe|, |Dissolve|, |Glitter| and |Replace|. When you are using
a viewer that understands PDF 1.5, you can also use |Fly|, |Push|,
|Cover|, |Uncover| or |Fade|. It is important to notice that most
viewers are case sensitive, so, for instance, |box| will not work.

The default effect is |Replace| which just replaces one slide with
another when browsing the slides. Note that some PDF viewers (like
Acrobat Reader 5 and higher) only produce the transition effect in
full screen mode. If you want to use a custom transition effect that
is not listed in the list above (for instance, a wipe effect with a
custom wipe direction), then that is possible. However,
\pf{powerdot} will put a warning in your log file that the effect
that you have chosen, might not work in the PDF viewer. Here is an
example that does work.
\begin{example}
 trans=Wipe /Di 0
\end{example}
In Acrobat (Reader), this wipes from left to right instead of the
default top to bottom. For more information, see a PDF Reference
Manual.
\item\DescribeOption{counters}\option{counters}\\
The |counters| option lists counters that you might want to protect
on overlays. As material on overlays (see
section~\ref{sec:overlays}) is processed multiple times, also
\LaTeX\ counters, like the |equation| counter, might be increased
too often. To avoid that your equations get different numbers on
every overlay, use this option. The |equation|, |table|, |figure|,
|footnote| and |mpfootnote| counters are already protected for you.
If you use extra counters, for instance for theorems, list them in
this option. Example:
\begin{example}
 counters={theorem,lemma}
\end{example}
\item\DescribeOption{list}\option{list}\\
This option takes a list of options that will be passed on to the
\pf{enumitem} package that controls the layout of lists created by
the |enumerate| and |itemize| environments. Example:
\begin{example}
 list={labelsep=1em,leftmargin=*,itemsep=0pt,topsep=5pt,parsep=0pt}
\end{example}
See for more information on controlling the layout of lists the
\pf{enumitem} package \cite{enumitem}.
\item\DescribeOption{enumerate}\option{enumerate}\\
As the |list| option, but only control |enumerate| environments.
\item\DescribeOption{itemize}\option{itemize}\\
As the |list| option, but only control |itemize| environments.
\end{description}

Here is an example of a |\pdsetup| command that one could use to set up
the presentation.
\begin{example}
 \pdsetup{
   lf=My first presentation,
   rf=For some conference,
   trans=Wipe,
   theslide=\arabic{slide}
 }
\end{example}
This sets the left and right footers and will initialize the
transition effect to |Wipe|. Further, slide numbers will not include
the number of the last slide, but only the number of the current
slide.

A small note is necessary with respect to the appearance of footers.
The slide number (controlled by the |theslide| option) will be added
to a footer. Most styles add it too the right footer. If both the
footer and the slide number are non empty, |~--~| will be inserted
in between them to separate them. Styles might modify this default
behavior however.

\section{Making slides}\label{sec:slides}
\subsection{The title slide}\label{sec:titleslide}
\DescribeMacro{\title}
\DescribeMacro{\author}
\DescribeMacro{\and}
\DescribeMacro{\date}
\DescribeMacro{\maketitle}
The title slide is created by the |\maketitle| command. Its use is
the same as in the standard \LaTeX\ document classes. See an example
below.
\begin{example}
 \documentclass{powerdot}
   \title{Title}
   \author{You \and me}
   \date{August 21, 2005}
 \begin{document}
    \maketitle
    ...
 \end{document}
\end{example}
The |author|, |title| and |date| declarations provide the text to be
used when making a title page. The design of the title page is
specific to the style in use. Notice the use of |\and| for
separating multiple authors. See a \LaTeX\ manual \cite{companion}
for more information on commands such as |\title| and |\author|.

\subsection{Other slides}\label{sec:otherslides}
\DescribeEnv{slide} The centerpiece of every presentation is the
slide. In \pf{powerdot}, the content of each slide is placed in a
|slide| environment.
\begin{command}
 `\cs{begin}\texttt{\{slide\}}\oarg{options}\marg{slide title}'
 `\meta{body}'
 `\cs{end}\texttt{\{slide\}}'
\end{command}

In section~\ref{sec:overlays} we'll see how to give some life to the
slides, but for now, let's look at a simple example.
\begin{example}
 \begin{slide}{First slide}
   Hello World.
 \end{slide}
\end{example}
The slide environment has one required argument, namely the slide
title. When a slide is created, the slide title is used to create an
entry in the table of contents and in the list of bookmarks. The
table of contents is a listing of the slides and section titles in
the presentation that appears on each slide.

The table of contents is clickable (when the presentation is
compiled into PDF) and serves as a nice way to jump from location to
location within the presentation. The bookmark list is only present
when compilation is taken all the way to the PDF file format. It
also serves as a table of contents, but this list does not appear on
\textit{any} of the slides, but in a separate window in a PDF
viewer. In the example above, the entries in both table contents and
the list of bookmarks would be titled |First slide|.

The \meta{options} for the |slide| environment allow the user to
specify alternative titles for the table of contents and bookmark
entries. There is also a |trans| option that works only for the
current slide.
\begin{description}
\item\DescribeOption{toc}\option{toc}\\
When specified, the value is used for the entry in the table of
contents; otherwise, the slide title is used. If |toc=| is
specified, then no entry is created.
\item\DescribeOption{bm}\option{bm}\\
When specified, the value is used for the bookmark entry; otherwise,
the slide title is used. If |bm=| is specified, then no entry is
created.
\item\DescribeOption{trans}\option{trans}\\
This works the same as the |trans| option described in
section~\ref{sec:pdsetup}, except that it sets the transition effect
of the current slide only (when used in the slide \meta{options})
and not for the entire presentation.
\end{description}

These optional arguments are especially useful when the title of a
slide is extremely long or when the title contains \LaTeX\ commands
that do not render correctly in the bookmarks.\footnote{The
bookmarking procedure uses \cs{pdfstringdef} from the \pf{hyperref}
package, and it can process accented characters such as \cs{"i}.}
When specifying entries, be sure to hide special characters `|,|'
and `|=|' between curly brackets `|{|' and `|}|'. Let's look at an
example that uses these optional arguments.
\begin{example}
 \begin{slide}[toc=,bm={LaTeX, i*i=-1}]{\color{red}\LaTeX, $i^2=-1$}
   My slide contents.
 \end{slide}
\end{example}

In this example, the slide title will appear as {\color{red}\LaTeX,
$i^2=-1$}. This text will not render correctly in a bookmark entry.
An attempt is made to correct this, but often, the correction does
not produce an equivalent text. This particular title would be
rendered in the bookmark list as |redLaTeX, i2=-1|. On the other
hand, the manually specified bookmark entry is rendered as:
|LaTeX, i*i=-1|. Notice, no entry is created in the table of contents,
because of the use of |toc=|.

In addition to the |slide| environment, each individual style can
define its own environments. Many styles have a |wideslide|
environment. The idea is that one might have information that does
not fit nicely on a slide with a table of contents listed, as this
consumes some space. In such cases, it is preferable to use a slide
that does not list the table of contents. The |wideslide|
environment provides this functionality and has more space for the
actual slide content. See section~\ref{sec:styles} for information
on the various environments provided by the styles.

\section{Overlays}\label{sec:overlays}
It is often the case that you don't want all the information on the
slide to appear at once. Rather, the information should appear one
item at a time. In \pf{powerdot}, this is achieved with overlays.
Each slide can be comprised of many overlays, and the overlays are
displayed one at a time.

\subsection{The \cs{pause} command}\label{sec:pause}
\DescribeMacro{\pause} The easiest way to display information
sequentially is to use the |\pause| command.
\begin{command}
 `\cs{pause}\oarg{number}'
\end{command}
Below is a simple example:
\begin{example}
 \begin{slide}{Simple overlay}
   power\pause dot
 \end{slide}
\end{example}
The slide's information is displayed and continues until the
|\pause| command is encountered. No further output within the same
slide is displayed until the click of the mouse or the touch of the
keyboard. Then, the content will continue to display until all the
information is displayed or until another |\pause| command is
encountered. In this example, |power| is displayed on the first
overlay, and |powerdot| is the displayed on the second overlay.  The
|\pause| command is often used within the |itemize| and |enumerate|
environments. For example,
\begin{example}
 \begin{slide}{Multiple pauses}
   power\pause dot \pause
   \begin{itemize}
     \item Let me pause\ldots \pause
     \item \ldots while I talk \pause and chew bubble gum. \pause
     \item Perhaps you'll be persuaded.
     \item Perhaps not.
   \end{itemize}
 \end{slide}
\end{example}
Since |\pause| was used before the |itemize| environment, no item
will appear until the third overlay. Then, each item will be
displayed one at a time, each on their own overlay. More information
on using lists will follow in the next section.

The optional argument of the |\pause| command specifies the number
of overlays to pause. An example usage is:
\begin{example}
 \begin{slide}{Pause longer}
   \begin{itemize}
     \item A \pause
     \item B \pause[2]
     \item C
   \end{itemize}
 \end{slide}
\end{example}
In the example above, item |C| will appear on the fourth overlay.
The usefulness of this option will become more apparent in the next
section; so we will revisit a similar example at that time.

\subsection{List environments}\label{sec:lists}
The list environments, |itemize| and |enumerate|, have special
treatments in \pf{powerdot}. They have an optional argument that
will be taken care off by the \pf{enumitem} package (see
\cite{enumitem}). \pf{powerdot} supplies an extra key for this
optional argument. In the examples that follow, features will be
described using the |itemize| environment but they also apply to the
|enumerate| environment.

Here is the typical usage of the |itemize| environment:
\begin{example}
 \begin{slide}{Basic itemize}
   \begin{itemize}
     \item A \pause
     \item B \pause
     \item C
   \end{itemize}
 \end{slide}
\end{example}
The display is simple, each item appears one at a time with each
overlay.

\DescribeOption{type}
Suppose we wanted every item to show, but we only wanted one item to
appear `active' at once. This can be accomplished via the |type|
option for the |itemize| environment. The default value is |0|.
\begin{example}
 \begin{slide}{Type 1 itemize}
   \begin{itemize}[type=1]
     \item A \pause
     \item B \pause
     \item C
   \end{itemize}
 \end{slide}
\end{example}
Now, every item will be displayed in the \emph{inactive
color}\index{inactive color} (which is defined by the style that you
use), and the item's font color will become the active one on the
overlay that it would normally appear on. The default behavior is
given by |type=0|.

Lists can also be nested to create complicated structures. When a
list is nested, it inherits the setting of the |type| option from
the `parent' list, but that can be overruled by specifying the
|type| option in the optional argument of the nested list. We
present here one example, but many more can be created by nesting
lists of different types in different ways.
\begin{example}
 \begin{slide}{Nested lists}
   \begin{itemize}
     \item A\pause
     \begin{itemize}[type=1]
       \item B\pause
     \end{itemize}
     \item C
   \end{itemize}
 \end{slide}
\end{example}
This displays |A| and |B| on the first overlay, but |B| is inactive.
On overlay 2, |B| will become active and on overlay 3, |C| will
become visible.

\subsection{The \cs{item} command}
\DescribeMacro{\item}
The |\item| command has an extra \emph{optional} argument in
\pf{powerdot} which allows for creating overlays in a more flexible
way then |\pause| provides.
\begin{command}
 `\cs{item}\oarg{label}\larg{overlays}'
\end{command}
This optional argument should contain an overlay specification
stating on which overlays you want the item to appear. This
specification is a comma separated list where each item can used the
notation as in table~\ref{tab:item}.
\begin{table}[htb]\centering
\begin{tabular}{d}
Syntax&Meaning\\\hline
\texttt{x}&Only overlay \texttt{x}\\
\texttt{-x}&All overlays up to and including \texttt{x}\\
\texttt{x-}&All overlays from \texttt{x}, including \texttt{x}\\
\texttt{x-y}&All overlays from \texttt{x} to \texttt{y},
including \texttt{x} and \texttt{y}\\
\end{tabular}
\caption{\cs{item} and \cs{onslide} notation}\label{tab:item}
\end{table}
The \meta{label} argument is the standard optional argument for
|\item| in \LaTeX. A \LaTeX\ manual \cite{companion} can tell you
more about this argument.

Here is an example.
\begin{example}
 \begin{slide}{Active itemize}
   \begin{itemize}[type=1]
    \item<1> A
    \item<2> B
    \item<3> C
   \end{itemize}
 \end{slide}
\end{example}
Here we have said that |A| should only be active on overlay 1, |B|
should only be active on overlay 2, and |C| should only be active on
overlay 3. Again, when the item is not active, it appears in the
inactive color because of |type=1|.

If |type=0| is specified and if each item is given an overlay
option, then each item will appear only when it is active. When the
item is not active, then it will not show on the slide at all. More
examples demonstrating the syntax for \meta{overlays} will be
discussed in the next section.

\subsection{The \cs{onslide} command}\label{sec:onslide}
\DescribeMacro{\onslide} Overlays can also be achieved using the
|\onslide| command.
\begin{command}
 `\cs{onslide}\marg{overlays}\marg{text}'
\end{command}
This command takes an \meta{overlays} specification as first
argument and the \meta{text} to apply it to as second argument. The
\meta{overlays} on which the text will appear are specified as a
comma separated list with syntax as in table~\ref{tab:item}. We
start off with a simple example.
\begin{example}
 \begin{slide}{Simple onslide}
   \onslide{1,2}{power}\onslide{2}{dot}
 \end{slide}
\end{example}
We have instructed |power| to appear on overlays one and two, and
|dot| to appear only on overlay two. As you might guess, this
example has the same output as our first |\pause| example. Yet, it
is clearly the case that our syntax is more complicated. However,
this slight ``complication'' also allows for much more flexibility.

\DescribeMacro{\onslide+}Consider the above example with the
following modifications:
\begin{example}
 \begin{slide}{Simple onslide+}
  \texttt{onslide }: \onslide{1}{power}\onslide{2}{dot}\\
  \texttt{onslide+}: \onslide+{1}{power}\onslide+{2}{dot}
 \end{slide}
\end{example}
The |\onslide+| command displays its content in a different manner
altogether. Now, |dot| appears on every overlay, but it is in
inactive color\index{inactive color} and matches the normal font
color \textit{only} on overlay two. This is comparable to the
|type=1| behavior for lists (see section~\ref{sec:lists}).

When executing this example, we will also notice that the |\onslide|
command does hide material, but still reserves the right amount of
space for it: on overlay 2, the |dot|s appear right above each
other. The next command does not reserve space.

\DescribeMacro{\onslide*} Instead of hiding and reserving space
(|\onslide|) or putting \meta{text} in the inactive color
(|\onslide+|) when the overlay doesn't match \meta{overlays}, this
command just eats the material altogether. To understand the
differences, consider the following example:
\begin{example}
 \begin{slide}{Simple onslide*}
  \texttt{onslide }: \onslide{1}{power}\onslide{2}{dot}\\
  \texttt{onslide+}: \onslide+{1}{power}\onslide+{2}{dot}\\
  \texttt{onslide*}: \onslide*{1}{power}\onslide*{2}{dot}
 \end{slide}
\end{example}
The output of the first two lines, we are already familiar with. The
third line displays |power| on overlay 1 and |dot| on overlay 2, but
no space for |power| is reserved on overlay 2. Hence |dot| will
start on the cursor position that |power| started on overlay 1 and
it is not aligned below the other two |dots|.

We finish with an example of the syntax that is possible with
|\item| and |\onslide|. Remember that these commands take a comma
separated list for the \meta{overlays} specification and that each
element can used the syntax as explained in table~\ref{tab:item}.
The various variations are demonstrated in the example below.
\begin{example}
 \begin{slide}{Lists}
   \onslide{10}{on overlay 10 only}\par
   \onslide{-5}{on every overlay before and including overlay 5}\par
   \onslide{5-}{on every overlay after  and including overlay 5}\par
   \onslide{2-5}{on overlays 2 through 5, inclusive}\par
   \onslide{-3,5-7,9-}{on every overlay except overlays 4 and 8}
 \end{slide}
\end{example}

\subsection{Relative overlays}
Sometimes it is a pain to keep track of when an item should appear
or become active. You might, for example, just care that some text
appears on the overlay \textit{after} some other item. This
functionality is provided through the use of relative overlays which
should not be used outside list environments that use |\item|. Let's
consider a simple, illuminating example.
\begin{example}
 \begin{slide}{Relative overlays}
   \begin{itemize}
     \item A \pause
     \item B \onslide{+1}{(visible 1 overlay after B)}\pause
     \item C \onslide{+2-}{(appears 2 overlays after C, visible until the end)}
     \pause
     \item D \onslide{+1-6}{(appears 1 overlay after D, visible until overlay 6)}
     \pause
     \item E \pause
     \item F \pause
     \item G \onslide{+1-+3}{(appears 1 overlay after G for 3 overlays)}\pause
     \item H \pause
     \item I \pause
     \item J \pause
     \item K
   \End{itemize}
 \end{slide}
\end{example}
As you can see, we still use |\onslide|. The only change is with the
syntax of the list of overlays. Now, we can specify a `|+|' symbol
in the list. In its simplest usage, |\onslide{+1}| will make text
display one overlay after the overlay it would \textit{normally}
appear on. You can still use the syntax in table~\ref{tab:item}.
These are demonstrated in the above example. Notice,
|\onslide{+1-6}| means that the text will appear one overlay after
the overlay it would normally appear on and that the text should
remain shown until overlay seven. To make text appear for a range of
relative overlays, see the final demonstration in the above example.

\section{Presentation structure}\label{sec:structure}
\subsection{Making sections}\label{sec:section}
\DescribeMacro{\section}
This section describes the |\section| command which provides a way
to structure a presentation.
\begin{command}
 `\cs{section}\oarg{options}\marg{section title}'
\end{command}
This command will produce a slide with \meta{section title} on it
and will also use this text to create sections in the table of
contents and in the bookmarks list. There are several \meta{options}
to control its output.

\DescribeOption{tocsection} This option controls the creation of a
section in the table of contents. The default value is |true|.
\begin{description}
\item\option{tocsection=true}\\
This does create a section in the table of contents. This means that
all following slides, until the next section, will be nested under
this section.
\item\option{tocsection=false}\\
This does not create a section in the table of contents and hence
the section will be listed as an ordinary slide.
\item\option{tocsection=hidden}\\
This does create a section in the table of contents, but this is
only visible when you view a slide that is part of this section.
This could be used to append a section to the presentation which you
can discuss if there is some extra time.
\end{description}

\DescribeOption{slide} This option controls whether the |\section|
command creates a slide. The default value is |true|.
\begin{description}
\item\option{slide=true}\\
A slide is created.
\item\option{slide=false}\\
No slide will be created. If also |tocsection| is |false|, the
|\section| command doesn't do anything. If it does create a table of
contents section (|tocsection=| |true| or |hidden|), its link will
point to the first slide in the section as the section itself
doesn't have a slide.
\end{description}

\DescribeOption{template} This option can be used to make the
section slide with another template. By default, a normal |slide|
environment is used to create the section slide, but if a style
offers other templates that could be used for this purpose (for
instance, the |wideslide| environment), then you can use this option
to select that template. See section~\ref{sec:styles} for an
overview of the available templates with every style.

Finally, all options available to normal slides are available to
slides created by |\section| as well (see section~\ref{sec:slides}).
However, when the section does make a |tocsection|, |toc=| or |bm=|
won't remove the table of contents entry or the bookmark
respectively.

\subsection{Making an overview}\label{sec:tableofcontents}
\DescribeMacro{\tableofcontents}
This command creates an overview of your presentation and can only
be used on a slide.
\begin{command}
 `\cs{tableofcontents}\oarg{options}'
\end{command}
There are several \meta{options} to control the output of this
command.

\DescribeOption{type}
This option controls whether certain material (depending on the
input in the |content| option below) will be hidden or displayed in
the inactive color\index{inactive color}. The default value is |0|.
Compare with the |type| option for list environments
(section~\ref{sec:lists}).

\begin{description}
\item\option{type=0}\\
When material is not of the requested type as specified in the
|content| option, it will be hidden.
\item\option{type=1}\\
As |type=0|, but instead of hiding material, it will be typeset in
the inactive color.
\end{description}

\DescribeOption{content}
The |content| option controls which elements will be included in the
overview. The default value is |all|. The description below assumes
that |type=0| was chosen, but the alternative text for |type=1| can
easily be deduced.

\begin{description}
\item\option{content=all}\\
This will display a full overview of your presentation including all
sections and slides, except the slides in hidden sections (see
section~\ref{sec:section}).
\item\option{content=sections}\\
This displays only the sections in the presentation.
\item\option{content=currentsection}\\
This displays the current section only.
\item\option{content=future}\\
This displays all content starting from the current slide.
\item\option{content=futuresections}\\
This displays all sections, starting from the current section.
\end{description}

We finish this section with a small example that will demonstrate
how you can make a presentation that contains an overall overview of
sections in the presentation, giving a general idea of the content,
and per section a detailed overview of the slides in that section.
\begin{example}
 \begin{slide}[toc=,bm=]{Overview}
   \tableofcontents[content=sections]
 \end{slide}
 \section{First section}
 \begin{slide}[toc=,bm=]{Overview of the first section}
   \tableofcontents[content=currentsection,type=1]
 \end{slide}
 \begin{slide}{Some slide}
 \end{slide}
 \section{Second section}
 ...
\end{example}

\section{Miscellaneous}
\subsection{The \cs{twocolumn} command}\label{sec:twocolumn}
\DescribeMacro{\twocolumn}
The |\twocolumn| macro allows to split content into two columns.
\begin{command}
 `\cs{twocolumn}\marg{options}\marg{left}\marg{right}'
\end{command}
This typesets \meta{left} and \meta{right} in two columns. The
dimensions of those columns can be controlled by \meta{options}.
Below are the available options.
\begin{description}
\item\DescribeOption{lineheight}\option{lineheight}\\
If |lineheight| is specified, a line of the specified height will be
created using |\psline| in between the two columns. Example:
|lineheight=6cm|.
\item\DescribeOption{lineprop}\option{lineprop}\\
Any \pf{pstricks} declaration to specify the line properties. Example:
\begin{example}
 lineprop={linestyle=dotted,linewidth=3pt}
\end{example}
\item\DescribeOption{lfrheight}\option{lfrheight}\\
Creates a frame of the specified height around the left
column.
\item\DescribeOption{lfrprop}\option{lfrprop}\\
As |lineprop|, but for the left frame.
\item\DescribeOption{rfrheight}\option{rfrheight}\\
Creates a frame of the specified height around the right
column.
\item\DescribeOption{rfrprop}\option{rfrprop}\\
As |lineprop|, but for the left frame.
\item\DescribeOption{frsep}\option{frsep}\\
Space between text and the frames. Default: |1.5mm|.
\item\DescribeOption{colsep}\option{colsep}\\
Space between the two columns. Default: |0.06\linewidth|.
\item\DescribeOption{lcolwidth}\option{lcolwidth}\\
Width of the left column. Default: |0.47\linewidth|.
\item\DescribeOption{rcolwidth}\option{rcolwidth}\\
Width of the right column. Default: |0.47\linewidth|.
\item\DescribeOption{topsep}\option{topsep}\\
The extra space (additional to |\baselineskip|) between text above
the columns and the text within the columns. Default: |0cm|.
\item\DescribeOption{bottomsep}\option{bottomsep}\\
Idem for the bottom of the columns. Default: |0cm|.
\item\DescribeOption{indent}\option{indent}\\
Horizontal indent left to the left column. Default: |0cm|.
\end{description}
The dimensions described above are represented graphically in
figure~\ref{fig:twocolumndim}.
\begin{figure}[htb]
\centering
\begin{pspicture}(0,.5)(13,10.5)
\psline(0,0.5)(0,10)
\rput[tl](.05,9.95){Top}
\psframe[dimen=middle](1,9)(7,2)
\psline{C-C}(8.5,9)(11,9)
\psline{C-C}(8.5,2)(8.5,9)
\psline{C-C}(8.5,2)(11,2)
\qdisk(1.7,8.3){.1cm}
\psset{linestyle=dashed}
\psline{C-C}(1.7,8.3)(6.3,8.3)
\psline{C-C}(1.7,8.3)(1.7,3)
\psline{C-C}(6.3,5)(6.3,8.3)
\psline{C-C}(11,9)(12,9)
\psline{C-C}(11,2)(12,2)
\psline{C-C}(11,7)(12,7)
\psline{C-C}(9.2,8.3)(12,8.3)
\psline{C-C}(9.2,8.3)(9.2,3)
\rput[tl](1.75,8.25){Left column text}
\rput[tl](9.25,8.25){Right column text}
\rput[tl](.05,1){Bottom}
\psset{linestyle=dotted,dotsep=2pt}
\psline(0,8.3)(1.7,8.3)
\psline(0,9.6)(1,9.6)
\psline(0,2)(1,2)
\psline(0,1.1)(1,1.1)
\psset{linestyle=solid}
\psline{<->}(.2,8.33)(.2,9.57)
\psline{<->}(4,8.33)(4,8.97)
\psline{<->}(1.73,7)(6.27,7)
\psline{<->}(1.03,6.5)(1.67,6.5)
\psline{<->}(0.03,5.5)(1.67,5.5)
\psline{<->}(6.33,7.4)(9.17,7.4)
\psline{<->}(8.53,6.5)(9.17,6.5)
\psline{<->}(6.33,6.5)(6.97,6.5)
\psline{<->}(10.7,8.33)(10.7,8.97)
\psline{<->}(7.3,8.97)(7.3,2.03)
\psline{<->}(.2,1.13)(.2,1.97)
\psline{->}(1.7,9.3)(1.7,8.45)
\psline{<-}(9.23,7)(11,7)
\cput(4,6.6){\small 1}
\cput(11.1,6.6){\small 2}
\cput(8,7){\small 3}
\cput(7.7,3){\small 4}
\cput(4.4,8.65){\small 5}
\cput(1.35,6.1){\small 5}
\cput(8.85,6.1){\small 5}
\cput(11.1,8.65){\small 5}
\cput(6.65,6.1){\small 5}
\cput(0.6,8.95){\small 6}
\cput(0.6,5.1){\small 7}
\cput(0.6,1.55){\small 8}
\cput(1.7,9.6){\small 9}
\end{pspicture}
\begin{tabular}{c p{4cm}cl}
\multicolumn{4}{c}{Meaning of the labels}\\\hline
1&|lcolwidth|&5&|frsep|\\
2&|rcolwidth|&6&|topsep|\\
3&|colsep|&7&|indent|\\
4&|lfrheight|, |rfrheight|,&8&|bottomsep|\\
&|lineheight|&9&Reference point
\end{tabular}
\caption{Two-column dimensions.}\label{fig:twocolumndim}
\end{figure}
Important to notice is that the |\twocolumn| macro uses the current
cursor position as the reference point to position the first line of
text of the left column (see also figure~\ref{fig:twocolumndim}). This
means that optional frames can extend to the text on the previous
line. Use for instance |topsep=0.3cm| in that case to add extra
space between the two lines of text. The default value of |topsep|
is based on the situation that there is no text on top of the two
columns. In that case, it is best to locate the first line of text
of the left column at the same spot as text that is not created by
|\twocolumn| on other slides. The setting |topsep=0cm| does exactly
this. However, with a combination of |topsep| and |indent| you can
change this behavior and position the first line of text of the left
column anywhere you want.

The |\twocolumn| macro computes the height of the construction to
position text below the construction correctly. The computation is
done by taking the maximum height of |lfrheight|, |rfrheight|,
|lineheight| (if requested) and the left and right column content.
Hence when frames nor a line is requested, |bottomsep| is the
vertical space between the lowest line of text in the columns and
the text below the columns (additional to |\baselineskip|). Here is
an example.
\begin{example}
 \begin{slide}{Two columns}
   Here are two columns.
   \twocolumn{lfrprop={linestyle=dotted,linewidth=3pt},
     lfrheight=4cm,rfrheight=5cm,lineheight=3cm,topsep=0.3cm
   }{left}{right}
   That were two columns.
 \end{slide}
\end{example}

\subsection{Notes}\label{sec:notes}
\DescribeEnv{note}
The |note| environment can be used to make personal notes that
accompany a slide. You can control displaying notes using the
|display| option (see section~\ref{sec:classopts}). Here is an
example.
\begin{example}
 \begin{slide}{Chewing gum}
 ...
 \end{slide}
 \begin{note}{Reminder for chewing gum}
   Don't forget to mention that chewing gum is sticky.
 \end{note}
\end{example}

\subsection{Empty slides}\label{sec:emptyslides}
\DescribeEnv{emptyslide}
The |emptyslide| environment creates a totally empty slide. The text
box on the slide can be used for special things like displaying
photos. This allows for creating a dia show. Example:
\begin{example}
 \begin{emptyslide}{}
   \centering
   \vspace{\stretch{1}}
   \includegraphics[height=0.8\slideheight]{me_chewing_gum.eps}
   \vspace{\stretch{1}}
 \end{emptyslide}
\end{example}
The |\includegraphics| command is defined by the \pf{graphicx}
package \cite{graphics}. The |\stretch| command is used to
vertically center the picture. Both commands are described in your
favorite \LaTeX\ manual, for instance \cite{companion}. Note that
you can use the lengths |\slideheight| and |\slidewidth| to scale
pictures to fit nicely on the slide.

\subsection{Bibliography slide}\label{sec:bib}
\DescribeEnv{thebibliography}
\pf{powerdot} redefines the standard \pf{article}
|thebibliography| environment to suppress the creation of a section
heading and running headers. All other properties are maintained.
You can do either of the next two (depending whether you are
using BiB\TeX\ or not):\\
\begin{minipage}[t]{.49\linewidth}
\begin{example}
 \begin{slide}{Slide}
   \cite{someone}
  \end{slide}
 \begin{slide}{References}
   \begin{thebibliography}{1}
   \bibitem{someone} Article of someone.
   \end{thebibliography}
 \end{slide}
\end{example}
\end{minipage}\hfill
\begin{minipage}[t]{.49\linewidth}
\begin{example}
 \begin{slide}{Slide}
   \cite{someone}
 \end{slide}
 \begin{slide}{References}
   \bibliographystyle{plain}
   \bibliography{YourBib}
 \end{slide}
\end{example}
\end{minipage}

In case you have a big reference list that you want to spread over
multiple slides, have a look at the packages \pf{natbib} and
\pf{bibentry} \cite{natbib}. Using both packages allows you to do:
\begin{example}
 \begin{slide}{References (1)}
   \bibliographystyle{plain}
   \nobibliography{YourBib}
   \bibentry{someone1}
   \bibentry{someone2}
 \end{slide}
 \begin{slide}{References (2)}
   \bibentry{someone3}
 \end{slide}
\end{example}
Have a look at your favorite \LaTeX\ manual for more information
about citations and bibliographies.

\section{Available styles}\label{sec:styles}
\pf{powerdot} comes with a number of styles which are listed in the
overview below. The characteristics of each style are described
shortly.
\begin{description}
\item\pf{default}\\
This style has as main colors light blue and white. A flower in the
top left corner decorates the slide. The style supports a
|wideslide| and portrait orientation. Slides have a table of
contents on the left part of the paper in landscape orientation and
on the bottom part in portrait orientation. The style requires the
\pf{pifont} package.
\item\pf{simple}\\
This is a simple style in black and white. The style supports a
|wideslide| and portrait orientation. A table of contents is present
on slides at the left hand side in landscape mode and in the bottom
of the slide in portrait mode. It requires the \pf{amssymb} and
\pf{pifont} packages.
\item\pf{tycja}\\
This style is set in shades of yellow and dark blue. The style
supports a |wideslide| and portrait orientation. The table of
contents on slides is on the right side of the paper in landscape
orientation and on the bottom part in portrait. The style requires
the \pf{pst-grad} \cite{PSTricksWeb,PSTricks} and \pf{pifont}
packages.
\item\pf{ikeda}\\
This style uses dark shades of red and blue and a light text color.
It has nice patterns on the slide for decoration. The style supports
a |wideslide| and portrait orientation. The table of contents is on
the left side of the paper in landscape orientation and on the
bottom part in portrait. The style requires the \pf{calc} and
\pf{pifont} packages.
\item\pf{fyma}\\
This style was originally created by Laurent Jacques for
\pf{prosper}. Based on that style, he created a version for
\pf{HA-prosper} with extended features. With his kind permission,
this style has been converted by Shun'ichi J. Amano for
\pf{powerdot}. The style has an elegant design with a light blue and
white gradient background. The style supports a |wideslide| and
portrait orientation. It has special templates for sections on
slides and sections on wide slides. The table of contents is on the
left side of the paper in landscape orientation and on the bottom
part in portrait. The style requires the \pf{pst-grad}
\cite{PSTricksWeb,PSTricks} package.
\item\pf{ciment}\\
This style was originally created by Mathieu Goutelle for
\pf{prosper} and \pf{HA-prosper}. With his permission, this style
has been converted for \pf{powerdot}. The style has a background
that is hatched with light gray horizontal lines. Titles and table
of contents highlighting are done with dark red. The table of
contents is on the left side of the paper in landscape orientation
and on the bottom part in portrait. The style requires the
\pf{pifont} package.
\end{description}

\section{Compiling your presentation}\label{sec:compiling}
\subsection{Dependencies}\label{sec:dependencies}
In table~\ref{tab:dependencies} is a list of packages that
\pf{powerdot} uses to perform specific tasks. Dependencies of
packages in this table are not listed. Notice further that styles
may have extra requirements (see section~\ref{sec:styles}). In the
table, `required' means that you should have a version \emph{at
least} as new as listed and `tested' means that \pf{powerdot} was
tested with this version, but that it could equally well work with
an older or newer version than the one listed in the table. So, when
trying to solve an error, first concentrate on solving version
issues for the `required' packages. To find out which version of a
package you are currently using, put |\listfiles| on the first line
of your document, run it with \LaTeX, open the |.log| file and read
the file list (see a \LaTeX\ manual for more information). If you
need to update a package, you can get it from CTAN \cite{CTAN}.
\begin{table}[htb]
\centering
\begin{tabular}{e}
Package/file & Version & Date & Required/tested\\\hline
\pf{xkeyval} \cite{xkeyval} & 2.5c & 2005/07/10 & required\\
\texttt{pstricks.sty} \cite{PSTricksWeb,PSTricks} & 0.2l & 2004/05/12 & required\\
\pf{xcolor} \cite{xcolor} & 1.11 & 2004/05/09 & required\\
\pf{enumitem} \cite{enumitem} & 1.1 & 2005/05/12 & required\\
\pf{article} class & 1.4f & 2004/02/16 & tested\\
\pf{geometry} \cite{geometry} & 3.2 & 2002/07/08 & tested\\
\pf{hyperref} \cite{hyperref} & 6.74m & 2003/11/30 & tested\\
\pf{graphicx} \cite{graphics} & 1.0f & 1999/02/16 & tested\\
\pf{float} \cite{float} & 1.3d & 2001/11/08 & tested
\end{tabular}
\caption{Dependencies}\label{tab:dependencies}
\end{table}

\subsection{Creating and viewing output}\label{sec:creation}
To compile your presentation, run it with \LaTeX. The DVI that is
produced this way can be viewed with MiK\TeX's DVI viewer
YAP.\footnote{Unless you are using \pf{pstricks-add} which distorts
the coordinate system in DVI.} Unfortunately, xdvi and kdvi (kile)
do not support all PostScript specials and hence these will display
the presentation incorrectly. If your DVI viewer does support
this, make sure that your DVI display settings match that of the
presentation. In case you are using the |screen| paper, you should
set the DVI display setting to using the letter paper format. If
your DVI viewer allows for custom paper formats, use 8.25 inch by 11
inch.

Note that certain things that are produced with PostScript or PDF
techniques will not work in a DVI viewer. Examples are hiding of
material via postscript layers (as is done, for instance, by
|\pause|, see section~\ref{sec:overlays}) and hyperlinks, for
instance in the table of contents.

If you want to produce a postscript document, run dvips over the DVI
\emph{without any particular command line options related to
orientation or paper size}. \pf{powerdot} will write information to
the DVI file that helps dvips and ps2pdf (ghostscript) to create a
proper document. If you have some reason that this does not work for
you and you want to specify the paper and orientation yourself, you
should use the |nopsheaders| option that is described in
section~\ref{sec:setup}. The PostScript document could, for
instance, be used to put multiple slides on a page using the |psnup|
utility.

To create a PDF document for your presentation, run ps2pdf over the
PS file created with dvips. Also here, you can \emph{leave out any
command line arguments related to paper size or orientation}. If
this is problematic for you somehow, use the |nopsheaders| option as
before and specify the paper and orientation at each intermediate
step yourself.

\section{Creating your own style}\label{sec:writestyle}
\subsection{General information}
Writing or customizing styles is simple. If you want to modify a
style or build a new one, locate the style that you want to use as
basis in your \TeX\ tree (styles are named as
|powerdot-<style_name>.sty|), copy that and rename it as to avoid
license\footnote{The \LaTeX\ Public Project License requires
renaming files when modifying them, see
\url{http://www.latex-project.org/lppl}.} or naming conflicts. You
might want to install the new style in your local \TeX\ tree to be
able to access it from any place on your hard drive. See your
\LaTeX\ distribution for more information.

Once that has been taken care of, we can start creating the style.
We strongly recommend to study a style file (for example,
|powerdot-default.sty|) while reading the remainder of this section
as it provides good examples for the content of this section.

A style has several components. We describe these components below.
\begin{description}
\item\textbf{Identification}\\
This identifies the package in the log of a presentation. The default
style contains:
\begin{example}
 \NeedsTeXFormat{LaTeX2e}[1995/12/01]
 \ProvidesPackage{powerdot-default}[2005/09/04 v1.0 default style (HA)]
\end{example}
See for more information about these commands a \LaTeX\ manual, for
instance \cite{companion}.
\item\textbf{Color definitions}\\
This section contains the definitions of the colors that you want to
use in the style. \pf{powerdot} uses \pf{xcolor} (via
\pf{pstricks}). Hence, for more information about colors, see the
\pf{xcolor} documentation.
\item\textbf{Template definitions}\\
We will come back to this in the subsequent sections.
\item\textbf{Custom declarations}\\
These can include anything that you want to be part of the style.
The default style, for instance, includes definitions for the labels
in list environments like |itemize| and some initializations for
lists in general (done with |\pdsetup|, see
section~\ref{sec:pdsetup}).
\item\textbf{Color and font initializations}\\
This is a very important part. Here you \emph{must} initialize the
main text color and put font definitions (which can be done by
loading a package like \pf{helvet}). If you don't initialize the
color here, most of the text will appear in black.
\end{description}

\subsection{Defining templates}
We start off with a definition of what a template is. A template is
a collection of settings for slide components together with custom
definitions, which controls the visual appearance of a slide. A
style can contain multiple templates.

\begin{command}
 `\cs{pddefinetemplate}\oarg{basis}\marg{name}\marg{options}\marg{commands}'
\end{command}
\DescribeMacro{\pddefinetemplate}
This defines the environment \meta{name} to produce a slide with
characteristics determined by \meta{basis}, \meta{options} and
\meta{commands}. We will discuss these elements in more detail
in the coming sections.

If you want to create several templates that differ only slightly
from each other, define a \meta{basis} template, and then use it to
define other templates. All \meta{options} and \meta{commands} for
the new template \meta{name} will be appended to the existing list
of \meta{options} and \meta{commands} from the \meta{basis}
template.

Make sure you choose a \emph{proper} name for the template, and
avoid redefining existing templates or environments. \pf{powerdot}
defines |blackslide|, |note| and |emptyslide| internally, so you
shouldn't use these names unless you know what you're doing.
Furthermore, each style needs to define at least the templates
|slide| and |titleslide|. The |titleslide| environment will be used
to create the title slide and |slide| will (by default) also be used
to create section slides. Titles and sections are a bit special in
the way they use the \meta{options} and will be discussed in more
detail in section~\ref{sec:specialtemps}.

\subsection{Controlling setup}
\DescribeOption{ifsetup}
The \meta{options} (keys) are described in the following sections.
You can control how these options apply to the various setups by
using the |ifsetup| key. Any key appearing before the first
|ifsetup| declaration in \meta{options} will apply to every possible
setup. Once the |ifsetup| key is used, then all subsequent key
declarations will apply \textit{only} to the setups declared in the
|ifsetup| key. The |ifsetup| key can be used multiple times.

By possible setups, we mean the allowed values of the |mode|,
|paper|, |orient|, and |display| keys that are described in
section~\ref{sec:classopts}. If a value (or values!) for any of
these four keys is not specified in a |ifsetup| declaration, then
all subsequent key declarations will apply to any layout of that
type. Consider the following as an example.
\begin{example}[numbers=left,numberstyle=\tiny\ttfamily,%
  escapeinside=`',numbersep=1em,xleftmargin=1em]
 ...
 textpos={.2\slidewidth,.3\slideheight},%
 ifsetup={portrait,screen},%
 textpos={.3\slidewidth,.2\slideheight}%
 ...
 ifsetup=landscape,%
 ...
 ifsetup,
 ...
\end{example}
Assuming there was no |ifsetup| declaration before the first
|textpos| declaration, this first |textpos| will apply to every
possible setup. However, for the screen format in portrait
orientation, the next |textpos| declaration will be used. In fact,
all declarations that appear until we switch to the next |ifsetup|
(which specifies all paper sorts and only landscape orientation)
will be used in the portrait screen layout. All keys after the next
|ifsetup| declaration will be used in landscape orientation,
\emph{with any paper, mode and display}. If, after declaring some
specializations, you want to switch back to settings that apply to
all possible setups, set |ifsetup| to empty as is done in the
example. All subsequent declarations will then again be applied
under any setup.

The following command is a stand-alone implementation of the
mechanism described above. It allows you to control the setup
outside the \meta{options} argument of the |\pddefinetemplate|
command.
\begin{command}
 `\cs{pdifsetup}\marg{desired}\marg{yes}\marg{no}'
\end{command}
\DescribeMacro{\pdifsetup}
This macro executes \meta{yes} when the setup that the user chose
matches with the \meta{desired} setup, \meta{no} in all other cases.
For instance, if the user has chosen landscape, then
\begin{example}
 \pdifsetup{landscape}{yes}{no}
\end{example}
will typeset |yes|. If the user would have chosen portrait instead,
then |no| would have been typeset.

This macro can be used to check setup requests from the user and,
for instance, generate an error if a certain setup is not supported
by your style. \pf{powerdot} provides one predefined error message
which can be used in one of the first lines of your style.
\begin{command}
 `\cs{pd@noportrait}'
\end{command}
\DescribeMacro{\pd@noportrait}
This macro generates an error when the user requests portrait
orientation. Notice that the handout mode only works in portrait
orientation. This macro takes that into account and doesn't generate
an error in that case.

\subsection{Main components}
The \meta{options} control several key components of a slide. Every
component has several properties. A key that can be used in the
\meta{options} argument is the name of the component postfixed by
its property that you want to control.

The components |title|, |text|, |toc|, |stoc| and |ntoc| have
properties |hook|, |pos|, |width| and |font|. Additionally, the
|text| component has a |height| property. The components |lf| and
|rf| have properties |hook|, |pos|, |temp| and |font|. Hence,
examples of valid keys are |titlefont|, |tocpos| and |lftemp|. All
components and properties will be discussed below.

Here is an overview of the components that can be controlled from
the \meta{options} argument in |\pddefinetemplate|.
\begin{description}
\item\DescribeOption{title-}\option{title-}\\
The slide title.
\item\DescribeOption{text-}\option{text-}\\
The main text box on the slide.
\item\DescribeOption{toc-}\option{toc-}\\
The (full) table of contents on a slide containing sections and
slides.
\item\DescribeOption{stoc-}\option{stoc-}\\
This is a table of contents containing only the sections. See also
|ntoc| below.
\item\DescribeOption{ntoc-}\option{ntoc-}\\
This is a table of contents containing only the entries for the
active section. Together with |stoc|, this can be used to create a
split table of contents. In a particular template, one would usually
have a |toc|, a combination of |stoc| and |ntoc| or no table of
contents at all.
\item\DescribeOption{lf-}\option{lf-}\\
The left footer.
\item\DescribeOption{rf-}\option{rf-}\\
The right footer.
\end{description}

Notice that all positioning of components described above will be
done with |\rput| from \pf{pstricks} \cite{PSTricksWeb,PSTricks}
internally. See the \pf{pstricks} documentation for more information
about this command. It should also be noted that all components
(except |lf| and |rf|) put their content in a |minipage|
environment.

Now we list all properties of the components listed above and
describe what they mean. Remember that keys are formed by combining
a component name and a property.

\begin{description}
\item\DescribeOption{-hook}\option{-hook}\\
This option defines the |\rput| hook that will be used when
positioning the item. This can be |tl|, |t|, |tr|, |r|, |Br|, |br|,
|b|, |bl|, |Bl|, |l|, |B| and |c|. See the \pf{pstricks}
documentation for more information.
\item\DescribeOption{-pos}\option{-pos}\\
This defines the position of the |hook| on the paper. The lower left
corner of the paper is given by the point |{0,0}| and the upper right
corner by the point |{\slidewidth,\slideheight}|. So if you want to
position the main text box at 20\% from the left edge and 30\% from
the top edge of the paper, you have to do the following.
\begin{example}
 textpos={.2\slidewidth,.7\slideheight}
\end{example}
If the position of any component has not been specified, this
component will not be placed on the slide. This gives an opportunity
to design slides without footers or table of contents, for instance.
\item\DescribeOption{-width}\option{-width}\\
The width of the component. All component positioned by
\pf{powerdot} will be put in a surrounding |minipage| environment.
The |width| property determines the width of the |minipage|. Example:
\begin{example}
 textwidth=.7\slidewidth
\end{example}
This property does not exist for the |lf| and |rf| components.
\item\DescribeOption{-height}\option{-height}\\
This option is only available for the |text| component. In other
words, for this property, there is only one key, namely
|textheight|. This can be used to specify the height of the
|minipage| used for the main text. This does not imply that users
are restricted to this length or that \pf{powerdot} does automatic
slide breaking. This height is only used for vertical alignments of
material, for instance by footnotes. The default value is
|\slideheight|.
\item\DescribeOption{-font}\option{-font}\\
This will be inserted just before the text that is about to be
typeset. This can be used to declare deviations from the main text
font and color. It can be a font declaration, like
|\large\bfseries|, but can also contain other things like
|\color{red}| or |\raggedright|.
\item\DescribeOption{-temp}\option{-temp}\\
This property is only available for the footers (|lf| and |rf|) and
can be used to change the template of the footers. This means that
you can, for instance, add content to the footer, besides the
content specified by the user. The default declaration by
\pf{powerdot} is the following.
\begin{example}
 rftemp=\pd@@rf\ifx\pd@@rf\@empty
   \else\ifx\theslide\@empty\else~--~\fi\fi\theslide
\end{example}
Here |\pd@@rf| will contain the content of the right footer defined
by the user via the |\pdsetup| command. Similarly, |\pd@@lf|
contains the content of the left footer. The above declaration
checks whether the footer and |\theslide| are both non-empty and if
so, it inserts |~--~| to separate both.
\end{description}

Below, we have copied the default setting for the keys described
above done by \pf{powerdot}. These will be used if you didn't supply
other input for these keys in a particular template. If the default
value meets your needs, you don't have to specify it again in your
style.
\begin{example}
 titlehook=Bl,titlepos,titlewidth=\slidewidth,
 titlefont=\raggedright,texthook=lt,textpos,
 textwidth=\slidewidth,textfont=\raggedright,
 textheight=\slideheight,
 tochook=lt,tocpos,tocwidth=.2\slidewidth,
 tocfont=\tiny\raggedright,
 stochook=lt,stocpos,stocwidth=.2\slidewidth,
 stocfont=\tiny\raggedright,
 ntochook=lt,ntocpos,ntocwidth=.2\slidewidth,
 ntocfont=\tiny\raggedright,
 tocfrsep=.5mm,tocsecsep=2ex,tocitemsep=0ex,
 tocsecm,toctcolor=black,tochlcolor=black,tochltcolor=white,
 lfhook=lB,lfpos,lffont=\scriptsize,lftemp=\pd@@lf,
 rfhook=rB,rfpos,rffont=\scriptsize,rftemp=\pd@@rf\ifx\pd@@rf\@empty
   \else\ifx\theslide\@empty\else~--~\fi\fi\theslide,
 iacolor=lightgray,sectionskip=0pt,sectionfont=\centering
\end{example}

\subsection{Slide toc}
The small table of contents that is placed on slides can be
controlled by four macros and several options.

\DescribeMacro{\pd@tocentry}
\DescribeMacro{\pd@tocsection}
These macros take one argument. When building the table of contents,
\pf{powerdot} first passes the content through |\pd@tocentry| or
|\pd@tocsection|, depending on the type of entry that it is building
at that moment. You could, for instance, do
\begin{example}
 \def\pd@tocentry#1{$\bullet$\ #1}
 \def\pd@tocsection#1{#1}
\end{example}
which will prefix all normal entries (not the sections) with a
bullet. By default, these two macros are defined to just pass on
their argument.

\DescribeMacro{\pd@tocdisplay}
\DescribeMacro{\pd@tochighlight}
These two macros also take one argument. After processing an entry
with the command |\pd@tocentry| or |\pd@tocsection|, \pf{powerdot}
continues building the entry by passing it through |\pd@tocdisplay|,
when the entry needs to be displayed only, or |\pd@tochighlight|,
when the entry needs to be highlighted. These macros are a little
more involved and take care of putting the content in the proper
font and color in a |minipage|. Further, |\pd@tochighlight| also
puts a box around the item.

Several aspects of this process can be controlled via the keys that
are available in the |\pddefinetemplate| command that will be
described in a moment. If these keys do not provide enough handles
to do what you want, you might need to have a look at these two
macros in the source and decide to rewrite them in your style as to
fit your needs.

\begin{description}
\item\DescribeOption{tocfrsep}\option{tocfrsep}\\
This length is the distance between the box around the content
created by the |minipage| and the highlight frame box created by
|\pd@tochighlight|. Default: |0.5mm|.
\item\DescribeOption{tocsecsep}\option{tocsecsep}\\
The vertical distance inserted above a section (unless it is the
first element in the table of contents). Default: |2ex|
\item\DescribeOption{tocitemsep}\option{tocitemsep}\\
The vertical distance inserted above other entries (unless it is the
first element in the table of contents). Default: |0ex|.
\item\DescribeOption{tocsecm}\option{tocsecm}\\
This is inserted just before typesetting a section. This can be used
to mark a section, for instance with a line as in the \pf{default}
style. Default: empty.
\item\DescribeOption{toctcolor}\option{toctcolor}\\
This is the text color used for non-highlighted elements in the
table of contents. Default: |black|.
\item\DescribeOption{tochltcolor}\option{tochltcolor}\\
This is the text color used for highlighted elements in the table of
contents. Default: |white|.
\item\DescribeOption{tochlcolor}\option{tochlcolor}\\
This is the color used for the frame behind highlighted elements.
Default: |black|.
\end{description}

\subsection{Miscellaneous options}\label{sec:miscoptions}
There are some options the fall outside of the scope of the previous
section. These will be discussed here.

\DescribeOption{iacolor}
The |iacolor| option can be used to specify the color that is used
for inactive things, produced for instance by |\onslide|, |\pause|
(see section~\ref{sec:overlays}) and |\tableofcontents| (see
section~\ref{sec:tableofcontents}). As \pf{xcolor} is used by
\pf{powerdot}, one can use special notation here, like
\begin{example}
 iacolor=black!20
\end{example}
The default value for this key is |lightgray|.

\DescribeOptions{sectionskip,sectionfont}
These options can be specified in any template and will be used when
the user chooses to use this specific template for creating a
section with |\section| (see section~\ref{sec:section}). The section
title will be put in the main text box at a vertical distance
specified by |sectionskip| from the top and in |sectionfont|. The
default values are |sectionskip=0pt| and |sectionfont=\centering|.
Specify values for these keys for every template that can be used
for creating sections. More details about this are in
section~\ref{sec:specialtemps}.

\subsection{The background}
This leaves only one argument of the |\pddefinetemplate| macro
undiscussed. This is the \meta{commands} argument. This argument can
contain any code that you want to execute \textit{after} setting the
options and \textit{before} building the slide components like the
slide title, main text, and footers. This argument is designed to
contain declarations that will build the background of a template
using, for instance, \pf{pstricks}, but it can also hold other
commands you might need for building your template.

\subsection{Title slide, titles and sections}\label{sec:specialtemps}
As mentioned before, the style that you write needs to define at
least the templates |slide| and |titleslide|. The latter treats some
of the keys in a special way. Besides, a section slide is also done
in a special way.

The title slide (made with |\maketitle|) puts the title with
author(s) and date in the main text box. This means that you have to
supply a position for the main text box (|textpos|). It will use the
main text font for the text (together with declarations in the
|textfont| key) for the author(s) and the date. But it will use the
declarations in |titlefont| for the title of the presentation. This
is done so that title and author(s) form a coherent block and to
make sure that long titles can push down the author(s) instead of
overwriting it.

\DescribeMacro{\pd@slidetitle}
The |\pd@slidetitle| macro is used to typeset the slide title on
slides. This macro is comparable to for instance |\pd@tocentry|. The
macro takes one argument which is the slide title in the right font
and formatting. By default, this macro just passes on the content
for typesetting, but you could redefine this macro so do something
with its input prior to typesetting it. An example is in the
\pf{fyma} style which underlines the title using the following code.
\begin{example}
 \def\pd@slidetitle#1{%
   \settowidth\@tempdima{#1}%
   \psline[linewidth=.8pt,linecolor=pddblue]%
     (0,-.15cm)(\@tempdima,-.15cm)#1%
 }
\end{example}
\DescribeMacros{\pd@title,\pd@sectiontitle}
These macros are similar to |\pd@slidetitle| and typeset the title
on the title slide and the title on section slides respectively. By
default, these also pass there argument (which is the presentation
title or section title), but these can be redefined to do something
with the input prior to typesetting it, just as |\pd@slidetitle|.

\DescribeOptions{sectemp,widesectemp}
The |\section| command uses (by default) the |slide| environment and
puts the section title in the main text box with font |sectionfont|
and vertical distance from the top of the main text box
|sectionskip| (see section~\ref{sec:miscoptions}). You should
specify values for these keys for all templates that the user can
use to create a section. If you want to change the default use of
the |slide| environment for sections to, for instance, the
|sectionslide| environment or any other especially designed section
template, change the section template preset in your style, using
\begin{example}
 \setkeys[pd]{section}{sectemp=sectionslide}
\end{example}
This means that if the user asks for |template=slide| in the
|\section| command, the |sectionslide| environment will be used
silently. A similar option is available in case the user asks for
|template=wideslide|. One could for instance do the following.
\begin{example}
 \setkeys[pd]{section}{widesectemp=sectionwideslide}
\end{example}
Whenever the user requests a |wideslide| to be used for a
|\section|, instead, the |sectionwideslide| environment will be
used. Other input to the |template| key by the user does not get a
special treatment.

Notice that these keys are available in the |section| family of keys
and that you cannot use them in the |\pddefinetemplate| command.

\subsection{Testing the style}\label{sec:styletest}
\pf{powerdot} has a test file that should test most of the style.
This test file can be produced by running \LaTeX\ over
|powerdot.dtx|. This generates |powerdot-styletest.tex| which will
help you with the testing job. Feel free to contact us when you
would like to contribute your style to \pf{powerdot}. See also
section~\ref{sec:questions}.

\section{Questions}\label{sec:questions}
\subsection{Frequently Asked Questions}\label{sec:FAQ}
This section is devoted to Frequently Asked Questions. Please read
it carefully; your problem might be solved by this section.
\begin{itemize}[leftmargin=0pt]
\question
Does \pf{powerdot} have example files? Where can I find them?
\answer
\pf{powerdot} comes with several examples that should be in the doc
tree of your \LaTeX\ installation. More precisely:
\url{/doc/latex/powerdot}. If you can't find them there, download
them from \url{CTAN:/macros/latex/contrib/powerdot} \cite{CTAN}.
\question I'm getting errors on the simplest example!
\answer Did you read section~\ref{sec:dependencies}?
\question I made a typo in the slide code, ran the file, got an
error, corrected the typo and reran, but now get an error that
doesn't go away.
\answer Remove the |.bm| and |.toc| files and try again.
\question
My figure/table produces the error: \texttt{Not in outer par mode}.
\answer
You have asked \LaTeX\ to float the figure or table using something like
\begin{example}
 \begin{figure}[htb]
\end{example}
\LaTeX\ has nowhere to float the figure or table to. Remove the
optional argument (here |[htb]|) and the figure or table will work.
\question
Can I contribute to this project?
\answer
Certainly. If you find bugs\footnote{Make sure that you confirm that
the bug is really caused by \pf{powerdot} and not by another package
that you use.} or typos, please send a message to the mailinglist
(see section~\ref{sec:mailinglist}), If you have developed your own
style and would like to see it included in \pf{powerdot}, please
inform us by private e-mail. Notice that contributions will fall
under the overall \pf{powerdot} license and copyright notice, but
that your name will be included in the documentation when you make a
contribution.
\end{itemize}

If your question has not been answered at this point, advance to the
next section where to find more answers.

\subsection{Mailinglist}\label{sec:mailinglist}
\pf{powerdot} has a mailinglist from \url{freelists.org} and has its
website here:
\begin{center}
\url{http://www.freelists.org/list/powerdot}
\end{center}
There is a link to `List Archive'. Please search this archive before
posting a question. Your problem might already have been solved in
the past.

If that is not the case, use the box on the page to type your e-mail
address, choose the action `Subscribe' and click `Go!'. Then follow
the instructions that arrive to you by e-mail. At a certain moment,
you can login for the first time using an authorization code sent to
you by e-mail. After logging in, you can create a password for
future sessions using the `Main Menu' button. The other buttons
provide you some info and options for your account.

When you are all set, you can write to the list by sending an e-mail
to
\begin{center}
\url{powerdot@freelists.org}
\end{center}

When writing to the list, please keep in mind the following very
important issues.
\begin{enumerate}[leftmargin=0pt]
\item We are volunteers!
\item Always supply a \emph{minimal} example demonstrating your
problem.
\item Don't send big files over the list.
\end{enumerate}

We hope you will enjoy this service.

\section{Source code documentation}\label{sec:sande}
In case you want regenerate the package files from the source or
want to have a look at the source code description, locate
|powerdot.dtx|, search in the file for |\OnlyDescription| and remove
that and do
\begin{example}
 latex powerdot.dtx
 latex powerdot.dtx
 bibtex powerdot
 makeindex -s gglo.ist -o powerdot.gls powerdot.glo
 makeindex -s gind.ist -o powerdot.ind powerdot.idx
 latex powerdot.dtx
 latex powerdot.dtx
\end{example}

\end{document}
