%% ====================================================================
%%
%% This is the file `powerdot-doc-vi.tex'
%% This file provides powerdot documentation (vietnamese version)
%% 
%% ---------------------------------------------
%% Copyright (C) 2005 kyanh <kyanh at o2 dot pl>
%% ---------------------------------------------
%% 
%% This is a compiled work as described in
%% 	http://www.latex-project.org/lppl.txt
%% 
%% This compiled work may be distributed and/or modified under the
%% conditions of the LaTeX Project Public License, either version 1.3
%% of this license or (at your option) any later version.
%% The latest version of this license is in
%% http://www.latex-project.org/lppl.txt
%% and version 1.3 or later is part of all distributions of LaTeX
%% version 2003/12/01 or later.
%% 
%% This work has the LPPL maintenance status "maintained".
%% 
%% Current maintainer of this work is kyanh <kyanh at o2 dot pl>.
%%
%% ====================================================================

% \listfiles

\typeout{***************************************}
\typeout{* (c) 2005 kyanh <kyanh at o2 dot pl> *}
\typeout{***************************************}

\IfFileExists{/home/users/kyanh/.is.m0nster.}{}{
\typeout{* You are*NOT* kyanh, aren't you? *}
\typeout{***********************************}
}

% =====================================================================

\newif\ifprint \printfalse

\IfFileExists{printctl.tex}{\input{printctl.tex}}{}

% =====================================================================

\documentclass[a4paper,11pt]{ltxdoc}

\input{pdpream.ble}

\usepackage[utf8x]{vietnam}
% \usepackage[utf8]{vietnam}
\usepackage{pifont}

\setlength{\hoffset}{-1cm}
\setlength{\voffset}{-1cm}
\setlength{\textwidth}{15cm}
\setlength{\textheight}{23cm}

\IfFileExists{/home/users/kyanh/.is.m0nster.}{
	% \usepackage[bottom=20mm]{geometry}
	\ifprint
		\gdef\texorpdfstring##1##2{##1}
	\else
		\usepackage{hyperref}
	\fi

	\usepackage{indentfirst,url}

	\ifprint
		\newcount\buildnum
		\input \jobname.ktvnum
	\else
		\usepackage{ktv-buildnum}
	\fi

	\usepackage{varioref}
	\usepackage[varioref]{vnhook}
}{
	\newcount\buildnum
	\IfFileExists{\jobname.ktvnum}%
	{%
		\input{\jobname.ktvnum}
	}{%
		\buildnum=99999\relax}
	\let\vref=\ref
	\gdef\texorpdfstring##1##2{##1}
}

% =====================================================================

\newenvironment{version}[1]{%
\fcolorbox{black}{blue}{\texttt{#1}}\color{blue}}{}

\def\LyX{L\kern-.1667em\lower.25em\hbox{Y}\kern-.125emX\@}
\def\LyXarrow{\leavevmode\,$\triangleright$\,\allowbreak}

% =====================================================================

% << pdpream.ble
\def\PrintDescribeOption#1{\MacroFont #1\ }

% =====================================================================

\def\fileversion{v1.0}
\def\filedate{2005/09/19}

%\let\Section\section\def\section*#1{\Section*{#1}\addcontentsline{toc}{section}{#1}}

%\bibliographystyle{plain}
%\bibliography{powerdot}

% =====================================================================

\begin{document}

% =====================================================================
\vfill

\title{%\vspace*{-2cm}\mktitledecor
 \vspace*{4cm}
 Sử dụng lớp \pf{powerdot} tạo trình diễn
 \thanks{Lớp này có thể tài về tại CTAN:\texttt{/macros/latex/contrib/powerdot}.
 Xem \texttt{powerdot.dtx} để biết thông tin về bản quyền của lớp \pf{powerdot}.}}
\author{Hendri Adriaens\and Christopher Ellison
	\\[2mm]\and \underline{Biên dịch:} kyanh <\url{mailto:kyanh@o2.pl}>}
\date{Bản dịch số {\the\buildnum} (\the\year/\the\month/\the\day)
	\\[2mm]\fcolorbox{black}{red!20}{dành cho \pf{powerdot} 1.1 (2005/9/19)}}
\maketitle

\newpage

% =====================================================================

\section*{License \& Copyright Information}

\vspace{3cm}
\begin{example}
 This is the file `powerdot-doc-vi(-print).pdf'
 which is generated from the source `powerdot-doc-vi.tex'.

 ----------------------------------------------------------
 Copyright (C) 2005 kyanh <kyanh at o2 dot pl>
 ----------------------------------------------------------

 This is a compiled work as described in
 	http://www.latex-project.org/lppl.txt

 This compiled work may be distributed and/or modified under the
 conditions of the LaTeX Project Public License, either version 1.3
 of this license or (at your option) any later version.
 The latest version of this license is in
	http://www.latex-project.org/lppl.txt
 and version 1.3 or later is part of all distributions of LaTeX
 version 2003/12/01 or later.

 This work has the LPPL maintenance status "maintained".

 Current maintainer of this work is kyanh <kyanh at o2 dot pl>.
 
 The following files constitute the `powerdot-doc-vi' bundle
 and *MUST* be distributed as a whole
 
  README
  powerdot-doc-vi.pdf
  powerdot-doc-vi-print.pdf
  img/
   lst-bookmarks.png
   tab-contents.png
   tab-slide-contents.png
  exa/
   README
   example-1.tex
   example-1.pdf
\end{example}

\newpage

\begin{abstract}
\noindent
%% \pf{powerdot} is a presentation class for \LaTeX\ that allows for
%% the quick and easy development of professional presentations. It
%% comes with many tools that enhance presentations and aid the
%% presenter. Examples are automatic overlays, personal notes and a
%% handout mode. To view a presentation, DVI, PS or PDF output can be
%% used. A powerful template system is available to easily develop new
%% styles.
\pf{powerdot} là lớp \LaTeX\ cho phép tạo trình diễn nhanh chóng,
chuyên nghiệp với kiểu dáng dễ dàng thay đổi. Lớp cung cấp nhiều công cụ
giúp thiết kế trình diễn: overlay, ghi chú cá nhân, chế độ handout.
Để xem trình diễn, bản DVI, PS, PDF của tài liệu đều có thể dùng được.
Lớp cung cấp hệ thống mẫu mạnh, giúp phát triển dễ dàng các mẫu mới.
\end{abstract}

\begin{multicols}{2}
[\section*{Mục lục}
\setlength{\columnseprule}{.4pt}
\setlength{\columnsep}{18pt}]
\tableofcontents
\end{multicols}

\newpage

% =====================================================================

\section*{Acknowledgements}
\addcontentsline{toc}{section}{Acknowledgements}

The authors are grateful to Darren Dale and Herbert Vo\ss\ for
testing the class and its output. We also wish to thank Mael
Hill\'ereau for contributing the |LyX| layout file and description.
Further, we like to thank all style contributors (see
section~\ref{sec:styles}) and everyone who provided feedback
through the mailinglist.

% =====================================================================

\section*{\texorpdfstring{Chú ý}{Chu y}}
\addcontentsline{toc}{section}{\texorpdfstring{Chú ý}{Chu y}}

% =====================================================================

\subsection*{\texorpdfstring{So sánh các phiên bản}{So sanh cac phien ban}}
\addcontentsline{toc}{subsection}{\texorpdfstring{So sánh các phiên bản}{So sanh cac phien ban}}

Vài tính năng ở phiên bản này có thể không được hỗ trợ ở phiên bản khác.
Mục này điểm qua các thay đổi quan trọng nhất bạn cần phải biết.
Chú ý đến các thay đổi được đánh dấu \ding{53}.

\begin{description}
\item[1.1] Cập nhật ngày 20/9/2005.
	\begin{dinglist}{51}
	\item[\ding{53}] phải dùng |size=12pt| thay vì |size=12| khi xác định
	cỡ chữ khi khai báo lớp. (Các cỡ khác tương tự.) Xem Mục~\vref{sec:classopts}.
	\item Thêm các kiểu trình diễn |elcolors|, |aggie|, |husky|, |sailor|.
	Các kiểu |tycja|, |ciment|, |fyma| được cải tiến. Xem Mục~\vref{sec:styles}.
	\item Hỗ trợ \LyX. Xem Mục~\vref{sec:lyx}.
	\end{dinglist}
\item[1.0] Phiên bản đầu tiên, công bố ngày 5/9/2005.
\end{description}


% =====================================================================

\subsection*{\texorpdfstring{Ghi chú của kyanh}{Ghi chu cua kyanh}}
\addcontentsline{toc}{subsection}{\texorpdfstring{Ghi chú của kyanh}{Ghi chu cua kyanh}}

Tài liệu này được biên dịch từ tài liệu chính thức của lớp \pf{powerdot}.
Có nhiều phần được thêm vào, một số phần khác được lược bớt.
Riêng Mục~\vref{sec:writestyle} chưa được dịch vì chưa cần thiết lắm
cho người dùng bình thường.

%% \medskip
%% \begin{version}{1.1}
%% Một số tùy chọn, hướng dẫn,... chỉ phù hợp với phiên bản nhất định của \pf{powerdot}.
%% Ví dụ, dòng này chỉ phù hợp với phiên bản 1.1 (về sau) của \pf{powerdot}.
%% \end{version}

\medskip
Nếu bạn muốn in (đen-trắng) tài liệu này, sử dụng tập tin |powerdot-doc-vi-print.pdf|.

\medskip
Là người dùng Việt, bạn có thể tìm kiếm sự giúp đỡ tại \url{http://www.viettug.org/}.

\medskip
Tài liệu này (bản |PDF|) được phân phối cùng với các ví dụ và hình ảnh như sau
đây. Bạn có thể tải về bản đầy đủ của tài liệu ở \url{http://www.viettug.org/}.
\begin{description}
\item[img/]
\item[\ding{42} lst-bookmarks.png] Danh sách |bookmark| của một trình diễn
\item[\ding{42} tab-contents.png] Bảng Mục lục ở một |slide|
\item[\ding{42} tab-slide-contents.png] Slide Mục lục và Bảng Mục lục
\item[exa/]
\item[\ding{42} example-1.tex] Ví dụ về \pf{powerdot} (nguồn).
\item[\ding{42} example-1.pdf] Ví dụ về \pf{powerdot} (PDF).
\end{description}

\medskip
Các hình ảnh minh họa các cho kiểu trình diễn
có thể tìm thấy trong tập tin nén |powerdot-dot-styles.zip|
(tải về ở \url{http://www.viettug.org/}).

% =====================================================================

\section{\texorpdfstring{Giới thiệu}{Gioi thieu}}
\label{sec:intro}

%% This class gives you the possibility to easily create professionally
%% looking slides. The class is designed to make the development of
%% presentations as simple as possible so that you can concentrate on
%% the actual content instead of keeping yourself busy with technical
%% details. Of course, some knowledge of \LaTeX\ is still required
%% though.
Lớp \pf{powerdot} cung cấp khả năng tạo trình diễn dễ dàng, chuyên nghiệp.
Lớp được thiết kế để việc thiết kế trình diễn trở nên đơn giản nhất có thể,
nhờ đó bạn không phải tốn thời gian với các yếu tố kỹ thuật.
Tất nhiên, bạn phải có các kiến thức cơ bản về \LaTeX{}.

%% This class builds on and extends the \pf{prosper} class
%% \cite{prosper} and the \pf{HA-prosper} package \cite{HA-prosper}.
%% The \pf{HA-prosper} package was initially intended to extend
%% \pf{prosper} and correct some bugs and problems of that class. As
%% developments on that package progressed, it was found that
%% unfortunately, not all of the problems could be overcome with the
%% package. That discovery was the start of a new project set up to
%% make a new class to replace the \pf{prosper} plus \pf{HA-prosper}
%% combination. You're currently reading the result of that project.
Lớp được xây dựng nhờ mở rộng lớp \pf{prosper} \cite{prosper}
và gói \pf{HA-prosper} \cite{HA-prosper}. Gói \pf{HA-prosper}
có mục đích ban đầu là mở rộng và khắc phục vài lỗi, nhược điểm của lớp
\pf{prosper}. Thật không may là, không phải mọi nhược điểm của \pf{propser}
đều có thể khắc phục được. Chính vì lý do này, một dự án mới ra đời
nhằm thay thế cho cả \pf{prosper} và \pf{HA-prosper}.
Bạn đang đọc tài liệu về chính dự án đó ;)

%% The remainder of this section will be devoted to giving a feel of
%% what the \pf{powerdot} presentation source looks like and giving an
%% overview of this documentation.
Phần còn lại của mục này giúp bạn có cái nhìn tổng quan về lớp \pf{powerdot}
và tài liệu hướng dẫn này.

%% The document structure of a presentation is always the same. You can
%% find it in the example below.
Cấu trúc của trình diễn luôn tương tự như ví dụ sau đây:

\begin{example}
 \documentclass[<class options>]{powerdot}
 \pdsetup{<pd options>}
 \begin{document}
   \begin{slide}{slide}
     noi dung
   \end{slide}
   \section{section}
   \begin{slide}[<slide options>]{slide}
     noi dung
   \end{slide}
   \begin{note}{ghi chu ca nhan}
     ghi chu
   \end{note}
 \end{document}
\end{example}

%% There are several elements that define the document structure. First
%% of all, the class accepts some class options that control the output
%% of the class, for instance, paper type and style. These class
%% options will be discussed in section~\ref{sec:classopts}. Then there
%% are presentation specific options which control some of the elements
%% of the presentation globally, for instance, the footers. These will
%% be discussed in section~\ref{sec:pdsetup}.
Có vài yếu tố tạo nên cấu trúc đó.
Đầu tiên, lớp chấp nhận vài tuỳ chọn (|class options|) cho phép điều khiển kết quả xuất
ví dụ, cỡ giấy, kiểu. Các tuỳ chọn này được bàn kỹ đến trong Mục~\vref{sec:classopts}.
Thứ đến, là các tuỳ chọn trình diễn (|pd options|) điều khiển toàn cục các
tính chất của trình diễn, ví dụ, các ghi chú ở chân trang.
Những tuỳ chọn này đưọc nói đến ở Mục~\vref{sec:pdsetup}.

%% Once the setup has been decided on, you can use the slide
%% environment to produce slides (see section~\ref{sec:slides}) and the
%% note environment to produce notes that go with the slides (see
%% section~\ref{sec:notes}). You can use overlays to display material
%% in steps. This is described in section~\ref{sec:overlays}. The
%% |\section| command provides a way to structure your presentation.
%% This is discussed in section~\ref{sec:structure}.
%% Section~\ref{sec:styles} will show an overview of the styles that
%% come with this class and the characteristics of each style.
%% Section~\ref{sec:compiling} will tell you more about how to produce
%% output. This section contains important information on required
%% packages.
Sau khi thiết lập với các tuỳ chọn, bạn có thể dùng môi truờng |slide|
để tạo các trang (|slide|) trình diễn (xem Mục~\vref{sec:slides}) và môi trường |note|
để tạo các ghi chú đi cùng với |slide| (xem Mục~\vref{sec:notes}).
Bạn có thể dùng |overlay| để thể hiện nội dung theo từng bước
(xem Mục~\vref{sec:overlays}). Lệnh |\section| giúp bạn tạo cấu trúc
cho trình diễn, giống như việc tạo chương, mục với tài liệu \LaTeX{} thông thường
(xem Mục~\vref{sec:structure}). Bạn cũng có thể lựa chọn các kiểu dáng
của trình diễn sau khi xem Mục~\vref{sec:styles}. Cuối cùng,
với Mục~\vref{sec:compiling}, bạn sẽ biết cách biên dịch tài liệu nguồn
để có kết quả là trình diễn thật sự. Ở mục này cũng có vài lưu ý về
việc cài đặt lớp \pf{powerdot}.

%% Section~\ref{sec:writestyle} is mostly interesting for people that
%% want to develop their own style for this class or want to modify
%% an existing style. This documentation concludes with a section
%% devoted to questions (section~\ref{sec:questions}), like `Where can
%% I find examples?'. It also tells you where to turn to in case your
%% questions are still not solved.
Mục~\vref{sec:writestyle} có lẽ là phần hấp dẫn đối với ai quan tâm
đến việc tạo kiểu dáng riêng cho trình diễn của mình, hoặc làm đẹp
các kiểu dáng đã có.

Tài liệu này kết thúc với Hỏi-Đáp (Mục~\vref{sec:questions}), có thể
giúp bạn trong những bước đầu làm quen với lớp \pf{powerdot}.

% =====================================================================

%\section{Setting up the presentation}
\section{\texorpdfstring{Thiết lập cho trình diễn}{Thiet lap cho trinh dien}}
\label{sec:setup}

% This section will describe all options that are available to control
% the output of the presentation and the looks of it.
Mục này mô tả các tuỳ chọn để điều khiển kết quả xuất của trình diễn.

% =====================================================================

%\subsection{Document class options}
\subsection{\texorpdfstring{Tuỳ chọn cho lớp}{Tuy chon cho lop}}

\label{sec:classopts}

%% We will start with the class options that are typed in the |\documentclass|
%% command as a comma-separated list. For each option, the default
%% value will be mentioned in the description. This is the value that
%% will be used if you decide to not give a value to the option or not
%% use the option at all.
Các tuỳ chọn được cho khi gọi lệnh |\documentclass|, cách nhau
bởi dấu phảy. Với mỗi tuỳ chọn dưới đây, giá trị mặc định được nêu
trong mô tả -- đó là giá trị sẽ được dùng khi bạn không nêu ra tuỳ chọn
khi gọi lệnh |\documentclass|, hoặc khi bạn nêu tuỳ chọn đó nhưng không
đi kèm giá trị nào.

\DescribeOption{mode}
%% This options controls the kind of output that we want to produce.
%% The default value is |present|.
Tuỳ chọn này xác định loại kết quả xuất, mặc định là |present|.
\begin{description}
\item\option{mode=present}\\
%% This mode is used when you want to create the actual presentation. It
%% will enable overlays and transition effects. You can read more about
%% overlays in section~\ref{sec:overlays}.
Dùng |present| nếu bạn muốn tạo trình diễn thật sự (thay vì chỉ để kiểm tra
trên màn hình). Với tuỳ chọn này, bạn có thể dùng |overlay| hoặc các hiệu ứng
biến đổi. Đọc thêm ở Mục~\vref{sec:overlays} về |overlay|.
\item\option{mode=print}\\
%% This mode can be used when printing the slides including their visual
%% markup, but without any overlay or transition effects.
Dùng |print| nếu bạn muốn in trình diễn của mình -- với bản in, hiển nhiên
các hiệu ứng hoặc overlay sẽ bị bỏ qua.
\item\option{mode=handout}\\
%% This mode will produce a black and white overview of your slides that
%% can be used to make personal notes on, for distribution to students,
%% a personal guide during your talk, etcetera.
Dùng |handout| để có được bản xem qua (|overview|) trình diễn, với hai màu
đen và trắng. Bản này thích hợp cho mục đích cá nhân, để phân phối cho sinh viên,
để minh họa trong khi bạn đang thuyết trình,...
\begin{description}
\item\option{nopagebreaks}\\
%% By default, the handout mode produces a document with two slides per
%% page. If you want to fit more slides on a page, specify this option
%% in the |\documentclass| command and \pf{powerdot} will let \LaTeX\
%% decide on the places to insert a page break, namely when a page is
%% full.
Theo mặc định, với bản |handout|, mỗi trang sẽ bố trí hai |slide|.
Nếu bạn muốn có nhiều hơn hai |slide| trên một trang, bạn hãy dùng tuỳ chọn
này (tuỳ chọn này không có giá trị) khi gọi |\documentclass|, nhờ đó
\pf{powerdot} giúp \LaTeX{} lựa chọn cách ngắt trang thích hợp.
\end{description}
\end{description}

\DescribeOption{paper}
%% This option has three possible values. The default value is |screen|.
Tuỳ chọn này có thể nhận ba giá trị sau. Mặc định là |screen|.
\begin{description}
\item\option{paper=screen}\\
%% This is a special format with screen optimized ratio (4/3). The
%% actual page dimensions will be 8.25 inch by 11 inch. This paper format
%% is not available for print or handout mode. In these modes, \pf{powerdot}
%% will switch to a4 paper and put a warning that it did this in the
%% log file of your presentation.
Trang trình diễn có tỷ lệ thông thường của màn hình (4/3).
Kích thước trang thực sự là 8.25 inch -- 11 inch. Kiểu này không
phù hợp với |mode=print| hoặc |mode=handout|. Nếu bạn cố dùng
|paper=screen| cùng với một trong |mode| vừa nói, \pf{powerdot} sẽ
tự động chuyển qua kiểu trang a4 kèm với lời nhắc nhở.
\item\option{paper=a4paper}\\
%A4 paper will be used for the presentation or handout.
Trang trình diễn bằng khổ a4, dùng với |mode=present| hoặc |mode=handout|.
\item\option{paper=letterpaper}\\
%Letter size paper will be used.
Trang trình diễn kiểu |letter|
\end{description}
%% Some important information with respect to paper size, compiling and
%% viewing presentations is available in section~\ref{sec:compiling}.
Vài thông tin quan trọng về cỡ trang, biên dịch và xem trình diễn
đưọc cho ở Mục~\vref{sec:compiling}.

\DescribeOption{orient}
%% This controls the orientation of the
%% presentation. The default value is |landscape|.
Tuỳ chọn này điều khiển hướng của trang, mặc định là |landscape| (nằm ngang).
\begin{description}
\item\option{orient=landscape}\\
%% The presentation will be in landscape format. This value is not
%% available in handout mode. In that mode, \pf{powerdot} will switch
%% to portrait orientation and will warn you about this in the log
%% file.
Trang trình diễn nằm ngang. Giá trị này không thích hợp với |mode=handout|.
Nếu cố dùng |mode| đó với tuỳ chọn |landscape|, \pf{powerdot} sẽ tự động
chuyển qua kiểu |portrait| kèm theo lời nhắc nhở.
\item\option{orient=portrait}\\
%% This produces slides in portrait
%% format. Notice that not all styles support portrait orientation. Please
%% refer to section~\ref{sec:styles} for information about which styles
%% do support the portrait orientation.
Bố trí trang theo chiều đứng. Chú ý rằng không phải mọi kiểu trình diễn
đều hỗ trợ chiều này. Vui lòng xem Mục~\vref{sec:styles} để biết thêm chi tiết.
\end{description}

\DescribeOption{display}
%% This controls the production of slides and
%% notes. The default value is |slides|.
Điều khiển việc xuất ra các trang trình diễn (|slide|) và ghi chú cá nhân (|note|).
Mặc định là |slides|.
\begin{description}
\item\option{display=slides}\\
%% This will only typeset the slides in your presentation.
Chỉ trình bày các trang trình diễn.
\item\option{display=slidesnotes}\\
%% This will typeset both the slides and the notes in your
%% presentation. See also section~\ref{sec:notes} for more information
%% about notes.
Trình bày đồng thời trang trình diễn và ghi chú cá nhân.
Xem thêm ở Mục~\vref{sec:notes}.
\item\option{display=notes}\\
%% This will typeset the notes only. To
%% be able to typeset the slide numbers of the notes correctly, one should
%% first run the presentation in slidesnotes mode once.
Chỉ xuất ra kết quả gồm các ghi chú cá nhân.
Để có thể cho đúng số trang, bạn cần phải biên dịch ở chế độ |slidenotes| trước.
\end{description}

%Here are some more options to control the output.
Dưới đây là vài tuỳ chọn khác điều khiển kết quả xuất.
\begin{description}
\item\DescribeOption{size}\option{size}\\
%% This is the size of the normal text font in points. Possible values
%% are 8, 9, 10, 11, 12, 14, 17, 20 and the default value is
%% 11.\footnote{Note that sizes other than 10, 11 and 12 are
%% non-standard and it is assumed that you have the \pf{extsizes}
%% bundle \cite{extsizes} installed, which provides these sizes.}
Xác định cỡ chữ (văn bản). Các giá trị có thể là 8pt, 9pt, 10pt, 11pt, 12pt, 14pt, 17pt, 20pt.
Giá trị mặc định là 11pt.\footnote{Chú ý rằng, các cỡ không phải là 10pt, 11pt hay 12pt
không phải là kiểu chuẩn -- nên nếu dùng, trên hệ thống của bạn
phải có cài đặt gói \pf{extsizes} \cite{extsizes}.}

\textbf{\underline{Chú ý quan trọng:}}\\
Nếu dùng \pf{powerdot} phiên bản 1.0, các giá trị của |size|
là các số 8, 9, \ldots Với \pf{powerdot} phiên bản 1.1 trở về sau,
bạn chỉ ra 8pt, 9pt, \ldots khi dùng tuỳ chọn này.
\item\DescribeOption{style}\option{style}\\
%% This controls the style to be loaded for the presentation. By
%% default, the \pf{default} style will be loaded. For more styles, see
%% section~\ref{sec:styles}.
Xác định kiểu trình diễn, mặc định là \pf{default}. Xem thêm Mục~\vref{sec:styles}
để biết thêm chi tiết về các kiểu.
\item\DescribeOption{fleqn}\option{fleqn}\\
%% This option makes equations flushed left. It does the same as the
%% equally named option for the article class.
Đánh chỉ số cho phương trình ở bên trái.
\item\DescribeOption{leqno}\option{leqno}\\
% Put equation numbers at the left. Also the same as in the article class.
Đánh chỉ số cho phương trình ở bên phải.
\item\DescribeOption{nopsheader}\option{nopsheader}\\
%% By default, \pf{powerdot} will write a postscript command to the ps
%% file to make sure that post processors like ps2pdf know which paper
%% to use without the need to specify it on the command line. See also
%% section~\vref{sec:compiling}. If you experience problems with post
%% processing or printing or you want to specify the paper size in the
%% post processing steps yourself, use this option.
Theo mặc định, \pf{powerdot} sẽ ghi các lệnh |postscript| vào kết quả
xuất dạng |PS|, nhờ đó các chương trình tương tự |ps2pdf| khi chuyển
kết quả xuất |PS| sang dạng |PDF| sẽ xác định đúng kiểu trang -- bạn không
cần phải chỉ ra khi gọi các chương trình đó ở dòng lệnh. Xem thêm
Mục~\vref{sec:compiling}. Nếu bạn gặp trở ngại khi chuyển sang dạng |PDF|
với các chương trình đó, hoặc khi in ấn, bạn có thể dùng tuỳ chọn này --
và khi đó bạn nhớ chỉ ra kiểu giấy khi gọi các chương trình chuyển.
\item\DescribeOption{hlentries}\option{hlentries}\\
%% This highlights table of contents entries when the entry matches
%% with the current slide and is |true| by default. See also
%% section~\vref{sec:structure}. If you don't want highlighting of table
%% of contents entries (for instance in print mode), use
%% |hlentries=false|.
Ở Bảng Mục lục,\footnote{Liệt kê các mục và các slide của trình diễn;
bảng Mục lục xuất hiện ở mọi slide, trừ vài trường hợp đặc biệt.}
mỗi thành phần sẽ được tô màu nổi bật (|highlight|)
nếu nó tương ứng với |slide| hiện tại. Nếu bạn không muốn điều này,
hãy dùng |hlentries=false|. Mặc định là |hlentries=true|. Xem thêm
ở Mục~\vref{sec:structure}.
\item\DescribeOption{hlsections}\option{hlsections}\\
%% This highlights table of contents sections when the section matches
%% with the current section in the presentation and is |false| by
%% default. See also section~\vref{sec:structure}. Specifying this
%% option turns highlighting of sections on. This could be useful when
%% you are using a style that implements a split table of contents.
Tương tự như trên, nhưng là các thành phần tương đương với |section|.
Giá trị mặc định là |false|. Giá trị |true| có ích khi bạn dùng các kiểu
trình diễn chia nhỏ bản Mục lục.
\item\DescribeOption{blackslide}\option{blackslide}\\
%% This option inserts a black slide in the presentation on page 1 and
%% will automatically advance to page 2 when opening the presentation
%% in a PDF viewer like Acrobat (Reader). The black slide has an
%% embedded target called |blackslide| and you can make a clickable
%% link to this slide by using
Với tuỳ chọn này, một trang |slide| đen sẽ chèn vào đầu trình diễn (như vậy
|slide| đen ở trang 1); trình diễn của bạn sẽ tự động nhảy qua trang 2 khi 
được mở trong chương trình xem PDF (ví dụ Acrobat Reader). Trang |slide| đen
đã được nhúng trước một đích gọi là |blackslide| nhờ đó bạn có thể thêm
ở trang |slide| khác, chẳng hạn liên kết sau
\begin{example}
 \hyperlink{blackslide}{Click here to go to the black slide}
\end{example}
%% When you click anywhere in the black slide, you will go back to the
%% originating slide. This option can be used to temporarily pause a
%% presentation, for instance, to do a proof on the black board.
Khi bạn |click| ở bất kỳ vị trí nào của |slide| đen, bạn sẽ đi về đúng
trang |slide| ban đầu. Tuỳ chọn này có ích, chẳng hạn khi bạn muốn tạm
ngưng trình diễn để thuyết minh trên bảng đen.
\end{description}

%Here is an example of a |\documentclass| command.
Dưới đây là vài ví dụ giúp bạn hình dung lại các điều đã nói.
\begin{example}
 \documentclass[
   size=12,
   paper=screen,
   mode=present,
   display=slidesnotes,
   style=tycja,
   nopagebreaks,
   blackslide,
   fleqn
 ]{powerdot}
\end{example}
%% This example sets up a presentation in \pf{tycja} style, with a black
%% slide, normal size 12 points and flushed left equations.
Với ví dụ này, trình diễn sẽ dùng kiểu \pf{tycja}, với trang |slide| đen,
cỡ chữ 12 điểm, các phương trình được đánh số bên trái.
\begin{example}
 \documentclass[
   size=12,
   paper=letterpaper,
   mode=handout,
   display=slidesnotes,
   style=tycja,
   nopagebreaks,
   blackslide,
   fleqn
 ]{powerdot}
\end{example}
%% Changing the |paper| and |mode| options, now produces a handout with
%% possibly more than two slides per page due to the |nopagebreaks|
%% option.
Ở ví dụ này, ta thay đổi kiểu trang (|paper|) và chọn |mode=handout|
cùng với tuỳ chọn |nopagebreaks| -- nhờ đó nhiều hơn hai trang |slide|
sẽ được bố trí trên một trang giấy!

% =====================================================================

%\subsection{Setup options}
\subsection{\texorpdfstring{Thiết lập trình diễn}{Thiet lap trinh dien}}
\label{sec:pdsetup}

\DescribeMacro{\pdsetup}
%% There are several extra options that can help customizing your
%% presentation. These options are not available via the
%% |\documentclass| command. This has a technical reason.\footnote{The
%% interested reader is referred to the section about the \pf{xkvltxp}
%% package in the \pf{xkeyval} package documentation \cite{xkeyval}.}
%% The options can be accessed via the |\pdsetup| command, which can
%% only be used in the preamble of your presentation. The command takes
%% one argument, which should contain a comma-separated list of
%% options. The available options are described below.
Dưới đây là các tuỳ chọn giúp bạn điều khiển tốt hơn nữa trình diễn của mình.
Các tuỳ chọn này không phải chỉ ra khi gọi lệnh |\documentclass|,
mà được chỉ ra trong tham số của lệnh |\pdsetup|.
Có sự phân biệt này là vì lý do kỹ thuật.\footnote{%
Nếu bạn quan tâm, thử tìm đọc mục về \pf{xkvltxp}
trong tài liệu \cite{xkeyval} về gói \pf{xkeyval}).}
Lệnh |\pdsetup| chấp nhận chỉ một tham số, là chuỗi các tuỳ chọn cách
nhau bằng dấu phảy.
\begin{description}
\item\DescribeOption{lf}\option{lf}\\
%This determines the content of the left footer. By default, this is empty.
Xác định nội dung của |footer| (chân trang) bên trái. Mặc định là rỗng.
\item\DescribeOption{rf}\option{rf}\\
% This determines the content of the right footer. By default, this is empty.
Như trên, nhưng cho bên phải.
\item\DescribeOption{theslide}\option{theslide}\\
%% This option controls how the slide number appears on the slide. By
%% default this has the value |\arabic{slide}~/~\pageref*{lastslide}|,
%% which could appear like |5/22|. Notice that the |\arabic{slide}|
%% typesets the number of the current slide and that
%% |\pageref*{lastslide}| typesets the number of the last
%% slide.\footnote{We use the starred version of \cs{pageref} which is
%% defined by \pf{hyperref} and does not create a link to the page that
%% it is referring to.}
Tuỳ chọn này xác định cánh đáng số các |slide|.
Giá trị mặc định là
\begin{command}
  `\cs{arabic}'{slide}~/~`\cs{pageref}'*{lastslide}
\end{command}
mà với nó, các |slide| sẽ đánh số tương tự như |5/22| (slide 5, tổng số slide 22).
Chú ý rằng, |\arabic{slide}| cho biết |slide| hiện tại, và |\pageref*{lastslide}|
tổng số |slide|.%
\footnote{Ta phải dùng \cs{pageref*}, phiên bản có sao của \cs{pageref},
được định nghĩa bởi \pf{hyperref} nhờ đó ta thu đưọc số trang chứ không phải
là liên kết đến trang.}
\item\DescribeOption{thenote}\option{thenote}\\
%% This is similar to the |theslide| option, but typesets the slide
%% numbers of notes. The default value is
%% |note~\arabic{note}~of~slide~\arabic{slide}| and |\arabic{note}|
%% here typesets the number of the current note that goes with the
%% current slide. This could appear like |note 2 of slide 7|.
Tương tự như tuỳ chọn |theslide|, nhưng dành cho các ghi chú.
Giá trị mặc định là
\begin{command}
  note~`\cs{arabic}'{note}~of~slide~`\cs{arabic}'{slide}
\end{command}
ở đây |\arabic{note}| là số chỉ ghi chú hiện tại. Kết quả sẽ tương tự
như |note 2 of slide 7|. Với người dùng Việt, có lẽ phải định nghĩa lại
như sau:
\begin{command}
  thenote=ghi chu~`\cs{arabic}'{note}~/~`\cs{arabic}'{slide}
\end{command}
\item\DescribeOption{trans}\option{trans}\\
%% This option sets the default transition effect to be used in the
%% presentation. These transition effects only work after compiling the
%% presentation to PDF format. See also section~\ref{sec:compiling}.
%% The following transition effects are supported: |Split|, |Blinds|,
%% |Box|, |Wipe|, |Dissolve|, |Glitter| and |Replace|. When you are using
%% a viewer that understands PDF 1.5, you can also use |Fly|, |Push|,
%% |Cover|, |Uncover| or |Fade|. It is important to notice that most
%% viewers are case sensitive, so, for instance, |box| will not work.
Xác định hiệu ứng biến đổi mặc định cho trình diễn. Các hiệu ứng này
chỉ thấy được trong kết quả |PDF|. Xem thêm ở Mục~\vref{sec:compiling}.
Các hiệu ứng được hỗ trợ bao gồm: |Split|, |Blinds|,
|Box|, |Wipe|, |Dissolve|, |Glitter| và |Replace|.
Nếu bạn dùng chương trình xem hiểu dạng |PDF| 1.5 trở lên, bạn
có thêm các hiệu ứng |Fly|, |Push|, |Cover|, |Uncover| và |Fade|.
Điều quan trọng cần phải nhớ là, hầu hết các trình xem |PDF| đều phân
biệt hoa thường, nên bạn phải chỉ ra |Box| thay vì |box|,...

%% The default effect is |Replace| which just replaces one slide with
%% another when browsing the slides. Note that some PDF viewers (like
%% Acrobat Reader 5 and higher) only produce the transition effect in
%% full screen mode. If you want to use a custom transition effect that
%% is not listed in the list above (for instance, a wipe effect with a
%% custom wipe direction), then that is possible. However,
%% \pf{powerdot} will put a warning in your log file that the effect
%% that you have chosen, might not work in the PDF viewer. Here is an
%% example that does work.
Hiệu ứng mặc định là |Replace| -- theo đó, khi xem trình diễn, |slide|
sau sẽ thay thế |slide| trước. Để ý rằng, vài trình xem |PDF| (ví dụ
Acrobat Reader 5) chỉ tạo các hiệu ứng khi xem ở chế độ toàn màn hình.
Nếu bạn dùng một hiệu ứng riêng không có trong danh sách trên (ví dụ
hiệu ứng |Wipe| với hướng theo ý bạn), \pf{powerdot} sẽ nhắc nhở về
hiệu ứng đó (rằng có thể chúng không xem được). Dưới đây là ví dụ
\begin{example}
  trans=Wipe /Di 0
\end{example}
%% In Acrobat (Reader), this wipes from left to right instead of the
%% default top to bottom. For more information, see a PDF Reference
%% Manual.
Trong Acrobat (Reader), hiệu ứng |Wipe| khai báo như trên sẽ có hướng
từ trái sang phải, thay vì từ trên xuống dưới như mặc định.
Để biết thêm chi tiết, xem tài liệu tham khảo về định dạng |PDF|.
\item\DescribeOption{counters}\option{counters}\\
%% The |counters| option lists counters that you might want to protect
%% on overlays. As material on overlays (see
%% section~\ref{sec:overlays}) is processed multiple times, also
%% \LaTeX\ counters, like the |equation| counter, might be increased
%% too often. To avoid that your equations get different numbers on
%% every overlay, use this option. The |equation|, |table|, |figure|,
%% |footnote| and |mpfootnote| counters are already protected for you.
%% If you use extra counters, for instance for theorems, list them in
%% this option. Example:
Danh sách |counter| (bộ đếm) cần được bảo vệ khi |overlay|.
Tại sao? Các phương trình chẳng hạn, khi |overlay| (xem Mục~\vref{sec:overlays}
sẽ được xử lý nhiều lần, do đó, số đếm của chúng không ngừng tăng lên
nếu không được bảo vệ. Các bộ đếm được bảo vệ theo mặc định là |equation|,
|table|, |figure|, |footnote| và |mpfootnote|. Nếu bạn muốn bảo vệ thêm
chẳng hạn |theorem| và |lemma|, hãy dùng như ví dụ sau:
\begin{example}
  counters={theorem,lemma}
\end{example}
\item\DescribeOption{list}\option{list}\\
%% This option takes a list of options that will be passed on to the
%% \pf{enumitem} package that controls the layout of lists created by
%% the |enumerate| and |itemize| environments. Example:
Tuỳ chọn này chấp nhận một tham số, là danh sách các tuỳ chọn
dành riêng cho gói \pf{enumitem} để điều khiển kết quả của danh sách
tạo bởi các môi trường |enumerate| và |itemize|. Ví dụ
\begin{example}
  list={labelsep=1em,leftmargin=*,itemsep=0pt,topsep=5pt,parsep=0pt}
\end{example}
%% See for more information on controlling the layout of lists the
%% \pf{enumitem} package \cite{enumitem}.
Hãy xem tài liệu hướng dẫn của gói \pf{enumitem} để có thêm các tuỳ chọn.
\item\DescribeOption{enumerate}\option{enumerate}\\
%As the |list| option, but only control |enumerate| environments.
Tương tự tuỳ chọn |list|, nhưng chỉ dành cho môi trường |enumerate|.
\item\DescribeOption{itemize}\option{itemize}\\
%As the |list| option, but only control |itemize| environments.
Tương tự tuỳ chọn |list|, nhưng chỉ dành cho môi trường |itemize|.
\end{description}

%% Here is an example of a |\pdsetup| command that one could use to set up
%% the presentation.
Dưới đây là ví dụ về cài đặt cho trình diễn dùng lệnh |\pdfsetup|.
\begin{example}
 \pdsetup{
   lf=Trinh dien cua toi,
   rf=VietTUG,
   trans=Wipe,
   theslide=\arabic{slide}
 }
\end{example}
%% This sets the left and right footers and will initialize the
%% transition effect to |Wipe|. Further, slide numbers will not include
%% the number of the last slide, but only the number of the current
%% slide.
Với thiết lập như trên, sẽ có chân trang bên trái và bên phải,
hiệu ứng mặc định cho trình diễn là |Wipe|; ngoài ra,
chỉ số |slide| sẽ không bao gồm tổng số |slide| như mặc định,
mà chỉ gồm số chỉ |slide| hiện tại.

%% A small note is necessary with respect to the appearance of footers.
%% The slide number (controlled by the |theslide| option) will be added
%% to a footer. Most styles add it too the right footer. If both the
%% footer and the slide number are non empty, |~--~| will be inserted
%% in between them to separate them. Styles might modify this default
%% behavior however.
Một chú ý nhỏ về cách thể hiện ở chân trang (|footer|).
Chỉ số |slide| (điều khiển bởi tuỳ chọn |theslide|) sẽ được thêm vào
chân trang (|footer|). Hầu hết các kiểu trình diễn đều thêm vào bên phải.
Nếu cả |footer| và chỉ số |slide| đều không rỗng, |~--~| sẽ được thêm
vào để ngăn cách chúng. Khi thiết kế kiểu, ta có thể thay đổi
cách xử lý này.

% =====================================================================

\section{\texorpdfstring{Tạo trình diễn}{Tao trinh dien}}
\label{sec:slides}

% =====================================================================

\subsection{\texorpdfstring{Trang tiêu đề}{Trang tieu de}}
\label{sec:titleslide}

\DescribeMacro{\title}
\DescribeMacro{\author}
\DescribeMacro{\and}
\DescribeMacro{\date}
\DescribeMacro{\maketitle}
%% The title slide is created by the |\maketitle| command. Its use is
%% the same as in the standard \LaTeX\ document classes. See an example
%% below.
Trang tiêu đề của trình diễn được tạo bởi |\maketitle|, tương tự như cách
quen thuộc trong \LaTeX{} chuẩn. Ví dụ
\begin{example}
 \documentclass{powerdot}
   \title{Title}
   \author{You \and me}
   \date{September 14, 2005}
 \begin{document}
    \maketitle
    ...
 \end{document}
\end{example}
%% The |author|, |title| and |date| declarations provide the text to be
%% used when making a title page. The design of the title page is
%% specific to the style in use. Notice the use of |\and| for
%% separating multiple authors. See a \LaTeX\ manual \cite{companion}
%% for more information on commands such as |\title| and |\author|.
Các khai báo |author|, |title| và |date| cho biết lần lượt
tác giả, tiêu đề và ngày tháng cùa trình diễn. Trang tiêu đề được
trình bày theo thiết kế của kiểu trình diễn. Xem tài liệu \cite{companion}
về chi tiết của các lệnh |\title| và |\author|.

% =====================================================================

%\subsection{Other slides}
\subsection{\texorpdfstring{Tạo slide}{Tao slide}}

\label{sec:otherslides}

\DescribeEnv{slide}
%% The centerpiece of every presentation is the
%% slide. In \pf{powerdot}, the content of each slide is placed in a
%% |slide| environment.
Phần trung tâm của mọi trình diễn là |slide|.
Với \pf{powerdot}, nội dung của mỗi |slide| được đặt trong môi trường
|slide|.
\begin{command}
 `\cs{begin}\texttt{\{slide\}}\oarg{options}\marg{slide title}'
 `\meta{body}'
 `\cs{end}\texttt{\{slide\}}'
\end{command}

%% In section~\ref{sec:overlays} we'll see how to give some life to the
%% slides, but for now, let's look at a simple example.
Ở Mục~\vref{sec:overlays}, bạn sẽ biết tinh chỉnh với tuỳ chọn |overlay|.
Bây giờ, hãy xem một ví dụ đơn giản:
\begin{example}
 \begin{slide}{First slide}
   Hello World.
 \end{slide}
\end{example}
%% The slide environment has one required argument, namely the slide
%% title. When a slide is created, the slide title is used to create an
%% entry in the table of contents and in the list of bookmarks. The
%% table of contents is a listing of the slides and section titles in
%% the presentation that appears on each slide.
Môi trường |slide| có một tham số bắt buộc, là tựa của |slide|.
Khi một |slide| được tạo ra, tựa của nó sẽ được liệt kê trong
Bảng Mục lục và trong danh sách các |bookmark|. Bảng Mục lục
liệt kê tựa của tất cả các |slide| và các mục của trình diễn, và nó sẽ
xuất hiện ở mọi |slide|.

%% The table of contents is clickable (when the presentation is
%% compiled into PDF) and serves as a nice way to jump from location to
%% location within the presentation. The bookmark list is only present
%% when compilation is taken all the way to the PDF file format. It
%% also serves as a table of contents, but this list does not appear on
%% \textit{any} of the slides, but in a separate window in a PDF
%% viewer. In the example above, the entries in both table contents and
%% the list of bookmarks would be titled |First slide|.
Ở Bảng Mục lục, có các liên kết (khi biên dịch qua dạng |PDF|) nhờ đó
bạn có thể dễ dàng tìm kiếm các |slide| trong trình diễn của mình.
Trong khi đó, danh sách |bookmark| cũng tương tự như Bảng Mục lục, nhưng
nó không xuất hiện bất kỳ |slide| nào, mà ở một cửa sổ đặc biệt của trình
xem |PDF|. Trong ví dụ trên, tựa |First slide| sẽ xuất hiện trong cả
Bảng Mục lục và danh sách |bookmark|. Danh sách |bookmark| chỉ có khi
biên dịch kết quả thành dạng |PDF|.

%% The \meta{options} for the |slide| environment allow the user to
%% specify alternative titles for the table of contents and bookmark
%% entries. There is also a |trans| option that works only for the
%% current slide.
Phần tham số bổ sung (hay tuỳ chọn) (\meta{options})
của |slide| cho phép chỉ định tựa sẽ xuất hiện ở Bảng Mục lục
và danh sách |bookmark|. Tuỳ chọn cũng giúp ta chỉ định hiệu ứng (|trans|)
dùng riêng cho |slide| đó thôi.

\begin{description}
\item\DescribeOption{toc}\option{toc}\\
%% When specified, the value is used for the entry in the table of
%% contents; otherwise, the slide title is used. If |toc=| is
%% specified, then no entry is created.
Chỉ định tựa sẽ xuất hiện trong Bảng Mục lục, thay vì tựa của |slide|.
Nếu dùng |toc=|, sẽ không có phần tử tương ứng nào được tạo ra ở
Bảng Mục lục.
\item\DescribeOption{bm}\option{bm}\\
%% When specified, the value is used for the bookmark entry; otherwise,
%% the slide title is used. If |bm=| is specified, then no entry is
%% created.
Tương tự như trên, nhưng cho danh sách |bookmark|.
\item\DescribeOption{trans}\option{trans}\\
%% This works the same as the |trans| option described in
%% section~\ref{sec:pdsetup}, except that it sets the transition effect
%% of the current slide only (when used in the slide \meta{options})
%% and not for the entire presentation.
Cách dùng tùy chọn này hoàn toàn tương tự như mô tả ở Mục~\vref{sec:pdsetup};
các thay đổi mà tuỳ chọn này tạo ra chỉ có tác dụng trong |slide|
đang xét, không phải cho toàn bộ trình diễn hay các |slide| khác.
\end{description}

%% These optional arguments are especially useful when the title of a
%% slide is extremely long or when the title contains \LaTeX\ commands
%% that do not render correctly in the bookmarks.\footnote{The
%% bookmarking procedure uses \cs{pdfstringdef} from the \pf{hyperref}
%% package, and it can process accented characters such as \cs{"i}.}
%% When specifying entries, be sure to hide special characters `|,|'
%% and `|=|' between curly brackets `|{|' and `|}|'. Let's look at an
%% example that uses these optional arguments.
Các tuỳ chọn |toc| và |bm| đặc biệt hữu ích nếu tựa của |slide|
quá dài hoặc khi nó chứa các lệnh \LaTeX{} mà kết quả của lệnh đó
không thể hiện tốt trong danh sách |bookmark|.\footnote{Các bookmark
được tạo bằng cách dùng \cs{pdfstringdef} từ gói \pf{hyperref},
có thể chấp nhận vài ký tự có dấu chẳng hạn \cs{"i}.}
Khi dùng tuỳ chọn |toc| và |bm|, bạn nhớ đặt các ký tự đặc biệt
`|,|' và `|=|' bên trong cặp dấu ngoặc `|{|' và `|}|'. Dưới đây là ví dụ:

\begin{example}
 \begin{slide}[toc=,bm={LaTeX, i*i=-1}]{\color{red}\LaTeX, $i^2=-1$}
   My slide contents.
 \end{slide}
\end{example}

%% In this example, the slide title will appear as {\color{red}\LaTeX,
%% $i^2=-1$}. This text will not render correctly in a bookmark entry.
%% An attempt is made to correct this, but often, the correction does
%% not produce an equivalent text. This particular title would be
%% rendered in the bookmark list as |redLaTeX, i2=-1|. On the other
%% hand, the manually specified bookmark entry is rendered as:
%% |LaTeX, i*i=-1|. Notice, no entry is created in the table of contents,
%% because of the use of |toc=|.
Ở ví dụ này, tựa của |slide| sẽ là {\color{red}\LaTeX, $i^2=-1$}.
Kết quả này không thể nào thể hiện đúng trong danh sách |bookmark|.
Đôi khi, việc cố gắng sửa chữa tựa có thể mang lại kết quả, nhưng
thường thì cố gắng không như ý. Chính vì thế, ta đã dùng tuỳ chọn |bm|,
nhờ đó kết quả thu được ở danh sách |bookmark| sẽ là |LaTeX, i*i=-1|.
Chú ý rằng, không có tựa nào được tạo ra ở Bảng Mục lục, do ta đã chỉ
định |toc=|.

%% In addition to the |slide| environment, each individual style can
%% define its own environments. Many styles have a |wideslide|
%% environment. The idea is that one might have information that does
%% not fit nicely on a slide with a table of contents listed, as this
%% consumes some space. In such cases, it is preferable to use a slide
%% that does not list the table of contents. The |wideslide|
%% environment provides this functionality and has more space for the
%% actual slide content. See section~\ref{sec:styles} for information
%% on the various environments provided by the styles.
Cùng với môi trường |slide|, các kiểu trình diễn có cung cấp các
môi trường riêng để tạo |slide|. Phổ biến nhất là môi trường |wideslide|
để giải quyết vấn đề: đôi khi, nội dung của |slide| quá khổ, không thể
bố trí trên một trang cùng với Bảng Mục lục; môi trường |wideslide|
cho ta |slide| đặc biệt ở đó không xuất hiện Mục lục. Xem thêm Mục~\vref{sec:styles}
để biết thêm các môi trường cung cấp bởi các kiểu trình diễn.

% =====================================================================

\section{Overlay}
\label{sec:overlays}

%% It is often the case that you don't want all the information on the
%% slide to appear at once. Rather, the information should appear one
%% item at a time. In \pf{powerdot}, this is achieved with overlays.
%% Each slide can be comprised of many overlays, and the overlays are
%% displayed one at a time.
Với trình diễn, nhiều khi bạn không muốn mọi thông tin trên |slide|
xuất hiện cùng một lúc, mà tuần tự từng ý từng ý một. Với \pf{powerdot},
bạn có thể đạt kết quả này nhờ |overlay|. Mỗi |slide| hiển nhiên
có thể gồm nhiều |overlay|; tại mỗi thời điểm, chỉ có một |overlay|
được thể hiện.

% =====================================================================

%\subsection{The \cs{pause} command}
\subsection{\texorpdfstring{Lệnh}{Lenh} \cs{pause}}

\label{sec:pause}

\DescribeMacro{\pause}
%% The easiest way to display information
%% sequentially is to use the |\pause| command.
Cách đơn giản nhất để tuần tự hóa thông tin trên |slide|
là dùng lệnh |\pause|.
\begin{command}
 `\cs{pause}\oarg{number}'
\end{command}
Dưới đây là ví dụ đơn giản:
\begin{example}
 \begin{slide}{Simple overlay}
   power\pause dot
 \end{slide}
\end{example}
%% The slide's information is displayed and continues until the
%% |\pause| command is encountered. No further output within the same
%% slide is displayed until the click of the mouse or the touch of the
%% keyboard. Then, the content will continue to display until all the
%% information is displayed or until another |\pause| command is
%% encountered. In this example, |power| is displayed on the first
%% overlay, and |powerdot| is the displayed on the second overlay.  The
%% |\pause| command is often used within the |itemize| and |enumerate|
%% environments. For example,
Thông tin của |slide| được thể hiện và tiếp tục cho đến khi gặp lệnh |\pause|
đầu tiên. Không có thêm kết quả nào xuất hiện cho đến khi bạn |click|
chuột hoặc gõ phím bất kỳ. Sau đó, nội dung |slide| sẽ được thể hiện
cho đến khi mọi thông tin đều xuất hiện hoặc cho tới khi gặp lệnh |\pause|
kế tiếp. Trong ví dụ trên, |power| sẽ xuất hiện trong |overlay| đầu tiên,
và |powerdot| sẽ xuất hiện ở |overlay| tiếp theo. Lệnh |\pause| thường được
dùng cùng với các môi trường |itemize| hoặc |enumerate|. Ví dụ
\begin{example}
 \begin{slide}{Multiple pauses}
   power\pause dot \pause
   \begin{itemize}
     \item Let me pause\ldots \pause
     \item \ldots while I talk \pause and chew bubble gum. \pause
     \item Perhaps you'll be persuaded.
     \item Perhaps not.
   \end{itemize}
 \end{slide}
\end{example}
%% Since |\pause| was used before the |itemize| environment, no item
%% will appear until the third overlay. Then, each item will be
%% displayed one at a time, each on their own overlay. More information
%% on using lists will follow in the next section.
Bởi vì |\pause| được dùng trước môi trường |itemize|, không có
phần tử nào của danh sách xuất hiện cho tới |overlay| thứ ba.
Sau đó, các phần tử sẽ lần lượt xuất hiện trong các |overlay|
kế tiếp. Việc điều khiển danh sách sẽ được bàn kỹ hơn ở Mục kế tiếp.

%% The optional argument of the |\pause| command specifies the number
%% of overlays to pause. An example usage is:
Lệnh |\pause| chấp nhận tham số bổ sung cho biết phải dừng bao nhiêu |overlay|.
\begin{example}
 \begin{slide}{Pause longer}
   \begin{itemize}
     \item A \pause
     \item B \pause[2]
     \item C
   \end{itemize}
 \end{slide}
\end{example}
%% In the example above, item |C| will appear on the fourth overlay.
%% The usefulness of this option will become more apparent in the next
%% section; so we will revisit a similar example at that time.
Trong ví dụ trên, phần tử |C| của danh sách xuất hiện ở |overlay|
thứ bốn. Tuỳ chọn có vẻ vô dụng này sẽ được đề cập lại ở Mục tiếp theo,
ở đó ta sẽ gặp lại ví dụ tương tự ví dụ trên đây.

% =====================================================================

%\subsection{List environments}
\subsection{\texorpdfstring{Tạo danh sách}{Tao danh sach}}

\label{sec:lists}

%% The list environments, |itemize| and |enumerate|, have special
%% treatments in \pf{powerdot}. They have an optional argument that
%% will be taken care off by the \pf{enumitem} package (see
%% \cite{enumitem}). \pf{powerdot} supplies an extra key for this
%% optional argument. In the examples that follow, features will be
%% described using the |itemize| environment but they also apply to the
%% |enumerate| environment.
Các môi trường tạo danh sách |itemize| và |enumerate| được xử lý
đặc biệt trong \pf{powerdot}. Chúng đều có tham số tuỳ chọn điều khiển
bởi gói \pf{enumitem} (xem \cite{enumitem}). Lớp \pf{powerdot} cung
cấp nhiều khóa cho tuỳ chọn này. Trong các ví dụ sau đây, các tính năng
được minh họa với môi trường |itemize|, nhưng bạn cũng có thể áp dụng
cách làm tương tự cho môi trường |enumerate|.

%Here is the typical usage of the |itemize| environment:
Dưới đây là ví dụ đơn giản nhất về cách dùng môi trường |itemize|:
\begin{example}
 \begin{slide}{Basic itemize}
   \begin{itemize}
     \item A \pause
     \item B \pause
     \item C
   \end{itemize}
 \end{slide}
\end{example}
%% The display is simple, each item appears one at a time with each
%% overlay.
Kết quả của ví dụ là danh sách đơn giản, mỗi phần tử của nó
được thể hiện ở một |overlay|.

\DescribeOption{type}
%% Suppose we wanted every item to show, but we only wanted one item to
%% appear `active' at once. This can be accomplished via the |type|
%% option for the |itemize| environment. The default value is |0|.
Giả định rằng ta muốn mọi phần tử của danh sách sẽ được thể hiện,
nhưng tại mỗi thời điểm chỉ một và một phần tử của danh sách xuất hiện
ở |slide|. Điều này có thể đạt được bằng cách chỉ định kiểu với |type|
trong phần tham số bổ sung của môi trường (giá trị mặc định là |0|).
\begin{example}
 \begin{slide}{Type 1 itemize}
   \begin{itemize}[type=1]
     \item A \pause
     \item B \pause
     \item C
   \end{itemize}
 \end{slide}
\end{example}
%% Now, every item will be displayed in the \emph{inactive
%% color}\index{inactive color} (which is defined by the style that you
%% use), and the item's font color will become the active one on the
%% overlay that it would normally appear on. The default behavior is
%% given by |type=0|.
Bây giờ, mọi phần tử đều được thể hiện với \emph{màu nhạt}\index{tô màu nhạt}
(màu được định nghĩa bởi kiểu trình diễn); khi |overlay| của một phần
tử được thể hiện, phần tử đó sẽ được tô đậm để làm nổi bật nó so
với các phần tử khác. Sự phân biệt của hai |type| có thể minh họa như
sau (dấu |*| chỉ màu nhạt):
\begin{example}
     type=0        type=1
  1: x------       x------
  2: xx-----       *x----- 
  3: xxx----       **x---- 
  4: xxxx---       ***x--- 
  5: xxxxx--       ****x--
  6: xxxxxx-       *****x-
\end{example}

%% Lists can also be nested to create complicated structures. When a
%% list is nested, it inherits the setting of the |type| option from
%% the `parent' list, but that can be overruled by specifying the
%% |type| option in the optional argument of the nested list. We
%% present here one example, but many more can be created by nesting
%% lists of different types in different ways.
Các danh sách có thể lồng nhau. Khi xảy ra trường hợp này, các danh sách
thứ cấp sẽ thừa hưởng giá trị của |type| ở môi trường mẹ.
Tất nhiên, với bất kỳ danh sách nào, ta cũng có thể đặt lại giá trị cho |type|.
Dưới đây là ví dụ về hai danh sách lồng nhau:
\begin{example}
 \begin{slide}{Nested lists}
   \begin{itemize}
     \item A\pause
     \begin{itemize}[type=1]
       \item B\pause
     \end{itemize}
     \item C
   \end{itemize}
 \end{slide}
\end{example}
%% This displays |A| and |B| on the first overlay, but |B| is inactive.
%% On overlay 2, |B| will become active and on overlay 3, |C| will
%% become visible.
Kết quả là phần tử |A| và |B| được thể hiện ở |overlay| đầu tiên,
nhưng phần tử |B| được tô màu nhạt. Ở |overlay| thứ hai, phần tử |B|
sẽ được tô màu đậm, và ở |overlay| thứ ba, |C| sẽ được tô đậm.

% =====================================================================

%\subsection{The \cs{item} command}
\subsection{\texorpdfstring{Lệnh}{Lenh} \cs{item}}

\DescribeMacro{\item}
%% The |\item| command has an extra \emph{optional} argument in
%% \pf{powerdot} which allows for creating overlays in a more flexible
%% way then |\pause| provides.
Lệnh |\item| để tạo phần tử cho danh sách có thể nhận tham số bổ sung (\emph{optional})
cho phép tạo các |overlay| linh hoạt hơn lệnh |\pause|.
\begin{command}
 `\cs{item}\oarg{label}\larg{overlays}'
\end{command}
%% This optional argument should contain an overlay specification
%% stating on which overlays you want the item to appear. This
%% specification is a comma separated list where each item can used the
%% notation as in table~\ref{tab:item}.
Tham số bổ sung \meta{overlays} này là danh sách các |overlay|
mà phần tử sẽ xuất hiện (có nghĩa, một phần tử
có thể xuất hiện ở một hay nhiều |overlay| được chỉ ra). Trong
danh sách này, các |overlay| được cho bởi số, cách nhau bởi dấu phảy.
\begin{table}[htb]\centering
\begin{tabular}{d}
Cú pháp&Ý nghĩa\\\hline
\texttt{x}&Chỉ ở |overlay| \texttt{x}\\
\texttt{-x}&Các |overlay| nhỏ hơn hoặc bằng \texttt{x}\\
\texttt{x-}&Các |overlay| lớn hơn hoặc bằng \texttt{x}\\
\texttt{x-y}&Các |overlay| từ \texttt{x} đến \texttt{y},
bao gồm cả \texttt{x} và \texttt{y}\\
\end{tabular}
%\caption{\cs{item} and \cs{onslide} notation}
\caption{Quy ước cho \cs{item} và \cs{onslide}}
\label{tab:item}
\end{table}
%% The \meta{label} argument is the standard optional argument for
%% |\item| in \LaTeX. A \LaTeX\ manual \cite{companion} can tell you
%% more about this argument.
Tham số \meta{label} được hiểu tương tự như trong \LaTeX{} chuẩn,
xem chi tiết ở chẳng hạn \cite{companion}.

%Here is an example.
Dưới đây là ví dụ
\begin{example}
 \begin{slide}{Active itemize}
   \begin{itemize}[type=1]
    \item<1> A
    \item<2> B
    \item<3> C
   \end{itemize}
 \end{slide}
\end{example}
%% Here we have said that |A| should only be active on overlay 1, |B|
%% should only be active on overlay 2, and |C| should only be active on
%% overlay 3. Again, when the item is not active, it appears in the
%% inactive color because of |type=1|.
Với ví dụ này, phần tử |A| chỉ xuất hiện với màu tô đậm ở |overlay 1|,
phần tử |B| chỉ xuất hiện và được tô đậm ở |overlay| thứ hai,... Để ý rằng,
do chỉ định |type=1|, nếu phần tử không phải là phần tử hiện tại,
màu của nó được tô nhạt. Ta có thể minh họa như sau (chữ THƯỜNG chỉ màu nhạt):
\begin{example}
  1: --A---- 
  2: --aB--- 
  3: --abC--
\end{example}

%% If |type=0| is specified and if each item is given an overlay
%% option, then each item will appear only when it is active. When the
%% item is not active, then it will not show on the slide at all. More
%% examples demonstrating the syntax for \meta{overlays} will be
%% discussed in the next section.
Nếu chỉ định |type=0| và nếu mỗi phần tử của danh sách đều có thêm
tham số bổ sung chỉ định |overlay|, thì một phần tử sẽ bị giấu
hoàn toàn chứ không phải được tô màu nhạt. Ví dụ:
\begin{example}
  1: --A---- 
  2: ---B--- 
  3: ----C--
\end{example}

Ví dụ nhiều hơn về cú pháp của \meta{overlays} được cho ở Mục tiếp theo.

% =====================================================================

%\subsection{The \cs{onslide} command}
\subsection{\texorpdfstring{Lệnh}{Lenh} \cs{onslide}}
\label{sec:onslide}

\DescribeMacro{\onslide}
%Overlays can also be achieved using the |\onslide| command.
Các |overlay| có thể thu được nhờ lệnh |\onslide|.
\begin{command}
 `\cs{onslide}\marg{overlays}\marg{text}'
\end{command}
%% This command takes an \meta{overlays} specification as first
%% argument and the \meta{text} to apply it to as second argument. The
%% \meta{overlays} on which the text will appear are specified as a
%% comma separated list with syntax as in table~\ref{tab:item}. We
%% start off with a simple example.
Ở lệnh này, tham số \meta{overlays} ở vị trí thứ nhất;
tham số \meta{text} ở vị trí thứ hai chỉ nội dung cần biểu diễn.
Tham số \meta{overlays} là danh sách các |overlay|,
với cú pháp được nêu ở Bảng~\vref{tab:item}. Xét ví dụ sau đây.
\begin{example}
 \begin{slide}{Simple onslide}
   \onslide{1,2}{power}\onslide{2}{dot}
 \end{slide}
\end{example}
%% We have instructed |power| to appear on overlays one and two, and
%% |dot| to appear only on overlay two. As you might guess, this
%% example has the same output as our first |\pause| example. Yet, it
%% is clearly the case that our syntax is more complicated. However,
%% this slight ``complication'' also allows for much more flexibility.
Ở đây, chúng ta muốn |power| xuất hiện ở |overlay| thứ nhất và hai,
và |dot| chỉ xuất hiện ở |overlay| thứ hai. Bạn có thể đoán được rằng,
kết quả của ví dụ trên giống như khi dùng lệnh |\pause| ở ví dụ đầu tiên.
Bạn có thể ngạc nhiên là, nếu vậy thì việc đưa ra lệnh |\oneslide|
có vẻ rắc rối. Thực tế thì điều này có dụng ý, giúp cho bạn linh hoạt
hơn trong quá trình tạo |overlay|.

\DescribeMacro{\onslide+}
%% Consider the above example with the following modifications:
Ta hãy xét ví dụ trên, nhưng với một chút thay đổi như sau:
\begin{example}
 \begin{slide}{Simple onslide+}
   \onslide+{1}{power}\onslide+{2}{dot}
 \end{slide}
\end{example}
%% The |\onslide+| command displays its content in a different manner
%% altogether. Now, |dot| appears on every overlay, but it is in
%% inactive color\index{inactive color} and matches the normal font
%% color \textit{only} on overlay two. This is comparable to the
%% |type=1| behavior for lists (see section~\ref{sec:lists}).
%Lệnh |\onslide+| cho phép biểu diễn nội dung của nó nhiều cách khác nhau.
Bây giờ, cụm từ |dot| xuất hiện ở mọi |overlay|, nhưng nó
có màu tô nhạt,\index{tô màu nhạt} và được tô màu bình thường
chỉ khi nó xuất hiện ở |overlay| thứ hai. Điều này có nét tương
tự như khi dùng |type=1| đối với danh sách (xem Mục~\vref{sec:lists}).
Để ý rằng, ở đây ta nói đến các |overlay| tổng quát, không chỉ đơn
giản là danh sách. Ta minh họa kết quả trên như sau đây, chữ HOA
chỉ rằng chữ được tô màu bình thường, còn chữ THƯỜNG chỉ màu nhạt
\begin{example}
  1: POWERdot
  2: powerDOT
\end{example}
%% When executing this example, we will also notice that the |\onslide|
%% command does hide material, but still reserves the right amount of
%% space for it: on overlay 2, the |dot|s appear right above each
%% other. The next command does not reserve space.
Ta để ý rằng, mặc dù thực hiện 'giấu' (|\oneslide|)
hoặc tô màu nhạt (|\oneslide+|) cho nội dung \meta{text}, nó vẫn giữ một khoảng
trắng cho phần nội dung đó -- hệt như khi nội dung đó xuất hiện.
Lệnh |\onslide*| trong được mô tả sau đây sẽ không làm như vậy.

\DescribeMacro{\onslide*}
%% Instead of hiding and reserving space
%% (|\onslide|) or putting \meta{text} in the inactive color
%% (|\onslide+|) when the overlay doesn't match \meta{overlays}, this
%% command just eats the material altogether. To understand the
%% differences, consider the following example:
Thay vì 'giấu' (hoặc tô màu nhạt) cho phần nội dung \meta{text},
nếu nội dung \meta{text} không được thiết lập để hiển thị ở \meta{overlay},
lệnh này bỏ qua hoàn toàn nội dung đó. Để rõ hơn, ta hãy xét ví dụ sau đây:
\begin{example}
 \begin{slide}{Simple onslide*}
   \onslide{1}{power}\onslide{2}{dot}\\
   \onslide+{1}{power}\onslide+{2}{dot}\\
   \onslide*{1}{power}\onslide*{2}{dot}
 \end{slide}
\end{example}
%% The output of the first two lines, we are already familiar with. The
%% third line displays |power| on overlay 1 and |dot| on overlay 2, but
%% no space for |power| is reserved on overlay 2. Hence |dot| will
%% start on the cursor position that |power| started on overlay 1 and
%% it is not aligned below the other two |dots|.
Kết quả của hai dòng đầu tiên chúng ta đã quen thuộc. Hãy xét dòng thứ ba.
Chữ |power| xuất hiện ở |overlay| thứ nhất, còn |dot| ở |overlay| thứ hai.
Tuy nhiên, không có khoảng trắng nào được dành cho |power| ở |overlay| thứ
hai. Do đó, |dot| sẽ xuất hiện ở vị trí mà |power| xuất hiện. Kết quả có
thể minh họa như sau (chữ HOA: tô màu thường, chữ THƯỜNG: tô màu nhạt;
mỗi chữ x tượng trưng cho một vị trí bỏ trống).
\begin{example}
  1: POWERxxx   2: xxxxxDOT
  1: POWERdot   2: powerDOT
  1: POWERxxx   2: DOTxxxxx
\end{example}

%% We finish with an example of the syntax that is possible with
%% |\item| and |\onslide|. Remember that these commands take a comma
%% separated list for the \meta{overlays} specification and that each
%% element can used the syntax as explained in table~\ref{tab:item}.
%% The various variations are demonstrated in the example below.
Ta kết thúc mục này với ví dụ phức tạp. Nhớ rằng những lệnh
|\item| và |\onslide| nhận tham số \meta{overlays} là danh sách
các |overlay| theo cú pháp ở Bản~\vref{tab:item}.
\begin{example}
 \begin{slide}{Lists}
   \onslide{10}{chi o overlay 10}\par
   \onslide{-5}{cac onverlay <= 5}\par
   \onslide{5-}{cac overlay >= 5}\par
   \onslide{2-5}{cac overlay 2,3,4,5}\par
   \onslide{-3,5-7,9-}{overlay >= 1, tru 4 va 8}
 \end{slide}
\end{example}

% =====================================================================

%\subsection{Relative overlays}
\subsection{Overlay \texorpdfstring{tương đối}{tuong doi}}

%% Sometimes it is a pain to keep track of when an item should appear
%% or become active. You might, for example, just care that some text
%% appears on the overlay \textit{after} some other item. This
%% functionality is provided through the use of relative overlays which
%% should not be used outside list environments that use |\item|. Let's
%% consider a simple, illuminating example.
Thật phiền phải nhớ thứ tự các |overlay| để biểu diễn đúng nội dung nào
trước, nội dung nội sau. 
Trong thực tế, ta chỉ cần nhớ, chẳng hạn chữ |power| sẽ xuất hiện
trước chữ |dot| là đủ. Điều này có thể thực hiện nhờ |overlay| tương đối
và môi trường tạo danh sách. Hãy xem ví dụ đơn giản sau đây

\begin{example}
 \begin{slide}{Relative overlays}
   \begin{itemize}
     \item A \pause
     \item B \onslide{+1}{(visible 1 overlay after B)}\pause
     \item C \onslide{+2-}{(appears 2 overlays after C, visible until the end)}
     \pause
     \item D \onslide{+1-6}{(appears 1 overlay after D, visible until overlay 6)}
     \pause
     \item E \pause
     \item F \pause
     \item G \onslide{+1-+3}{(appears 1 overlay after G for 3 overlays)}\pause
     \item H \pause
     \item I \pause
     \item J \pause
     \item K
   \end{itemize}
 \end{slide}
\end{example}
%% As you can see, we still use |\onslide|. The only change is with the
%% syntax of the list of overlays. Now, we can specify a `|+|' symbol
%% in the list. In its simplest usage, |\onslide{+1}| will make text
%% display one overlay after the overlay it would \textit{normally}
%% appear on. You can still use the syntax in table~\ref{tab:item}.
%% These are demonstrated in the above example. Notice,
%% |\onslide{+1-6}| means that the text will appear one overlay after
%% the overlay it would normally appear on and that the text should
%% remain shown until overlay seven. To make text appear for a range of
%% relative overlays, see the final demonstration in the above example.
Ở đây, ta vẫn dùng |\oneslide|, nhưng với cú pháp mới của cho các |overlay|.
Đó là ta có thể sử dụng dấu `|+|' trong danh sách. Ở ví dụ đơn giản nhất,
|\onslide{+1}| sẽ thể hiện nội dung ở một |\overlay| kế tiếp. Ta vẫn có thể
dùng cú pháp đã nêu ở Bảng~\vref{tab:item}. Ví dụ, |\onslide{+1-6}{power}|
sẽ thể hiện chữ |power| ở các |overlay| kế tiếp cho đến |overlay| \underline{mang số 6}.
Ở ví dụ cuối cùng, |\onslide{+1-+3}{power}|, chữ |power| sẽ xuất hiện ở |overlay|
kế tiếp và sẽ tiếp tục xuất hiện thêm ở 3 |overlay| nữa. Ta minh họa kết quả
trên như sau đây (dấu |*| đại diện cho tham số của các lệnh |\onslide| ở trên:
\begin{example}
  1: A
  2: A_B
  3: A_B*_C
  4: A_B__C__D
  5: A_B__C*_D*_E
  6: A_B__C*_D*_E_F
  7: A_B__C*_D__E_F_G_
  8: A_B__C*_D__E_F_G*_H
  9: A_B__C*_D__E_F_G*_H_I
 10: A_B__C*_D__E_F_G*_H_I_J
 11: A_B__C*_D__E_F_G__H_I_J_K
\end{example}

% =====================================================================

%\section{Presentation structure}
\section{\texorpdfstring{Cấu trúc trình diễn}{Cau truc trinh dien}}

\label{sec:structure}

% =====================================================================

%\subsection{Making sections}
\subsection{\texorpdfstring{Tạo mục}{Tao muc}}
\label{sec:section}

\DescribeMacro{\section}
%% This section describes the |\section| command which provides a way
%% to structure a presentation.
Lệnh |\section| cho phép tạo một mục mới trong trình diễn, tương tự
như cách dùng lệnh |\section| của lớp \pf{article}.
\begin{command}
 `\cs{section}\oarg{options}\marg{section title}'
\end{command}
%% This command will produce a slide with \meta{section title} on it
%% and will also use this text to create sections in the table of
%% contents and in the bookmarks list. There are several \meta{options}
%% to control its output.
Lệnh này sẽ tạo một |slide| chỉ với nội dung là (tựa đề) \meta{section title}
(hãy xem thêm về tuỳ chọn |slide| ở bên dưới).
Tựa đề \meta{section title} cũng sẽ xuất hiện ở Bảng Mục lục và danh sách |bookmark|.
Có vài tuỳ chọn cho lệnh này như sau:

\DescribeOption{tocsection}
%% This option controls the creation of a
%% section in the table of contents. The default value is |true|.
Tuỳ chọn điều khiển việc tạo phần tử tương ứng cho |section| ở Bảng Mục lục.
Giá trị mặc định |true|.
\begin{description}
\item\option{tocsection=true}\\
%% This does create a section in the table of contents. This means that
%% all following slides, until the next section, will be nested under
%% this section.
Tạo mục tương ứng với |section| hiện tại trong Bảng Mục lục.
\item\option{tocsection=false}\\
%% This does not create a section in the table of contents and hence
%% the section will be listed as an ordinary slide.
Chỉ tạo |slide| cho mục nhưng không tạo phần tử tương ứng trong trang Mục lục.
\item\option{tocsection=hidden}\\
%% This does create a section in the table of contents, but this is
%% only visible when you view a slide that is part of this section.
%% This could be used to append a section to the presentation which you
%% can discuss if there is some extra time.
Tạo mục tương ứng trong Bảng Mục lục, nhưng mục này chỉ nhìn thấy khi bạn
đang xem một trong các |slide| của Mục đang xét. Điều này có ích, chẳng hạn khi
bạn có thêm thời gian để trình bày mục này (một cách dự trữ!).
\end{description}

\DescribeOption{slide}
%% This option controls whether the |\section|
%% command creates a slide. The default value is |true|.
Tuỳ chọn này cho phép hay không lệnh |\section|
tạo riêng cho nó một |slide|. Mặc định là |true|.
\begin{description}
\item\option{slide=true}\\
% A slide is created.
Một |slide| được tạo ra với nội dung là tựa của mục.
\item\option{slide=false}\\
%% No slide will be created. If also |tocsection| is |false|, the
%% |\section| command doesn't do anything. If it does create a table of
%% contents section (|tocsection=| |true| or |hidden|), its link will
%% point to the first slide in the section as the section itself
%% doesn't have a slide.
Không |slide| nào được tạo ra khi gặp lệnh |\section|.
Nếu đồng thời |tocsection| nhận giá trị |false|,
thì lệnh |\section| không làm gì cả. Nếu |tocsection| nhận giá trị |true|
hoặc |hidden|, thì chọn mục tương ứng trong Bảng Mục lục, ta sẽ nhày đến
|slide| đầu tiên của mục (bởi không có |slide| riêng cho Mục).
\end{description}

\DescribeOption{template}
%% This option can be used to make the
%% section slide with another template. By default, a normal |slide|
%% environment is used to create the section slide, but if a style
%% offers other templates that could be used for this purpose (for
%% instance, the |wideslide| environment), then you can use this option
%% to select that template. See section~\ref{sec:styles} for an
%% overview of the available templates with every style.
Tuỳ chọn này cho phép Mục đang xét chọn một mẫu khác.
Theo mặc định, môi trường |slide| được dùng để tạo ra |slide| cho mục,
nhưng nếu bạn muốn mẫu khác được dùng cho mục đích này,
ví dụ |wideslide|, thì bạn có thể dùng tuỳ chọn này để chỉ định mẫu đó.
Xem thêm ở Mục~\vref{sec:styles} để biết thêm chi tiết về các mẫu
của các kiểu khác nhau.

%% Finally, all options available to normal slides are available to
%% slides created by |\section| as well (see section~\ref{sec:slides}).
%% However, when the section does make a |tocsection|, |toc=| or |bm=|
%% won't remove the table of contents entry or the bookmark
%% respectively.

Cuối cùng, mọi tuỳ chọn của |slide| đều có thể dùng cho |\section|,
ví dụ |toc=|, |bm=|,\ldots Xem thêm ở Mục~\vref{sec:slides}.

% =====================================================================

%\subsection{Making an overview}
\subsection{\texorpdfstring{Tạo slide Mục lục}{Tao slide Muc luc}}
\label{sec:tableofcontents}

\DescribeMacro{\tableofcontents}
%% This command creates an overview of your presentation and can only
%% be used on a slide.
Lệnh này tạo |slide| Mục lục giúp bạn có cái nhìn tổng quan về trình diễn
hoặc một phần (mục) của trình diễn.
Lệnh  này chỉ được dùng bên trong môi trường tạo |slide|,
và bạn có thể gọi nó bao nhiêu lần tuỳ thích.

Nội dung được nói đến trong mục này là nội dung của |slide| Mục lục.
Để ý rằng, |slide| Mục lục khác với Bảng Mục lục. Một trình diễn chỉ
có một Bảng Mục lục nhưng có thể có nhiều |slide| mục lục; hơn nữa,
thường thì Bảng Mục lục được tự động tạo ra.

\begin{command}
 `\cs{tableofcontents}\oarg{options}'
\end{command}
%% There are several \meta{options} to control the output of this
%% command.
Có vài tuỳ chọn cho lệnh như sau:

\DescribeOption{type}
%% This option controls whether certain material (depending on the
%% input in the |content| option below) will be hidden or displayed in
%% the inactive color\index{inactive color}. The default value is |0|.
%% Compare with the |type| option for list environments
%% (section~\ref{sec:lists}).
Xác định giấu hoặc tô nhạt một vài phần (phụ thuộc vào giá trị của |content| dưói đây).
Giá trị mặc định là |0|. So sánh với tuỳ chọn |type| của môi trường
tạo danh sách ở Mục~\vref{sec:lists}.

\begin{description}
\item\option{type=0}\\
%% When material is not of the requested type as specified in the
%% |content| option, it will be hidden.
Nếu nội dung không đúng kiểu như chỉ ra ở tuỳ chọn |content|, nó sẽ được
giấu đi.
\item\option{type=1}\\
%% As |type=0|, but instead of hiding material, it will be typeset in
%% the inactive color.
Như trên, nhưng thay vì giấu đi, nội dung sẽ được hiện với màu tô nhạt.
\end{description}

\DescribeOption{content}
%% The |content| option controls which elements will be included in the
%% overview. The default value is |all|. The description below assumes
%% that |type=0| was chosen, but the alternative text for |type=1| can
%% easily be deduced.
Tuỳ chọn này cho phép xác định những phần tử nào sẽ được thể hiện ở
|slide| mục lục. Giá trị mặc định là |all|. Mô tả dưới đây giả định rằng
|type=0| được chọn. Bạn có thể dễ dàng suy ra kết quả khi |type=1| từ mô tả này.

\begin{description}
\item\option{content=all}\\
%% This will display a full overview of your presentation including all
%% sections and slides, except the slides in hidden sections (see
%% section~\ref{sec:section}).
Cho ra |slide| đầy đủ, gồm mọi mục và |slide| trong trình diễn
của bạn, trừ các mục ẩn (xem mô tả trong Mục~\vref{sec:section}). 
\item\option{content=sections}\\
%This displays only the sections in the presentation.
Chỉ liệt kê các Mục của trình diễn.
\item\option{content=currentsection}\\
%This displays the current section only.
Chỉ liệt kê các |slide| của mục hiện tại.
\item\option{content=future}\\
%This displays all content starting from the current slide.
Liệt kê mọi mục và |slide| bắt đầu từ |slide| hiện tại.
\item\option{content=futuresections}\\
%This displays all sections, starting from the current section.
Liệt kê mọi mục bắt đầu từ mục hiện tại.
\end{description}

%% We finish this section with a small example that will demonstrate
%% how you can make a presentation that contains an overall overview of
%% sections in the presentation, giving a general idea of the content,
%% and per section a detailed overview of the slides in that section.
Dưới đây là ví dụ nhỏ. Trình diễn ở ví dụ này gồm các mục,
dầu mỗi mục là |slide| mục lục có nhiệm vụ tóm tắt (liệt kê) các
phần của Mục đó.
\begin{example}
 \begin{slide}[toc=,bm=]{Overview}
   \tableofcontents[content=sections]
 \end{slide}
 \section{First section}
 \begin{slide}[toc=,bm=]{Overview of the first section}
   \tableofcontents[content=currentsection,type=1]
 \end{slide}
 \begin{slide}{Some slide}
 \end{slide}
 \section{Second section}
 ...
\end{example}

% =====================================================================

%\section{Miscellaneous}
\section{Linh tinh}

% =====================================================================

%\subsection{The \cs{twocolumn} command}
\subsection{\texorpdfstring{Lệnh}{Lenh} \cs{twocolumn} \texorpdfstring{chia cột}{chia cot}}

\label{sec:twocolumn}


\DescribeMacro{\twocolumn}
%The |\twocolumn| macro allows to split content into two columns.
Lệnh |\twocolumn| cho phép bố trí nội dung ở hai cột của trang.
\begin{command}
 `\cs{twocolumn}\marg{options}\marg{left}\marg{right}'
\end{command}
%% This typesets \meta{left} and \meta{right} in two columns. The
%% dimensions of those columns can be controlled by \meta{options}.
%% Below are the available options.
Lệnh này sẽ bố trí \meta{left} và \meta{right} vào hai cột bên trái
và bên phải của |slide|. Kích thước của các cột được cho ở tuỳ chọn
\meta{options}.
\begin{description}
\item\DescribeOption{lineheight}\option{lineheight}\\
%% If |lineheight| is specified, a line of the specified height will be
%% created using |\psline| in between the two columns. Example:
%% |lineheight=6cm|.
Nếu |lineheight| được chỉ ra, một dòng kẻ (tạo ra nhờ lệnh |\psline|)
với chiều cao chỉ định được chèn để phân cách hai cột. Ví dụ |lineheight=6cm|.
\item\DescribeOption{lineprop}\option{lineprop}\\
%Any \pf{pstricks} declaration to specify the line properties. Example:
Các khai báo \pf{pstricks} để chỉ thuộc tính của dòng kẻ. Ví dụ
\begin{example}
 lineprop={linestyle=dotted,linewidth=3pt}
\end{example}
\item\DescribeOption{lfrheight}\option{lfrheight}\\
%% Creates a frame of the specified height around the left column.
Tạo ra một khung (|frame|) với chiều cao chỉ định xung quanh cột bên trái.
\item\DescribeOption{lfrprop}\option{lfrprop}\\
%As |lineprop|, but for the left frame.
Như |lineprop|, nhưng cho đường kẻ của khung bên trái.
\item\DescribeOption{rfrheight}\option{rfrheight}\\
%% Creates a frame of the specified height around the right column.
Tạo khung bên phải với chiều cao chỉ định.
\item\DescribeOption{rfrprop}\option{rfrprop}\\
% As |lineprop|, but for the left frame.
Như |lineprop| nhưng dành cho đường kẻ của khung phải.
\item\DescribeOption{frsep}\option{frsep}\\
% Space between text and the frames. Default: |1.5mm|.
Khoảng cách giữa nội dung và khung (|frame|). Mặc định |1.5mm|.
\item\DescribeOption{colsep}\option{colsep}\\
% Space between the two columns. Default: |0.06\linewidth|.
Khoảng cách giữa hai cột. Mặc định |0.06\linewidth|.
\item\DescribeOption{lcolwidth}\option{lcolwidth}\\
%Width of the left column. Default: |0.47\linewidth|.
Chiều rộng của khung bên trái. Mặc định |0.47\linewidth|.
\item\DescribeOption{rcolwidth}\option{rcolwidth}\\
%Width of the right column. Default: |0.47\linewidth|.
Chiều rộng của khung bên phải. Mặc định |0.47\linewidth|.
\item\DescribeOption{topsep}\option{topsep}\\
%% The extra space (additional to |\baselineskip|) between text above
%% the columns and the text within the columns. Default: |0cm|.
Khoảng trắng thêm vào bên trên cột và bên trên các dòng nội dung
(cộng thêm vào giá trị đã có của |\baselineskip|). Mặc định |0cm|.
\item\DescribeOption{bottomsep}\option{bottomsep}\\
%Idem for the bottom of the columns. Default: |0cm|.
Nhưng trên, nhưng cho bên dưới cột và dòng. Mặc định |0cm|.
\item\DescribeOption{indent}\option{indent}\\
%Horizontal indent left to the left column. Default: |0cm|.
Khoảng cách thụt đầu dòng so với cột bên trái.\footnote{Horizontal indent left to the left column}
Mặc định |0cm|.
\end{description}
%% The dimensions described above are represented graphically in
%% figure~\ref{fig:twocolumndim}.
Mô tả hình học của các tuỳ chọn trên được cho ở Hình~\vref{fig:twocolumndim}.
\begin{figure}[htb]
\centering
\begin{pspicture}(0,.5)(13,10.5)
\psline(0,0.5)(0,10)
\rput[tl](.05,9.95){Top}
\psframe[dimen=middle](1,9)(7,2)
\psline{C-C}(8.5,9)(11,9)
\psline{C-C}(8.5,2)(8.5,9)
\psline{C-C}(8.5,2)(11,2)
\qdisk(1.7,8.3){.1cm}
\psset{linestyle=dashed}
\psline{C-C}(1.7,8.3)(6.3,8.3)
\psline{C-C}(1.7,8.3)(1.7,3)
\psline{C-C}(6.3,5)(6.3,8.3)
\psline{C-C}(11,9)(12,9)
\psline{C-C}(11,2)(12,2)
\psline{C-C}(11,7)(12,7)
\psline{C-C}(9.2,8.3)(12,8.3)
\psline{C-C}(9.2,8.3)(9.2,3)
\rput[tl](1.75,8.25){Nội dung bên trái}
\rput[tl](9.25,8.25){Nội dung bên phải}
\rput[tl](.05,1){Bottom}
\psset{linestyle=dotted,dotsep=2pt}
\psline(0,8.3)(1.7,8.3)
\psline(0,9.6)(1,9.6)
\psline(0,2)(1,2)
\psline(0,1.1)(1,1.1)
\psset{linestyle=solid}
\psline{<->}(.2,8.33)(.2,9.57)
\psline{<->}(4,8.33)(4,8.97)
\psline{<->}(1.73,7)(6.27,7)
\psline{<->}(1.03,6.5)(1.67,6.5)
\psline{<->}(0.03,5.5)(1.67,5.5)
\psline{<->}(6.33,7.4)(9.17,7.4)
\psline{<->}(8.53,6.5)(9.17,6.5)
\psline{<->}(6.33,6.5)(6.97,6.5)
\psline{<->}(10.7,8.33)(10.7,8.97)
\psline{<->}(7.3,8.97)(7.3,2.03)
\psline{<->}(.2,1.13)(.2,1.97)
\psline{->}(1.7,9.3)(1.7,8.45)
\psline{<-}(9.23,7)(11,7)
\cput(4,6.6){\small 1}
\cput(11.1,6.6){\small 2}
\cput(8,7){\small 3}
\cput(7.7,3){\small 4}
\cput(4.4,8.65){\small 5}
\cput(1.35,6.1){\small 5}
\cput(8.85,6.1){\small 5}
\cput(11.1,8.65){\small 5}
\cput(6.65,6.1){\small 5}
\cput(0.6,8.95){\small 6}
\cput(0.6,5.1){\small 7}
\cput(0.6,1.55){\small 8}
\cput(1.7,9.6){\small 9}
\end{pspicture}
\begin{tabular}{c p{4cm}cl}
\multicolumn{4}{c}{Ý nghĩa của nhãn}\\\hline
1&|lcolwidth|&5&|frsep|\\
2&|rcolwidth|&6&|topsep|\\
3&|colsep|&7&|indent|\\
4&|lfrheight|, |rfrheight|,&8&|bottomsep|\\
&|lineheight|&9&Điểm tham khảo
\end{tabular}
\caption{Các kích thước ở chế độ  hai cột.}\label{fig:twocolumndim}
\end{figure}
%% Important to notice is that the |\twocolumn| macro uses the current
%% cursor position as the reference point to position the first line of
%% text of the left column (see also figure~\ref{fig:twocolumndim}). This
%% means that optional frames can extend to the text on the previous
%% line. Use for instance |topsep=0.3cm| in that case to add extra
%% space between the two lines of text. The default value of |topsep|
%% is based on the situation that there is no text on top of the two
%% columns. In that case, it is best to locate the first line of text
%% of the left column at the same spot as text that is not created by
%% |\twocolumn| on other slides. The setting |topsep=0cm| does exactly
%% this. However, with a combination of |topsep| and |indent| you can
%% change this behavior and position the first line of text of the left
%% column anywhere you want.
Điều chú ý quan trọng là, lệnh |\twocolumn| dùng vị trí hiện tại của
con trỏ như  là điểm tham khảo đến dòng đầu tiên của nội dung ở cột
bên trái (xem ở Bảng~\vref{fig:twocolumndim}). Điều này có nghĩa rằng,
khung được tạo ra sẽ ảnh hưởng đến dòng trước đó. Hãy dùng chẳng hạn |topsep=0.3cm|
để thêm khoảng trắng thích hợp vào giữa hai dòng này. Giá trị mặc định của |topsep|
là |0cm| được lấy với giả định rằng không có nội dung nào
trước khi bắt đầu chế độ hai cột. Trong trường hợp này, dòng đầu tiên
của cột bên trái có vị trí như các dòng (đầu tiên) tạo ở các |slide| khác.
Việc cài đặt |topsep=0cm| bảo đảm điều này. Tuy nhiên, việc phối hợp
hai giá trị |topsep| và |indent| cho phép bạn định vị dòng đầu tiên của
cột bên trái theo ý bạn.

%% The |\twocolumn| macro computes the height of the construction to
%% position text below the construction correctly. The computation is
%% done by taking the maximum height of |lfrheight|, |rfrheight|,
%% |lineheight| (if requested) and the left and right column content.
%% Hence when frames nor a line is requested, |bottomsep| is the
%% vertical space between the lowest line of text in the columns and
%% the text below the columns (additional to |\baselineskip|). Here is
%% an example.
Lệnh |\twoclumn| tính toán chiều cao của cột dựa trên nội dung của nó,
nhờ đó mới có thể bố trí đúng nội dung vào các cột. Việc tính toán này
đưọc tinh chỉnh nhờ lấy giá trị lớn nhất trong hai giá trị |lfrheight| và |rfrheight|.
\begin{example}
 \begin{slide}{Two columns}
   Here are two columns.
   \twocolumn{lfrprop={linestyle=dotted,linewidth=3pt},
     lfrheight=4cm,rfrheight=5cm,lineheight=3cm,topsep=0.3cm
   }{left}{right}
   That were two columns.
 \end{slide}
\end{example}

% =====================================================================

\subsection{\texorpdfstring{Ghi chú}{Ghi chu}}\label{sec:notes}

\DescribeEnv{note}
%% The |note| environment can be used to make personal notes that
%% accompany a slide. You can control displaying notes using the
%% |display| option (see section~\ref{sec:classopts}). Here is an
%% example.
Môi trường |note| dùng để có các ghi chú (cá nhân) đi cùng với trình diễn.
Có thể điều làm xuất hiện các ghi chú này nhờ tuỳ chọn |display|
đã nói ở Mục~\ref{sec:classopts}. Dưới đây là ví dụ
\begin{example}
 \begin{slide}{Chewing gum}
 ...
 \end{slide}
 \begin{note}{Reminder for chewing gum}
   Don't forget to mention that chewing gum is sticky.
 \end{note}
\end{example}

% =====================================================================

\subsection{Slide \texorpdfstring{trắng}{trang}}

\label{sec:emptyslides}

\DescribeEnv{emptyslide}
%% The |emptyslide| environment creates a totally empty slide. The text
%% box on the slide can be used for special things like displaying
%% photos. This allows for creating a dia show. Example:
Môi trường |emptyslide| tạo ra một |slide| trắng (không có nội dung gì).
Nhờ đó, bạn có thể nạp vào hình vẽ chẳng hạn. Nên nhớ rằng, nếu
không dùng |slide| trắng, bạn phải chỉ ra tựa đề của |slide|
và điều này sẽ ảnh hưởng đến Bảng Mục lục cũng như danh sách |bookmakr|.
Ví dụ
\begin{example}
 \begin{emptyslide}{}
   \centering
   \vspace{\stretch{1}}
   \includegraphics[height=0.8\slideheight]{me_chewing_gum.eps}
   \vspace{\stretch{1}}
 \end{emptyslide}
\end{example}
%% The |\includegraphics| command is defined by the \pf{graphicx}
%% package \cite{graphics}. The |\stretch| command is used to
%% vertically center the picture. Both commands are described in your
%% favorite \LaTeX\ manual, for instance \cite{companion}. Note that
%% you can use the lengths |\slideheight| and |\slidewidth| to scale
%% pictures to fit nicely on the slide.
Lệnh |\includegraphics| được lấy từ gói \pf{graphicx} \cite{graphics}.
Lệnh |\stretch| dùng để canh hình vẽ theo chiều đứng. Cả hai lệnh này
đưọc mô tả trong \cite{companion}. Chú ý rằng, các biến độ dài
|\slideheight| và |slidewidth| được dùng để bố trí hình vẽ
vừa vẹn lên |slide|.

% =====================================================================

\subsection{Slide \texorpdfstring{tài liệu tham khảo}{tai lieu tham khao}}

\label{sec:bib}

\DescribeEnv{thebibliography}
\pf{powerdot}
%% redefines the standard \pf{article}
%% |thebibliography| environment to suppress the creation of a section
%% heading and running headers. All other properties are maintained.
%% You can do either of the next two (depending whether you are
%% using BiB\TeX\ or not):\\
định nghĩa lại môi trường |thebibliography| của lớp chuẩn \pf{article}
để tạo trang tài liệu tham khảo. Sự khác biệt là môi trường mới không
tạo ra tựa đề và không tạo các dòng chữ ở đầu trang (|header|); các
tính chất khác đều được bảo toàn. Bạn có thể dùng một trong hai cách
sau đây (tuỳ thuộc bạn có dùng BiB\TeX{} hay không):\\
\begin{minipage}[t]{.49\linewidth}
\begin{example}
 \begin{slide}{Slide}
   \cite{someone}
  \end{slide}
 \begin{slide}{References}
   \begin{thebibliography}{1}
   \bibitem{someone} Article of someone.
   \end{thebibliography}
 \end{slide}
\end{example}
\end{minipage}\hfill
\begin{minipage}[t]{.49\linewidth}
\begin{example}
 \begin{slide}{Slide}
   \cite{someone}
 \end{slide}
 \begin{slide}{References}
   \bibliographystyle{plain}
   \bibliography{YourBib}
 \end{slide}
\end{example}
\end{minipage}

%% In case you have a big reference list that you want to spread over
%% multiple slides, have a look at the packages \pf{natbib} and
%% \pf{bibentry} \cite{natbib}. Using both packages allows you to do:
Trong trường hợp bạn có một danh sách rất dài các tài liệu tham khảo,
bạn có thể muốn chia danh sách đó thành nhiều |slide|, giống như cách làm
của gói \pf{natbib} và \pf{bibentry} \cite{natbib}. Việc dùng cả hai
gói đó sẽ làm bạn vừa ý:
\begin{example}
 \begin{slide}{References (1)}
   \bibliographystyle{plain}
   \nobibliography{YourBib}
   \bibentry{someone1}
   \bibentry{someone2}
 \end{slide}
 \begin{slide}{References (2)}
   \bibentry{someone3}
 \end{slide}
\end{example}
%% Have a look at your favorite \LaTeX\ manual for more information
%% about citations and bibliographies.
Nên tìm kiếm tài liệu về việc trích dẫn và các tạo danh sách
tài liệu tham khảo trong \cite{companion}.

% =====================================================================

%\section{Available styles}
\section{\texorpdfstring{Các kiểu trình diễn}{Cac kieu trinh dien}}
\label{sec:styles}

Lớp \pf{powerdot}
%% comes with a number of styles which are listed in the
%% overview below. The characteristics of each style are described
%% shortly.
được phân phối cùng với một số kiểu (|style|, xì-tin :) được mô tả
dưới đây. Bạn nên thử qua chúng để chọn lấy kiểu ưa thích. Việc lựa
chọn kiểu đã nói đến trong Mục~\vref{sec:classopts}.

\medskip
Có thể xem hình ảnh minh họa cho các kiểu này có trong tập tin
|powerdot-styles.zip| (tải về từ \url{http://www.viettug.org}).

\begin{description}
\item\pf{default}\\
%% This style has as main colors light blue and white. A flower in the
%% top left corner decorates the slide. The style supports a
%% |wideslide| and portrait orientation. Slides have a table of
%% contents on the left part of the paper in landscape orientation and
%% on the bottom part in portrait orientation. The style requires the
%% \pf{pifont} package.
Kiểu mặc định, chỉ với màu xanh sáng và trắng. Một bông hoa ở góc trên
bên trái của mỗi |slide|. Kiểu này cung cấp môi trường |wideslide| và hướng
|portrait|. Mỗi |slide| có Bảng Mục lục ở bên trái (chế dộ |landscape|)
hoặc bên dưới (chế độ |portrait|). Kiểu này đòi hỏi gói \pf{pifont}.
\item\pf{simple}\\
%% This is a simple style in black and white. The style supports a
%% |wideslide| and portrait orientation. A table of contents is present
%% on slides at the left hand side in landscape mode and in the bottom
%% of the slide in portrait mode. It requires the \pf{amssymb} and
%% \pf{pifont} packages.
Kiểu đơn giản chỉ với hai màu đen, trắng. Hỗ trợ môi trường |wideslide|
và hướng |portrait|. Bảng Mục lục được bố trí bên trái (|landscape|) hoặc
bên dưới (|portrait|) của |slide|. Kiều này cần gói \pf{amssymb}
và gói \pf{pifont}.
\item\pf{tycja}\\
%% This style is set in shades of yellow and dark blue. The style
%% supports a |wideslide| and portrait orientation. The table of
%% contents on slides is on the right side of the paper in landscape
%% orientation and on the bottom part in portrait. The style requires
%% the \pf{pst-grad} \cite{PSTricksWeb,PSTricks} and \pf{pifont}
%% packages.
Kiểu với màu vàng và xanh thẫm. Hỗ trợ môi trường |widestyle| và
hướng |portrait|. Bảng Mục lục bố trí bên phải (|landscape|)
hoặc bên dưới (|portrait|). Đòi hỏi gói \pf{pst-grad}
\cite{PSTricksWeb,PSTricks} và gói \pf{pifont}.
\item\pf{ikeda}\\
%% This style uses dark shades of red and blue and a light text color.
%% It has nice patterns on the slide for decoration. The style supports
%% a |wideslide| and portrait orientation. The table of contents is on
%% the left side of the paper in landscape orientation and on the
%% bottom part in portrait. The style requires the \pf{calc} and
%% \pf{pifont} packages.
Kiểu với màu đỏ, xanh (màu khung, bóng) và màu sáng (cho nội dung).
Kiểu này thiết kế các đường khung, viền khá đẹp. Hỗ trợ môi trường
|wideslide| và hướng |portrait|. Bảng mục lục bố trí bên trái (|landscape|)
hoặc bên dưới (|portrait|). Đòi hỏi gói \pf{calc} và gói \pf{pifont}.
\item\pf{fyma}\\
%% This style was originally created by Laurent Jacques for
%% \pf{prosper}. Based on that style, he created a version for
%% \pf{HA-prosper} with extended features. With his kind permission,
%% this style has been converted by Shun'ichi J. Amano for
%% \pf{powerdot}. The style has an elegant design with a light blue and
%% white gradient background. The style supports a |wideslide| and
%% portrait orientation. It has special templates for sections on
%% slides and sections on wide slides. The table of contents is on the
%% left side of the paper in landscape orientation and on the bottom
%% part in portrait. The style requires the \pf{pst-grad}
%% \cite{PSTricksWeb,PSTricks} package.
Nguyên gốc của kiểu này là của Laurent Jacques dành cho \pf{prosper}.
Dựa trên bản gốc đó, L. Jacques tạo ra bản cho \pf{HA-prosper} với vài
tính năng nổi trội. Được phép của L. Jacques, kiểu này được Shun'ichi J. Amano
chuyển thành phiên bản cho \pf{powerdot}. Kiểu đưọc thiết kế gọn đẹp,
với màu xanh sáng và màu nền |gradient|. Kiểu hỗ trợ môi trường |wideslide|
và hướng |portrait|; ngoài ra, còn cung cấp các mẫu đặc biệt để tạo
Mục trong trình diễn. Bảng Mục lục được bố trí bên trái (|landscape|)
hoặc bên dưới (|portrait|). Đòi hỏi các gói \pf{pst-grad} \cite{PSTricksWeb,PSTricks}.
\item\pf{ciment}\\
%% This style was originally created by Mathieu Goutelle for
%% \pf{prosper} and \pf{HA-prosper}. With his permission, this style
%% has been converted for \pf{powerdot}. The style has a background
%% that is hatched with light gray horizontal lines. Titles and table
%% of contents highlighting are done with dark red. The table of
%% contents is on the left side of the paper in landscape orientation
%% and on the bottom part in portrait. The style requires the
%% \pf{pifont} package.
Bản nguyên thuỷ của kiểu này là của Matheiu Goutelle dành cho \pf{prosper}
và \pf{HA-prosper}. Được phép của tác giả, bản đó được thay đổi
để dành cho \pf{powerdot}. Với kiểu này, màu nền |slide| được trang điểm
bằng nhiều đường kẻ ngang màu xám. Bảng mục lục đưọc bố trí bên trái (|landscape|)
hoặc bên phải (|portrait|). Đòi hỏi gói \pf{pifont}.
\item[]
Dưới đây là các kiểu được thêm vào từ \pf{powerdot} phiên bản 1.1.
\item\pf{elcolors}\\
%% This is a style using light shades of the elementary colors red,
%% blue and yellow.\\
Kiểu với các màu cơ bản: đỏ, xanh và vàng.
\item\pf{aggie}\\
%% This style was created by Jack Stalnaker for \pf{HA-prosper} and he
%% has converted this style for \pf{powerdot}. The style uses dark red
%% and light brown colors.\\
Kiểu tạo bởi Jack Stalnaker dành cho \pf{HA-prosper}. Tác giả đã
chuyển đổi thành kiểu cho \pf{powerdot}. Sử dụng các màu đỏ sẫm và nâu sáng.
\item\pf{husky}\\
%% This style is created by Jack Stalnaker and has a background of
%% light gray sun beams combined with dark red highlights.\\
Tác giả là Jack Stalnaker. Sử dụng nền mô phỏng chùm ánh sáng mặt trời.
Màu cho khung là đỏ sẫm.
\item\pf{sailor}\\
%% This style is contributed by Mael Hill\'ereau and uses rounded shapes
%% in dark blue and shades of light blue.\\
Đóng góp bởi Mael Hill\'ereau. Sử dụng các màu xanh sáng và xanh tối.
\end{description}

% =====================================================================

%\section{Compiling your presentation}
\section{\texorpdfstring{Biên dịch. Cài đặt}{Bien dich. Cai dat}. Xem}
\label{sec:compiling}

% =====================================================================

%\subsection{Dependencies}
\subsection{\texorpdfstring{Gói phụ thuộc}{Goi phu thuoc}}

\label{sec:dependencies}

%% In table~\ref{tab:dependencies} is a list of packages that
%% \pf{powerdot} uses to perform specific tasks. Dependencies of
%% packages in this table are not listed. Notice further that styles
%% may have extra requirements (see section~\ref{sec:styles}). In the
%% table, `required' means that you should have a version \emph{at
%% least} as new as listed and `tested' means that \pf{powerdot} was
%% tested with this version, but that it could equally well work with
%% an older or newer version than the one listed in the table. So, when
%% trying to solve an error, first concentrate on solving version
%% issues for the `required' packages. To find out which version of a
%% package you are currently using, put |\listfiles| on the first line
%% of your document, run it with \LaTeX, open the |.log| file and read
%% the file list (see a \LaTeX\ manual for more information). If you
%% need to update a package, you can get it from CTAN \cite{CTAN}.

Ở Bảng~\vref{tab:dependencies} là danh sách các gói đòi hỏi bởi \pf{powerdot}
(gói phụ thuộc của các gói trong bảng này không được liệt kê). Chú ý rằng
mỗi kiểu có những đòi hỏi riêng (xem Mục~\vref{sec:styles}). Trong bảng dưới đây,
các gói `|bắt buộc|' cho biết bạn không thể dùng phiên bản cũ hơn của gói đó;
trong khi các gói `|đã kiểm tra|' có nghĩa \pf{powerdot} đã được kiểm tra
sự tương thích với phiên bản đã chỉ ra, nhưng có thể vẫn gặp lỗi với các
phiên bản \underline{cũ hơn} hoặc \underline{mới hơn} của gói đó.
Để biết phiên bản của các gói được dùng, bạn hãy thêm lệnh |\listfiles|
vào đầu tài liệu, biên dịch bằng \LaTeX{} rồi mở tập tin |.log| để xem
kết quả. Để có bản cập nhật của các gói trên, bạn có thể tải chúng về
từ CTAN \cite{CTAN}.% hoặc VietTUG \cite{VietTUG}.

\begin{table}[htb]
\centering
\begin{tabular}{e}
Gói/tập tin & Phiên bản & Ngày & Mức độ\\\hline
\pf{xkeyval} \cite{xkeyval} & 2.5c & 2005/07/10 & bắt buộc\\
\texttt{pstricks.sty} \cite{PSTricksWeb,PSTricks} & 0.2l & 2004/05/12 & bắt buộc\\
\pf{xcolor} \cite{xcolor} & 1.11 & 2004/05/09 & bắt buộc\\
\pf{enumitem} \cite{enumitem} & 1.1 & 2005/05/12 & bắt buộc\\\hline
\pf{article} class & 1.4f & 2004/02/16 & đã kiểm tra\\
\pf{geometry} \cite{geometry} & 3.2 & 2002/07/08 & đã kiểm tra\\
\pf{hyperref} \cite{hyperref} & 6.74m & 2003/11/30 & đã kiểm tra\\
\pf{graphicx} \cite{graphics} & 1.0f & 1999/02/16 & đã kiểm tra\\
\pf{float} \cite{float} & 1.3d & 2001/11/08 & đã kiểm tra
\end{tabular}
\caption{Gói phụ thuộc}\label{tab:dependencies}
\end{table}

% =====================================================================

\subsection{\texorpdfstring{Cài đặt}{Cai dat}}

Bạn tải về các gói sau đây:

\noindent
\url{ftp://tug.ctan.org/pub/tex-archive/macros/latex/contrib/powerdot.zip}\\
\url{ftp://tug.ctan.org/pub/tex-archive/macros/latex/contrib/enumitem.zip}\\
\url{ftp://tug.ctan.org/pub/tex-archive/macros/latex/contrib/xcolor.zip}\\
\url{ftp://tug.ctan.org/pub/tex-archive/macros/latex/contrib/xkeyval.zip}\\
\url{ftp://ftp.jaist.ac.jp/pub/TeX/tex-archive/graphics/pstricks/latex/pstricks.sty}

Sau khi xả nén các tập tin |.zip| (vào cùng thư mục nào đó),
bạn có được cấu trúc thư mục như sau.
\begin{example}
  enumitem
  `-- enumitem.sty
  powerdot
  |-- doc
  |-- run
  |   |-- powerdot-*.sty
  |   `-- powerdot.cls
  `-- source
      `-- powerdot.dtx
  xcolor
  |-- xcolor.ins
  xkeyval
  |-- doc
  |-- run
  |   |-- *.tex
  |   |-- *.sty
\end{example}

Bạn phải làm công việc tương đối phức tạp sau đây: biên dịch 
tập tin |xcolor.ins|. Nếu chương trình soạn thảo của bạn không
hỗ trợ  việc biên dịch tập tin này, bạn thử chạy từ dòng lệnh
(tại thư mục |xcolor|)
\begin{example}
  latex xcolor.ins
\end{example}
Việc biên dịch này đảm bảo rằng trong thư mục |xcolor| ở trên
sẽ xuất hiện thêm các tập tin |.def|, |.sty| và |.pro|.

Bây giờ hãy tìm đến thư mục |LocalTeXMF| trên hệ thống của bạn.
Đó thường là |c:\localtexmf| đối với Mik\TeX{} hoặc là |~/texmf|
đối với te\TeX. Hãy tạo các thư mục sau bên dưới thư mục |LocalTeXMF|
\begin{example}
  tex/latex/xcolor
  tex/latex/enumitem
  tex/latex/xkeyval
  tex/latex/powerdot
  tex/latex/pstricks
  dvips/xcolor
\end{example}
Hãy chép các tập tin vào thư mục tương ứng
\begin{example}
  dvips/xcolor        <= xcolor/*.pro
  tex/latex/xcolor    <= xcolor/*.def xcolor/*.sty
  tex/latex/enumitem  <= enumitem/enumitem.sty
  tex/latex/xkeyval   <= xkeyval/run/*
  tex/latex/powerdot  <= powerdot/run/*
  tex/latex/pstricks  <= pstricks.sty
\end{example}
Cuối cùng, hãy cập nhật hệ thống \TeX{} của bạn.
Với te\TeX{}, bạn thi hành lệnh sau
\begin{command}
  $ texhash ~/texmf
\end{command}
còn với Mik\TeX{}, bạn chạy chương trình |mo.exe| (từ Menu |RUN|
chẳng hạn) rồi chọn `\textbf{Refresh now}' để cập nhật hệ thống \TeX.

% =====================================================================

%\subsection{Creating and viewing output}
\subsection{\texorpdfstring{Biên dịch và xem kết quả}{Bien dich va xem ket qua}}
\label{sec:creation}

%% To compile your presentation, run it with \LaTeX. The DVI that is
%% produced this way can be viewed with MiK\TeX's DVI viewer
%% YAP.\footnote{Unless you are using \pf{pstricks-add} which distorts
%% the coordinate system in DVI.} Unfortunately, xdvi and kdvi (kile)
%% do not support all PostScript specials and hence these will display
%% the presentation incorrectly. If your DVI viewer does support
%% this, make sure that your DVI display settings match that of the
%% presentation. In case you are using the |screen| paper, you should
%% set the DVI display setting to using the letter paper format. If
%% your DVI viewer allows for custom paper formats, use 8.25 inch by 11
%% inch.
Để có được trình diễn thực sự, đầu tiên, bạn biên dịch tài liệu nguồn
bằng \LaTeX. Kết quả |DVI| cho bởi bước này có thể xem bằng trình |YAP|
của Mik\TeX.\footnote{Trừ khi bạn sử dụng gói \pf{pstricks-add} làm ảnh
hưởng đến cách tính tọa độ của YAP.} Thật không may, các chương trình |xdvi|
và |kdvi| trên hệ thống |*nix| không hỗ trợ tập tin |DVI| có nhúng mã
lệnh |PostScript|, do đó, bạn không thể thấy kết quả như ý với các trình xem này.
Nếu trình xem |DVI| hỗ trợ |PostScript|, hãy chắc rằng các cài đặt của trình xem
đó khớp với các thuộc tính tương ứng của trình diễn. Khi bạn tạo ra trình diễn
|screen|, bạn nên cài đặt để trình xem hiển thị khổ xem |letter|.
Tốt hơn nữa, nếu trình xem cho phép, bạn hãy chỉnh cỡ trang xem là 8.25 inch
và 11 inch.

%% Note that certain things that are produced with PostScript or PDF
%% techniques will not work in a DVI viewer. Examples are hiding of
%% material via postscript layers (as is done, for instance, by
%% |\pause|, see section~\ref{sec:overlays}) and hyperlinks, for
%% instance in the table of contents.
Chú ý rằng, các kết quả, hiệu ứng tạo với mã lệnh |PostScript| hoặc |PDF|
không thể thấy được với trình xem |DVI|. Ví dụ, việc che giấu các phần
tử của danh sách, việc tạm dừng |slide| bằng lệnh |\pause|, \ldots

%% If you want to produce a postscript document, run dvips over the DVI
%% \emph{without any particular command line options related to
%% orientation or paper size}. \pf{powerdot} will write information to
%% the DVI file that helps dvips and ps2pdf (ghostscript) to create a
%% proper document. If you have some reason that this does not work for
%% you and you want to specify the paper and orientation yourself, you
%% should use the |nopsheaders| option that is described in
%% section~\ref{sec:setup}. The PostScript document could, for
%% instance, be used to put multiple slides on a page using the |psnup|
%% utility.
Nếu bạn muốn có kết quả |postscript| của tài liệu, sử dụng chương trình 
|dvips| để chuyển từ tập tin |DVI| thu đưọc sang sạng |PS|. \emph{Nhớ
đừng chỉ ra cỡ trang ghi gọi lệnh này}, bởi \pf{powerdot} đã ghi sẵn các
mã lệnh |PostScript| để định cỡ trang.% Xem thêm ở Mục~\vref{sec:setup}.

Cuối cùng, từ kết quả |PS|, bạn có thể có được kết quả |PDF| bằng cách
sử dụng chương trình chuyển đổi |ps2pdf|.\footnote{Hãy xem thêm về các chương
trình ps2pdf12, ps2pdf13, ps2pdf14. Theo mặc định, ps2pdf tương đương với ps2pdf12.}
Để ý rằng, chương trình |ps2pdf| cần sử dụng các công cụ của |PostScript|.
Trên hệ thống Slackware Linux chẳng hạn, sau khi cài te\TeX{},
bạn có thể cài thêm các gói |espgs| và |gnu-gs-fonts|
để đảm bảo chương trình |ps2pdf| làm việc tốt. Với hệ thống \TeX{} trên |Windows|,
bạn phải cài thêm chương trình |ghostscript| chẳng hạn.

Từ kết quả |PS|, bạn cũng có thể dùng chương trình |psnup|
để đặt nhiều |slide| lên cùng một trang.

Trong quá trình chuyển đổi |DVI| sang sạng |PS| rồi |PDF|,
nếu bạn gặp trục trặc với cỡ trang hoặc hướng trang,
thử dùng tuỳ chọn |nopsheader| như đã nói ở Mục~\vref{sec:classopts};
khi đó, bạn nhớ chỉ ra thiết lập của mình ở dòng lệnh khi gọi các chương trình
|dvips| hoặc |ps2pdf|.

%% To create a PDF document for your presentation, run ps2pdf over the
%% PS file created with dvips. Also here, you can \emph{leave out any
%% command line arguments related to paper size or orientation}. If
%% this is problematic for you somehow, use the |nopsheaders| option as
%% before and specify the paper and orientation at each intermediate
%% step yourself.

Tóm lại, nếu bạn có tài liệu |foo.tex|, thì quá trình biên dịch
để tạo trình diễn gồm các bước sau (dấu \$ chỉ dấu nhắc ở dòng lệnh)
\begin{command}
  $ latex foo
  $ latex foo
  $ dvips -o foo.ps foo.dvi
  $ ps2pdf foo.ps
\end{command}
(phải biên dịch một, hai hoặc nhiều lần hơn bằng |latex| để các tham khảo chéo
và bảng Mục lục được chính xác.)

%\section{Creating your own style}

\section{\texorpdfstring{Tạo kiểu mới}{Tao kieu moi}}
\label{sec:writestyle}

Mục này bàn về việc thiết kế kiểu mới cho \pf{powerdot}.

Mục này sẽ được dịch trong bản tiếp theo của tài liệu ;)

% =====================================================================

%\section{Using \LyX\ for presentations}
\section{\texorpdfstring{Tạo trình diễn với \LyX}{Tao trinh dien voi \LyX}%
\protect\footnote{Mục này được dịch chay ;) kyanh chưa có thời gian và điều kiện kiểm tra các hướng dẫn ở mục này.
Vui lòng feedback. Cám ơn!}}
\label{sec:lyx}

%% \LyX\ \cite{LyXWeb} is a WYSIWYM (What You See Is What You Mean)
%% document processor based on \LaTeX. It supports standard \LaTeX\
%% classes but needs special files, called layout files, in order to
%% support non-standard classes such as \pf{powerdot}.
\LyX\ là chương trình soạn thảo \LaTeX{} với nguyên lý WYSIWYM
(What You See Is What You Mean), chạy trên môi trường |*nix|.
\LyX\ hỗ trợ các lớp \LaTeX{} chuẩn; với các lớp mới, ví dụ \pf{powerdot},
\LyX\ cần thêm các tập tin đặc biệt, gọi là tập tin |layout|.

%% To start using \LyX\ for \pf{powerdot} presentations, copy the
%% layout file |powerdot.layout| to the \LyX\ layout directory. You can
%% find this file in the doc tree of your \LaTeX\ installation:
%% \url{texmf/doc/latex/powerdot}. If you can't find it there, download
%% it from \url{CTAN:/macros/latex/contrib/powerdot/doc}. Once that is
%% done, reconfigure \LyX\ (\texttt{Edit\LyXarrow Reconfigure} and
%% restart \LyX\ afterwards). Now you can use the \pf{powerdot}
%% document class as any other supported class. Go to
%% \texttt{Layout\LyXarrow Document} and select \texttt{powerdot
%% presentation} as document class. For more information, see the \LyX\
%% documentation, which is accessible from the |Help| menu.
Để có thể dùng \pf{powerdot} trong \LyX, trước hết, hãy chép tập tin |powerdot.layout|
vào thư mục của \LyX. Tập tin |layout| có thể tìm thấy trong thư mục
\url{texmf/doc/latex/powerdot} (khi vừa xả nén tập tin |powerdot.zip|).
Nếu bạn không tìm thấy ở đó, hãy tải về từ \url{CTAN:/macros/latex/contrib/powerdot/doc}.
Sau đó, hãy cấu hình lại cho \LyX{} nhờ sử dụng \texttt{Edit\LyXarrow Reconfigure}
rồi khởi động lại chương trình \LyX. Bây giờ bạn có thể thưởng thức
\pf{powerdot} trong \LyX ;) Để biết thêm chi tiết, tham khảo tài liệu
của \LyX{} (xem từ Menu |Help| của \LyX).

% =====================================================================

\subsection{\texorpdfstring{Cách dùng layout}{Su dung layout the nao}}% {How to use the layout}

%% \pf{powerdot} \LyX\ layout provides some environments\footnote{Don't
%% confuse these with \LaTeX\ environments.} which can be used in \LyX.
%% Some of these environments (for instance |Title| or |Itemize|) are
%% natural to use since they exist also in the standard document
%% classes such as \pf{article}. For more information on these standard
%% environments, see the \LyX\ documentation.
|layout| dành cho \LyX{} của \pf{powerdot} hỗ trợ vài môi trường (không phải
là môi trường trong \LaTeX{}). Một vài môi trường như |Title| và |Itemize|
là chuẩn, đã có trong lớp \pf{article}. Xem thêm tài liệu của \LyX{}
về các môi trường chuẩn này.

%% This section will explain how to use the \pf{powerdot} specific
%% environments |Slide|, |WideSlide|, |EmptySlide| and |Note|. These
%% environments correspond to the \pf{powerdot} environments |slide|,
%% |emptyslide|, |wideslide| and |note|.
Mục này giải thích cách dùng các môi trường riêng của \pf{powerdot}
trong \LyX, đó là |Slide|, |WideSlide|, |EmptySlide| và |Note|.
Các môi trường này lần lượt tương ứng với |slide|, |wideslide|,
|emptyslide| và |note| (môi trường \LaTeX{} của \pf{powerdot}).

%We start with a simple example. The following \LaTeX\ code
Hãy bắt đầu bằng ví dụ đơn giản sau, tương ứng với đoạn mã \LaTeX{}
\begin{example}
 \begin{slide}{Slide title}
   Slide content.
 \end{slide}
\end{example}
%% can be obtained using the following \LyX\ environments. The right
%% column represents the text typed into the \LyX\ window and the left
%% column represents the environment applied to this text).
trong \LyX{} ta dùng các đoạn mã như sau (cột phải chỉ ra nội dung của |slide|,
cột bên trái là môi trường \LyX):
\begin{example}
  Slide         Slide title
  Standard      Slide content.
  EndSlide
  ...
\end{example}
%Some remarks concerning this example.
Có vài lưu ý liên quan đến ví dụ này
\begin{itemize}[leftmargin=0pt,itemsep=0pt,parsep=0pt]
\item
%% You can use the environment menu (under the menu bar, top-left
%% corner) to change the environment applied to text.
Bạn có thể dùng môi trường |menu| (dưới thanh công cụ |menu| nằm ở
góc trên bên trái của \LyX) để thay đổi môi trường cho phần nội dung được chọn.
\item
%%  The slide title should be typed on the line of the |Slide|
%% environment.
Tựa của |slide| (ở trên là |Slide title|) nên đặt trên cùng một dòng với
từ khóa |Slide| bắt đầu môi trường \LyX.
\item% |EndSlide| finishes the slide and its line is left blank.
|EndSlide| kết thúc |slide|, theo sau là khoảng trắng.
\end{itemize}

%% In the \LyX\ window, the |Slide| environment (that is, the slide
%% title) is displayed in magenta, the |WideSlide| style in green, the
%% |EmptySlide| style in cyan and the |Note| style in red and hence
%% these are easily identifiable.
Ở của sổ của \LyX, môi trường |Slide| (cùng với tựa đề của |slide|)
được bắt đầu với màu |magenta|, trong khi màu cho |WideSlide| là |green|,
cho |EmptySlide| là |cyan| và cho |Note| là đỏ. Rất dễ phân biệt, nhỉ ;)

%Here is another example.
Dưới đây là ví dụ khác (trước hết là mã \LaTeX{})
\begin{example}
 \begin{slide}{First slide title}
   The first slide.
 \end{slide}
 \begin{note}{First note title}
   The first note, concerning slide 1.
 \end{note}
 \begin{slide}{Second slide title}
   The second slide.
 \end{slide}
\end{example}
%This can be done in \LyX\ in the following way.
Tương ứng, ta có mã \LyX{} như sau đây:
\begin{example}
 Slide         First slide title
 Standard      The first slide.
 Note          First note title
 Standard      The first note, concerning slide 1.
 Slide         Second slide title
 Standard      The second slide.
 EndSlide
\end{example}
%% This example demonstrates that it is often sufficient to insert the
%% |EndSlide| style after the last slide or note only. Only when you
%% want certain material not to be part of a slide, you need to finish
%% the preceding slide manually using the |EndSlide| style. Example:
Ví dụ này cho thấy rằng chỉ cần dùng |EndSlide| một lần cho nhiều |slide|
kế tiếp nhau. Chỉ khi nào cần thêm nội dung không thuộc vào |slide| nào cả,
bạn mới phải dùng thêm |EndSlide| như ví dụ sau đây:
\begin{example}
 Slide         First slide title
 Standard      The first slide.
 EndSlide
 [ERT box with some material]
 Slide         Second slide title
 ...
\end{example}
%% You could use this, for instance, to have verbatim material on
%% slides (see also section~\ref{sec:FAQ}).
Trường hơp này xảy ra, ví dụ khi bạn cần sử dụng môi trường |verbatim|
với |slide|. Xem thêm ở Mục~\vref{sec:FAQ}.

\subsection{\texorpdfstring{Cú pháp được hỗ trợ}{Cu phap duoc ho tro}} % {Support of syntax}

%% This section lists options, commands and environments that are
%% supported through the \LyX\ interface directly, without using an ERT
%% box (\TeX-mode).
Mục này liệt kê các tùy chọn, lệnh và môi trường mà \pf{powerdot}
hỗ trợ cho \LyX{} một cách trực tiếp, không phải thông qua việc sử dụng
|ERT| |box| (hay chế độ \TeX).

%% All class options (see section~\ref{sec:classopts}) are supported
%% via the \texttt{Layout\LyXarrow Document} dialog (|Layout| pane).
%% Options for the |\pdsetup| command (see section~\ref{sec:setup})
%% should be specified in the |Preamble| pane of the
%% \texttt{Layout\LyXarrow Document} dialog.
Mọi tùy chọn cho lớp (Mục~\vref{sec:classopts}) được hỗ trợ thông
qua hộp thoại \texttt{Layout\LyXarrow Document}. Các tùy chọn của lệnh |\pdsetup|
đã nói ở Mục~\vref{sec:setup} được thay đổi thông qua phần |Preamble|
tìm thấy ở hộp thoại \texttt{Layout\LyXarrow Document} của \LyX.

%% Table \vref{tab:lyxcommands} lists the \pf{powerdot} commands that
%% are supported in \LyX.
Bảng~\vref{tab:lyxcommands} liệt kê các lệnh \pf{powerdot} hỗ trợ cho \LyX.
\begin{table}[htb]
\centering
\begin{tabular}{r|l}
Lệnh & Tương ứng trong \LyX\\\hline
\cs{title} & môi trường \texttt{Title}\\
\cs{author} & môi trường \texttt{Author}\\
\cs{date} & môi trường \texttt{Date} \\
\cs{maketitle} & điều khiển bởi \LyX.\\
\cs{section} & môi trường \texttt{Section}.
%% Options to this command (see section~\ref{sec:section}) can be specified using
%% \texttt{Insert\LyXarrow Short title} in front of the section title.\\
Tuỳ chọn cho lệnh này (xem Mục~\ref{sec:section}) được thay đổi\\
& thông qua \texttt{Insert\LyXarrow Short title} ở trước tựa của |slide|.\\
\cs{tableofcontents} & dùng \texttt{Insert\LyXarrow Lists \& TOC\LyXarrow Table of contents}.
\\&Cần dùng |ERT box| nếu lệnh này được dùng với tùy chọn. Xem bên dưới.
\end{tabular}
%\caption{Supported \pf{powerdot} commands in \LyX}
\caption{Lệnh \pf{powerdot} được hỗ trợ trong \LyX}
\label{tab:lyxcommands}
\end{table}
%% Table \vref{tab:lyxenvs} lists the \pf{powerdot} environments that,
%% besides the earlier discussed |slide|, |wideslide|, |note| and
%% |emptyslide| environments, are supported in \LyX.
Bảng~\vref{tab:lyxenvs} liệt kê các môi truờng \pf{powerdot}
được hỗ trợ trong \LyX, bên cạnh các môi trường |slide|, |wideslide|,
|emptyslide|, |note| đã nói.
\begin{table}[htb]
\centering
\begin{tabular}{r|l}
môi trường & tương ứng trong \LyX\\\hline
\texttt{itemize} & môi trường \texttt{Itemize} hoặc \texttt{ItemizeType1}.
\\& |Type1| tương ứng với |type=1| nói ở Mụ~\vref{sec:lists}).\\
\texttt{enumerate} & môi trường \texttt{Enumerate} hoặc \texttt{EnumerateType1}.\\
\texttt{thebibliography} & môi trường \texttt{Bibliography}.
\end{tabular}
%\caption{Supported \pf{powerdot} environments in \LyX}
\caption{Môi trường \pf{powerdot} được hỗ trợ trong \LyX}
\label{tab:lyxenvs}
\end{table}
%% Table \vref{tab:lyxERT} lists commands that can only be done by using
%% an ERT box (via \texttt{Insert\LyXarrow TeX}).
Bảng~\vref{tab:lyxERT} liệt kê các lệnh chỉ dùng được bằng cách sử dụng |ERT| |box|
(thông qua \texttt{Insert\LyXarrow TeX}).
\begin{table}[htb]
\centering
\begin{tabular}{r|l}
lệnh & tương ứng trong \LyX\\\hline
\cs{and} & bên trong môi trường \texttt{Author}.\\
\cs{pause} & \\
\cs{item} & %An ERT box is only required for the optional argument,
%not mandatory for overlays specifications.\\
ERT box chỉ cần thiết khi dùng tham số tùy chọn,\\
& tham số bắt buộc thì không cần ERT box\\
\cs{onslide} & cùng với \cs{onslide+} và \cs{onslide*}\\
\cs{twocolumn} & \\
\cs{tableofcontents} & chỉ dùng khi có tham số tuỳ chọn.
\end{tabular}
%\caption{\pf{powerdot} commands needing an ERT box in \LyX}
\caption{Lệnh \pf{powerdot} cần ERT box của \LyX}
\label{tab:lyxERT}
\end{table}
%% Note that you may use the clipboard in order to repeat often used
%% commands like |\pause|. Finally, table \vref{tab:lyxadd} lists
%% additional commands and environments that are supported by the layout.
Bạn có thể dùng bộ nhớ đệm |clipboard| để giữ các lệnh hay dùng như |\pause|.
Cuối cùng, Bảng~\vref{tab:lyxadd} liệt kê các lệnh và môi trường khác
được hỗ trợ trong \LyX.
\begin{table}[htb]
\centering
\begin{tabular}{r|l}
lệnh/môi trường & tương tứng trong \LyX\\\hline
\texttt{quote} & môi trường \texttt{Quote}.\\
\texttt{quotation} & môi trường \texttt{Quotation}.\\
\texttt{verse} & môi trường \texttt{Verse}.\\
\cs{caption} & môi trường \texttt{Caption} với float chuẩn
\end{tabular}
%\caption{Additional environments for \LyX}
\caption{Lệnh,môi trường bổ sung cho \LyX}
\label{tab:lyxadd}
\end{table}

\subsection{\texorpdfstring{Biên dịch với LyX}{Bien dich voi \LyX}} %{Compiling with \LyX}

%% First of all, make sure that you have also read
%% section~\ref{sec:compiling}. Then, in order to get a proper
%% PostScript or PDF file, you have to set your \LyX\ document
%% properties depending on which paper and orientation you want. When
%% your \LyX\ document is open, go to the \texttt{Layout\LyXarrow
%% Document} dialog. In the \texttt{Layout} pane, put the |nopsheader|,
%% |orient| and |paper| keys as class options (see
%% section~\ref{sec:classopts} for a description). Then, go to the
%% |Paper| pane and select corresponding paper size and orientation
%% (you may choose |letter| paper in the case you set |paper=screen| in
%% the class options). Finally, go to the |View| (or
%% \texttt{File\LyXarrow Export}) menu and select your output
%% (PostScript or PDF).
Trước hết, cần nắm được các ý ở Mục~\vref{sec:compiling}.
Để có kết quả |PostScript| hoặc |PDF| như ý với \LyX, cần phải
đặt cài đặt hợp lý cho tài liệu \LyX{}: kiểu giấy, hướng giấy.
Khi tài liệu \LyX{} được mở, hãy mở hộp thoại \texttt{Layout\LyXarrow Document}.
Ở phần \texttt{Layout}, thiết lập cho các tuỳ chọn |nopsheader|, |orient| và |paper|
(theo các đã trình bày ở Mục~\vref{sec:classopts}). Sau đó, ở phần |Paper|,
chọn khổ giấy và hướng giấy tương ứng với thiết lập ấy. Có thể chọn
|letter| nếu dùng tuỳ chọn |paper=screen|. Cuối cùng, trong menu |View|
hoặc (\texttt{File\LyXarrow Export}), chọn kết quả |PostScript| hoặc |PDF|.

%% \subsection{Extending the layout}
%% If you have created a custom style (see section~\ref{sec:writestyle})
%% which defines custom templates, you may want to extend the layout
%% file\footnote{The LPPL dictates to rename a file if you modify it as
%% to avoid confusion.} so that these templates are also supported in
%% \LyX. The explanation below assumes that you have defined a template
%% called |sunnyslide|.
%% 
%% To support this new template in \LyX, you have to use the following
%% command.
%% \begin{command}
%%  `\cs{pddefinelyxtemplate}\meta{cs}\marg{template}'
%% \end{command}
%% \DescribeMacro{\pddefinelyxtemplate}
%% This will define the control sequence \meta{cs} such that it will
%% create a slide with template \meta{template} (which has been defined
%% using |\pddefinetemplate|. This new control sequence can be used in
%% the layout file as follows.
%% \begin{example}
%%  # SunnySlide environment definition
%%  Style SunnySlide
%%    CopyStyle     Slide
%%    LatexName     lyxend\lyxsunnyslide
%%    Font
%%      Color       Yellow
%%    EndFont
%%    Preamble
%%      \pddefinelyxtemplate\lyxsunnyslide{sunnyslide}
%%    EndPreamble
%%  End
%% \end{example}
%% Note that you must begin the |LatexName| field with |lyxend|. The
%% definition of the \LyX\ template has been inserted in between
%% |Preamble| and |EndPreamble| which assures that the new \LyX\
%% environment will work. After modifying the layout file, don't forget
%% to restart \LyX. See for more information about creating \LyX\
%% environments, the documentation of \LyX\ in the |Help| menu.

% =====================================================================

\section{\texorpdfstring{Hỏi-Đáp}{Hoi-Dap}}

\label{sec:questions}

% =====================================================================

\subsection{\texorpdfstring{Câu hỏi thường gặp}{Cau hoi thuong gap}}

\label{sec:FAQ}

%% This section is devoted to Frequently Asked Questions. Please read
%% it carefully; your problem might be solved by this section.
Dưới đây là các câu hỏi thường gặp. Hãy đọc kỹ, vì trục trặc của bạn
có thể tìm thấy lời giải ở đây.
\begin{itemize}[leftmargin=0pt]
\question
%Does \pf{powerdot} have example files? Where can I find them?
Lớp \pf{powerdot} có cung cấp ví dụ? Tìm chúng ở đâu?
\answer
%% \pf{powerdot} comes with several examples that should be in the doc
%% tree of your \LaTeX\ installation. More precisely:
%% \url{/doc/latex/powerdot}. If you can't find them there, download
%% them from \url{CTAN:/macros/latex/contrib/powerdot} \cite{CTAN}.
Trong thư mục tài liệu |doc| của bộ cài đặt \LaTeX, bạn có thể
tìm thấy các ví dụ cho \pf{powerdot}. Nếu không thấy ở đó, có thể
tải chúng về từ \url{CTAN:/macros/latex/contrib/powerdot} \cite{CTAN}.
\question %I'm getting errors on the simplest example!
Sao tôi gặp lỗi chỉ với ví dụ đơn giản nhất?
\answer %Did you read section~\ref{sec:dependencies}?
Bạn đã đọc Mục~\vref{sec:dependencies} chưa?
\question %% I made a typo in the slide code, ran the file, got an
%% error, corrected the typo and reran, but now get an error that
%% doesn't go away.
Tôi gặp lỗi |typo| trong tài liệu nguồn, do đó gặp lỗi khi biên dịch
tài liệu. Sau đó, tôi đã sửa lỗi |typo| đó và biên dịch, nhưng lỗi đã
gặp vẫn xuất hiện.
\answer %Remove the |.bm| and |.toc| files and try again.
Hãy xóa các tập tin |.bm| và |.toc| và thử lại xem.
\question
%My figure/table produces the error: \texttt{Not in outer par mode}.
Các môi trường |figure| và |table| sinh ra lỗi: \texttt{Note in outer par mode}.
\answer
%You have asked \LaTeX\ to float the figure or table using something like
Bạn đã yêu cầu \LaTeX{} dùng các môi trường |float|, tương tự như ví dụ sau
\begin{example}
 \begin{figure}[htb]
\end{example}
%% \LaTeX\ has nowhere to float the figure or table to. Remove the
%% optional argument (here |[htb]|) and the figure or table will work.
Nguyên nhân sinh ra lỗi là \LaTeX{} không có chỗ để bố trí |float|.
Vì thế, hãy thử bỏ đi tham số của môi trường |figure| hoặc |table|,
trong ví dụ trên là |[htb]|.
\question
%How can I put verbatim text on slides?
Làm thế nào để dùng môi trường |verbatim| trong |slide|?
\answer
%% You can do this by using a temporary box to store the material
%% in or, for instance, the \pf{fancyvrb} package which has a special
%% environment for this. See two examples below.
Lưu nội dung của |verbatim| tạm thời vào một |box| hoặc sử dụng gói \pf{fancyvrb}.
Hãy xem ví dụ sau đây\\
\begin{minipage}[t]{.49\linewidth}
\begin{example}
  \documentclass{powerdot}
  \usepackage{listings}
  \newsavebox\someverb
  \begin{document}
  \begin{lrbox}{\someverb}
    \begin{lstlisting}
      Some text.
    \end{lstlisting}
  \end{lrbox}
  \begin{slide}{Listing}
    \usebox\someverb
  \end{slide}
  \end{document}
\end{example}
\end{minipage}\hfill
\begin{minipage}[t]{.49\linewidth}
\begin{example}
  \documentclass{powerdot}
  \usepackage{fancyvrb}
  \begin{document}
    \begin{SaveVerbatim}{someverb}
      Some text.
    \end{SaveVerbatim}
    \begin{slide}{Verbatim}
      \BUseVerbatim{someverb}
    \end{slide}
  \end{document}
\end{example}
\end{minipage}
\question
% |\pause| does not work in the |align| environment.
Lệnh |\pause| không làm việc bên trong môi trường |align|?
\answer
%% |align| does several tricky things, which make it impossible
%% to use |\pause|. Use |\onslide| instead. See section~\ref{sec:onslide}.
Môi trường |align| có vài xử lý đặc biệt không tương thích với |\pause|.
Sử dụng |\onslide| để thay thế cho |\pause|. Xem Mục~\vref{sec:onslide}.
\question
%Can I contribute to this project?
Tôi có thể đóng góp cho dự án \pf{powerdot}?
\answer
%% Certainly. If you find bugs\footnote{Make sure that you confirm that
%% the bug is really caused by \pf{powerdot} and not by another package
%% that you use.} or typos, please send a message to the mailinglist
%% (see section~\ref{sec:mailinglist}), If you have developed your own
%% style and would like to see it included in \pf{powerdot}, please
%% inform us by private e-mail. Notice that contributions will fall
%% under the overall \pf{powerdot} license and copyright notice, but
%% that your name will be included in the documentation when you make a
%% contribution.
Tất nhiên. Nếu bạn gặp (lỗi) |BUGS|\footnote{Hãy chắc rằng đó là lỗi do
\pf{powerdot} sinh ra chứ không phải bởi gói khác.} hoặc lỗi |typo|, vui lòng
gửi thư đến |mailinglist| (xem Mục~\vref{sec:mailinglist}). Nếu bạn thiết
kế kiểu trình diễn riêng và muốn chúng có trong bản phân phối của \pf{powerdot},
vui lòng gửi email riêng cho tác giả. Chú ý rằng, tác phẩm của bạn sẽ được
phân phối theo |license| và |copyright| của \pf{powerdot}; tên của bạn
sẽ được nêu trong tài liệu của \pf{powerdot}.
\end{itemize}

%% If your question has not been answered at this point, advance to the
%% next section where to find more answers.
Nếu bạn có các câu hỏi khác, không nằm trong danh sách trên, hãy xem
Mục kế tiếp để biết cách tìm câu trả lời.

% =====================================================================

\subsection{Mailinglist}

\label{sec:mailinglist}

\pf{powerdot}
%%has a mailinglist from \url{freelists.org} and has its
%% website here:
có |mailinglist| riêng ở
\begin{center}
\url{http://www.freelists.org/list/powerdot}
\end{center}
%% There is a link to `List Archive'. Please search this archive before
%% posting a question. Your problem might already have been solved in
%% the past.
Với vấn đề bạn gặp phải, thử tìm kiếm ở phần lưu trữ (|archive|)
của |mailinglist|.
%%
%% If that is not the case, use the box on the page to type your e-mail
%% address, choose the action `Subscribe' and click `Go!'. Then follow
%% the instructions that arrive to you by e-mail. At a certain moment,
%% you can login for the first time using an authorization code sent to
%% you by e-mail. After logging in, you can create a password for
%% future sessions using the `Main Menu' button. The other buttons
%% provide you some info and options for your account.
Nếu không tìm ra, bạn có thể đăng ký làm thành viên của |mailinglist|
bằng cách vào phần |subscribe| và nhấn nút |go|. Hãy kiểm tra hộp thư
của bạn vào theo các hướng dẫn trong đó. Sau khi là thành viên của
|mailinglist|, bạn có thể gửi các câu hỏi mới, bằng cách gửi email tới
%địa chỉ
%When you are all set, you can write to the list by sending an e-mail to
%\begin{center}
\url{powerdot@freelists.org}
%\end{center}

%% When writing to the list, please keep in mind the following very
%% important issues.
Khi gửi câu hỏi đến |mailinglist|, hãy nhớ kỹ những điều quan trọng sau
\begin{dinglist}{44}%[leftmargin=0pt]
\item %We are volunteers!
Trả lời cho bạn là những người tình nguyện.
\item %Always supply a \emph{minimal} example demonstrating your problem.
Luôn cung cấp ví dụ đơn giản nhất có thể để minh họa cho trục trặc của bạn.
\item %Don't send big files over the list.
Không gửi kèm các tập tin kích thước lớn đến |mailinglist|.
\end{dinglist}

%We hope you will enjoy this service.
Hy vọng rằng |mailinglist| sẽ giúp ích cho bạn.

% =====================================================================

%\section{Source code documentation}
\section{\texorpdfstring{Mã nguồn}{Ma nguon}}

% =====================================================================

\subsection{\texorpdfstring{Mã nguồn}{Ma nguon} powerdot}
\label{ssec:sande}

%% In case you want regenerate the package files from the source or
%% want to have a look at the source code description, locate
%% |powerdot.dtx|, search in the file for |\OnlyDescription| and remove
%% that and do
Mã nguồn của lớp \pf{powerdot} là tập tin |powerdot.dtx|.
Biên dịch mã nguồn này bạn thu được tài liệu đầy đủ nhất về \pf{powerdot},
kể cả mã \LaTeX{} của lớp với các chú thích kỹ thuật.

Bạn có thể biên dịch mã nguồn này theo các bước sau:
\begin{command}
  latex powerdot.dtx
  latex powerdot.dtx
  bibtex powerdot
  makeindex -s gglo.ist -o powerdot.gls powerdot.glo
  makeindex -s gind.ist -o powerdot.ind powerdot.idx
  latex powerdot.dtx
  latex powerdot.dtx
\end{command}

% =====================================================================

\subsection{\texorpdfstring{Mã nguồn}{Ma nguon} powerdot-doc-vi}

Mã nguồn tài liệu \pf{powerdot-doc-vi} có trong tập tin |powerdot-doc-vi-src-*.zip|.
Có hai tập tin |*.tex| là |powerdot-doc-vi.tex| và |powerdot-doc-vi-tcvn.tex|.
Bạn có thể biên dịch tập tin đầu tiên nếu gói |vntex| trên hệ thống của
bạn hỗ trợ |UTF-8| (có thể bạn phải thay |\usepackage[utf8x]| bởi |\usepackage[utf8]|).
Tập tin còn lại dùng với hỗ trợ |TCVN| của |vntex|.

Việc biên dịch tập tin (|foo.tex|) để có bản |PDF| như sau:
\begin{command}
  latex foo
  latex foo
  latex foo
  dvips -ofoo.ps foo.dvi
  ps2pdf foo.ps
\end{command}

Nếu bạn muốn có bản in (hai màu đen trắng), trước khi biên dịch như trên,
tạo tập tin |printctl.tex| như sau:
\begin{command}
  echo '\printtrue' > printctl.tex
\end{command}

Tập in |Makefile| đi kèm với mã nguồn chỉ dành cho mục đích tham khảo!
Nó có thể hoạt động không như ý, thậm chí phá hỏng hệ thống của bạn!

% =====================================================================

\newpage

\addcontentsline{toc}{section}{\texorpdfstring{Tài liệu}{Tai lieu}}

\small

\begin{thebibliography}{10}
\bibitem{VietTUG}
ViệtTUG.
\newblock {Cộng đồng người Việt sử dụng \TeX{}}.
\newblock \url{http://www.viettug.org/}.

\bibitem{HA-prosper}
Hendri Adriaens.
\newblock Gói \pf{HA-prosper}.
\newblock \url{CTAN:/macros/latex/contrib/HA-prosper}.

\bibitem{xkeyval}
Hendri Adriaens.
\newblock Gói \pf{xkeyval}.
\newblock \url{CTAN:/macros/latex/contrib/xkeyval}.

\bibitem{enumitem}
Javier Bezos.
\newblock Gói \pf{enumitem}.
\newblock \url{CTAN:/macros/latex/contrib/enumitem}.

\bibitem{graphics}
David Carlisle.
\newblock Họ gói \pf{graphics}.
\newblock \url{CTAN:/macros/latex/required/graphics}.

\bibitem{CTAN}
CTAN crew.
\newblock {The Comprehensive TeX Archive Network}.
\newblock \url{http://www.ctan.org}.

\bibitem{natbib}
Patrick~W. Daly.
\newblock Gói \pf{natbib}.
\newblock \url{CTAN:/macros/latex/contrib/natbib}.

\bibitem{prosper}
Fr\'ed\'eric Goualard và Peter~M\o ller Neergaard.
\newblock Lớp \pf{prosper}.
\newblock \url{CTAN:/macros/latex/contrib/prosper}.

\bibitem{xcolor}
Uwe Kern.
\newblock Gói \pf{xcolor}.
\newblock \url{CTAN:/macros/latex/contrib/xcolor}.

\bibitem{extsizes}
James Kilfiger and Wolfgang May.
\newblock Họ gói \pf{extsizes}.
\newblock \url{CTAN:/macros/latex/contrib/extsizes}.

\bibitem{float}
Anselm Lingnau.
\newblock Gói \pf{float}.
\newblock \url{CTAN:/macros/latex/contrib/float}.

\bibitem{companion}
Frank Mittelbach, Michel Goossens, Johannes Braams, David Carlisle, and Chris
  Rowley.
\newblock {\em The {\LaTeX} {C}ompanion, {S}econd {E}dition}.
\newblock {Addison-Wesley}, 2004.

\bibitem{hyperref}
Sebastian Rahtz và Heiko Overdiek.
\newblock Gói \pf{hyperref}.
\newblock \url{CTAN:/macros/latex/contrib/hyperref}.

\bibitem{geometry}
Hideo Umeki.
\newblock Gói \pf{geometry}.
\newblock \url{CTAN:/macros/latex/contrib/geometry}.

\bibitem{PSTricksWeb}
Herbert Vo\ss.
\newblock Trang chủ \pf{PSTricks}.
\newblock \url{http://pstricks.tug.org}.

\bibitem{PSTricks}
{Timothy Van} {Zandt et al.}
\newblock Gói \pf{PSTricks}, v1.07, 2005/05/06.
\newblock \url{CTAN:/graphics/pstricks}.

\bibitem{LyXWeb}
LyX crew.
\newblock Trang chủ \pf{LyX}.
\newblock \url{http://www.lyx.org}.
\end{thebibliography}

\typeout{***************************************}
\typeout{* (c) 2005 kyanh <kyanh at o2 dot pl> *}
\typeout{***************************************}

\end{document}
