%% $Id$ %%

\section*{\texorpdfstring{Chú ý}{Chu y}}
\addcontentsline{toc}{section}{\texorpdfstring{Chú ý}{Chu y}}

% =====================================================================

\subsection*{\texorpdfstring{So sánh các phiên bản}{So sanh cac phien ban}}
\addcontentsline{toc}{subsection}{\texorpdfstring{So sánh các phiên bản}{So sanh cac phien ban}}

Mục này điểm qua các thay đổi quan trọng nhất của |powerdot|.
Chú ý đến các thay đổi được đánh dấu \ding{53}.

\begin{description}
\item[1.3] Cập nhật ngày \pddate.
	\begin{dinglist}{51}
	\item[\ding{53}] Thêm tuỳ chọn |clock|, |method|
	\item[\ding{53}] Thêm hỗ trợ |palette|. Xem Mục~\ref{sec:gopts}
	\item[\ding{53}] Thêm hiệu ứng dấu chấm ngẫu nhiên
	Xem Mục~\ref{sec:glopts}
	\item Thêm tuỳ chọn |logohook|,|logopos|,|logocmd|
	\item Thêm tuỳ chọn cho lệnh |\maketitle|
	\item Thêm các kiểu: |horatio|, |paintings|, |pazik|, |jefka|, |klope|
	\item Thêm |palette| cho các kiểu |default|, |fyma| và |sailor|
	\end{dinglist}
\item[1.2] Cập nhật ngày 09/10/2005.
	\begin{dinglist}
	{51}
	\item Thêm kiểu |upen| và |bframe|.
	\item Tăng tốc độ biên dịch.
	\end{dinglist}
\item[1.1] Cập nhật ngày 20/9/2005.
	\begin{dinglist}{51}
	\item[\ding{53}] phải dùng |size=12pt| thay vì |size=12| khi xác định
	cỡ chữ khi khai báo lớp. (Các cỡ khác tương tự.) Xem Mục~\vref{sec:classopts}.
	\item Thêm các kiểu trình diễn |elcolors|, |aggie|, |husky|, |sailor|.
	Các kiểu |tycja|, |ciment|, |fyma| được cải tiến. Xem Mục~\vref{sec:styles}.
	\item Hỗ trợ \LyX.
	% Xem Mục~\vref{sec:lyx}.
	\end{dinglist}
\item[1.0] Phiên bản đầu tiên, công bố ngày 5/9/2005.
\end{description}


% =====================================================================

\subsection*{\texorpdfstring{Ghi chú của kyanh}{Ghi chu cua kyanh}}
\addcontentsline{toc}{subsection}{\texorpdfstring{Ghi chú của kyanh}{Ghi chu cua kyanh}}

Tài liệu này được biên dịch từ tài liệu chính thức của lớp \pf{powerdot}.
Có nhiều phần được thêm vào, một số phần khác được lược bớt.
Riêng Mục~\vref{sec:writestyle} chưa được dịch vì chưa cần thiết lắm
cho người dùng bình thường.

%% \medskip
%% \begin{version}{1.1}
%% Một số tùy chọn, hướng dẫn,... chỉ phù hợp với phiên bản nhất định của \pf{powerdot}.
%% Ví dụ, dòng này chỉ phù hợp với phiên bản 1.1 (về sau) của \pf{powerdot}.
%% \end{version}

%% \medskip
%% Nếu bạn muốn in (đen-trắng) tài liệu này, sử dụng tập tin |powerdot-doc-vn-print.pdf|.
%% 
\medskip
Bản mới nhất của tài liệu này có thể tìm thấy tại
\url{http://download.viettug.org/}
%% Là người dùng Việt,
%% bạn có thể tìm kiếm sự giúp đỡ tại \url{http://www.viettug.org/}.

%% \medskip
%% Tài liệu này (bản |PDF|) được phân phối cùng với các ví dụ và hình ảnh như sau
%% đây.
%% \begin{description}
%% \item[img/]
%% \item[\ding{42} lst-bookmarks.png] Danh sách |bookmark| của một trình diễn
%% \item[\ding{42} tab-contents.png] Bảng Mục lục ở một |slide|
%% \item[\ding{42} tab-slide-contents.png] Slide Mục lục và Bảng Mục lục
%% \item[exa/]
%% \item[\ding{42} example-1.tex] Ví dụ về \pf{powerdot} (nguồn).
%% \item[\ding{42} example-1.pdf] Ví dụ về \pf{powerdot} (PDF).
%% \end{description}
%% 
%% \medskip
%% Các hình ảnh minh họa các cho kiểu trình diễn
%% có thể tìm thấy trong tập tin nén |powerdot-dot-styles.zip|
%% (tải về ở \url{http://www.viettug.org/}).

\endinput
